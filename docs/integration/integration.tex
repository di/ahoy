\RequirePackage{fix-cm}
\documentclass[titlepage]{article}

\usepackage[utf8]{inputenc}
\usepackage{fullpage}    % Use the whole page
\usepackage{fancyhdr}    % Nice headers/footers
\usepackage{graphicx}    % Importing graphics
\usepackage{hyperref}    % Hyperlink references and URLs
\usepackage[figure,table]{hypcap} % Hyperlink points to the top of figures
\usepackage[usenames,dvipsnames]{xcolor}	% Logo
\usepackage{tikz,ifthen}			% Logo
\usepackage{pgf}				% Logo
\usepackage{scalefnt}				% Logo
\usepgfmodule{shapes}				% Logo
\usepgfmodule{plot}				% Logo
\usetikzlibrary{shapes,arrows,shadows,fit}
\usepackage{pgf-umlsd}
\usepackage{multirow}
\usepackage{mdwlist}
\usepackage{colortbl}
\usepackage{calc}
\usepackage{float}
\usepackage{longtable}
\usepackage{amsmath}
\usepackage{appendix}
\usepackage{listings}   %for displaying code
\usepackage{rotating}	% for sideways column headings in table

\renewenvironment{itemize*}
    {\begin{itemize}
        \setlength{\itemsep}{0pt}%
        \setlength{\parskip}{0pt}%
        \setlength{\partopsep}{0pt}%
        \setlength{\topsep}{0pt}}%
    {\end{itemize}}

\newcommand{\testcase}[3]{
    \begin{center}
    \begin{tabular}{| l | p{0.7\textwidth}|}
        \hline
        \rowcolor[gray]{0.8}\textbf{Pre-Condition:} & #1 \\ \hline
        \textbf{Action:} & #2 \\ \hline
        \rowcolor[gray]{0.8}\textbf{Post-Condition:} & #3 \\ \hline
    \end{tabular}
    \end{center}
}

% Just so we don't have to specify this twice
\newcommand\mytitle{Integration Test Plan}
\newcommand\mydate{\today}
\newcommand\myversion{1}

\hypersetup{
    colorlinks=true,
    linkcolor=blue,
    urlcolor=blue,
    pdftitle={AHOY \mytitle V\myversion},
    pdfauthor={Dustin Ingram, Aaron Rosenfeld, Maria Kolakowska, Frank Clark}
}

% To make referencing sections more biggerer and clickyer 
\newcommand{\rrref}[2]{\hyperref[#2]{#1}}
\newcommand{\sref}[1]{\hyperref[#1]{Section~\ref*{#1}}}
\newcommand{\fref}[1]{\hyperref[#1]{Figure~\ref*{#1}}}
\newcommand{\aref}[1]{\hyperref[#1]{Appendix~\ref*{#1}}}

% So we can number subsubsections too
\setcounter{secnumdepth}{5}

% For headers and footers
\setlength{\headheight}{15pt}
\setlength{\headsep}{25pt}
\pagestyle{fancy}
	
% Page style for the title page
\fancypagestyle{plain}{
    \fancyhf{}
    \renewcommand{\headrulewidth}{0pt}
    \renewcommand{\footrulewidth}{0pt}
}

% Page style for every other page
\fancyhf{} % clear all header and footer fields
\fancyhead[L]{AHOY}
\fancyhead[C]{\mytitle}
\fancyhead[R]{\mydate}
\fancyfoot[C]{\thepage}
\renewcommand{\headrulewidth}{0.4pt}
\renewcommand{\footrulewidth}{0.4pt}

\title{\textbf{\mytitle}}
\author{
	Frank Clark \\\url{francis.j.clark@drexel.edu}
    \and Dustin Ingram \\\url{dustin.s.ingram@drexel.edu}
	\and Maria Kolakowska \\\url{maria.j.kolakowska@drexel.edu}
    \and Aaron Rosenfeld \\\url{aaron.rosenfeld@drexel.edu}
}
\date{\mydate\\Version \myversion}
\begin{document}
\pagenumbering{roman}

\begin{figure*}
   % \vspace{-2em}
    \centering
    \scalebox{0.8}{
\begin{tikzpicture}[scale=1]
	
	\pgfsetlinewidth{3pt}

	% Background
	\color{cyan!70!black}
	\pgfpathmoveto{\pgfpointxy{-5}{2}}
	\pgfpathlineto{\pgfpointxy{-5}{11}}
	\pgfpathlineto{\pgfpointxy{-2}{11.9}}	
	\pgfpathlineto{\pgfpointxy{2}{11.9}}	
	\pgfpathlineto{\pgfpointxy{5}{11}}
	\pgfpathlineto{\pgfpointxy{5}{2}}
	\pgfpathclose 
	\pgfusepath{fill,stroke} 

	% Base
	\color{green!70!black}
	\pgfsetstrokecolor{black}
	\pgfpathmoveto{\pgfpointxy{-2}{1.5}}
	\pgfpathcurveto{\pgfpointxy{-2}{1.5}}{\pgfpointxy{-6}{1.5}}{\pgfpointxy{-6}{2.5}}
	\pgfpathlineto{\pgfpointxy{-6}{4}}
	\pgfpathlineto{\pgfpointxy{6}{4}}
	\pgfpathlineto{\pgfpointxy{6}{2.5}}
	\pgfpathcurveto{\pgfpointxy{6}{1.5}}{\pgfpointxy{2}{1.5}}{\pgfpointxy{2}{1.5}}
	\pgfpathclose 
	\pgfusepath{fill,stroke} 

	% Curtains
	\color{red!70!black}
	\pgfsetstrokecolor{black}

	% Left
	\pgfpathmoveto{\pgfpointxy{-6}{11}}
	\pgfpathlineto{\pgfpointxy{-6}{2.5}}
	\pgfpathcurveto{\pgfpointxy{-6}{2.2}}{\pgfpointxy{-3.5}{2.2}}{\pgfpointxy{-3.5}{2.5}}
	\pgfpathcurveto{\pgfpointxy{-3.5}{3}}{\pgfpointxy{-3.5}{4}}{\pgfpointxy{-4.5}{5}}
	\pgfpathcurveto{\pgfpointxy{-2.5}{7}}{\pgfpointxy{-4}{11}}{\pgfpointxy{-3}{11.5}}
	\pgfpathcurveto{\pgfpointxy{-4}{11}}{\pgfpointxy{-2.5}{7}}{\pgfpointxy{-4.5}{5}}
	\pgfpathcurveto{\pgfpointxy{-2.5}{7}}{\pgfpointxy{-6}{11}}{\pgfpointxy{-3}{11.5}}
	\pgfpathcurveto{\pgfpointxy{-6}{11}}{\pgfpointxy{-2.5}{7}}{\pgfpointxy{-4.5}{5}}
	\pgfpathcurveto{\pgfpointxy{-2.5}{7}}{\pgfpointxy{-8}{11}}{\pgfpointxy{-3}{11.5}}
	\pgfpathcurveto{\pgfpointxy{-8}{11}}{\pgfpointxy{-2.5}{7}}{\pgfpointxy{-4.5}{5}}
	\pgfpathcurveto{\pgfpointxy{-2.5}{7}}{\pgfpointxy{-2.5}{11}}{\pgfpointxy{-3}{11.5}}
	\pgfusepath{fill,stroke}

	% Right
	\pgfsetlinewidth{3pt}
	\pgfpathmoveto{\pgfpointxy{6}{11}}
	\pgfpathlineto{\pgfpointxy{6}{2.5}}
	\pgfpathcurveto{\pgfpointxy{6}{2.2}}{\pgfpointxy{3.5}{2.2}}{\pgfpointxy{3.5}{2.5}}
	\pgfpathcurveto{\pgfpointxy{3.5}{3}}{\pgfpointxy{3.5}{4}}{\pgfpointxy{4.5}{5}}
	\pgfpathcurveto{\pgfpointxy{2.5}{7}}{\pgfpointxy{4}{11}}{\pgfpointxy{3}{11.5}}
	\pgfpathcurveto{\pgfpointxy{4}{11}}{\pgfpointxy{2.5}{7}}{\pgfpointxy{4.5}{5}}
	\pgfpathcurveto{\pgfpointxy{2.5}{7}}{\pgfpointxy{6}{11}}{\pgfpointxy{3}{11.5}}
	\pgfpathcurveto{\pgfpointxy{6}{11}}{\pgfpointxy{2.5}{7}}{\pgfpointxy{4.5}{5}}
	\pgfpathcurveto{\pgfpointxy{2.5}{7}}{\pgfpointxy{8}{11}}{\pgfpointxy{3}{11.5}}
	\pgfpathcurveto{\pgfpointxy{8}{11}}{\pgfpointxy{2.5}{7}}{\pgfpointxy{4.5}{5}}
	\pgfpathcurveto{\pgfpointxy{2.5}{7}}{\pgfpointxy{2.5}{11}}{\pgfpointxy{3}{11.5}}
	\pgfusepath{fill,stroke}

	% Top
	%     Top-left
	\pgfpathmoveto{\pgfpointxy{-2}{12}}
	\pgfpathcurveto{\pgfpointxy{-2}{12}}{\pgfpointxy{-6}{12}}{\pgfpointxy{-6}{11}}
	\pgfpathcurveto{\pgfpointxy{-5}{9}}{\pgfpointxy{-2}{11}}{\pgfpointxy{-2}{11.85}}
	\pgfpathcurveto{\pgfpointxy{-2}{11.5}}{\pgfpointxy{-4.5}{9.5}}{\pgfpointxy{-6}{11}}
	\pgfpathcurveto{\pgfpointxy{-4.5}{9.5}}{\pgfpointxy{-2}{11.5}}{\pgfpointxy{-2}{11.85}}
	\pgfpathcurveto{\pgfpointxy{-2}{12}}{\pgfpointxy{-3.5}{10.4}}{\pgfpointxy{-6}{11}}
	\pgfpathcurveto{\pgfpointxy{-3.5}{10.4}}{\pgfpointxy{-2}{12}}{\pgfpointxy{-2}{11.85}}

	%    Top-middle
	\pgfpathcurveto{\pgfpointxy{-1}{10.5}}{\pgfpointxy{1}{10.5}}{\pgfpointxy{2}{11.85}}
	\pgfpathcurveto{\pgfpointxy{1}{10.5}}{\pgfpointxy{-1}{10.5}}{\pgfpointxy{-2}{11.85}}	
	\pgfpathcurveto{\pgfpointxy{-1}{11}}{\pgfpointxy{1}{11}}{\pgfpointxy{2}{11.85}}
	\pgfpathcurveto{\pgfpointxy{1}{11}}{\pgfpointxy{-1}{11}}{\pgfpointxy{-2}{11.85}}	
	\pgfpathcurveto{\pgfpointxy{-1}{10}}{\pgfpointxy{1}{10}}{\pgfpointxy{2}{11.85}}

	%    Top-right
	\pgfpathcurveto{\pgfpointxy{2}{11.5}}{\pgfpointxy{4.5}{9.5}}{\pgfpointxy{6}{11}}
	\pgfpathcurveto{\pgfpointxy{4.5}{9.5}}{\pgfpointxy{2}{11.5}}{\pgfpointxy{2}{11.85}}
	\pgfpathcurveto{\pgfpointxy{2}{12}}{\pgfpointxy{3.5}{10.4}}{\pgfpointxy{6}{11}}
	\pgfpathcurveto{\pgfpointxy{3.5}{10.4}}{\pgfpointxy{2}{12}}{\pgfpointxy{2}{11.85}}
	\pgfpathcurveto{\pgfpointxy{2}{11}}{\pgfpointxy{5}{9}}{\pgfpointxy{6}{11}}
	\pgfpathcurveto{\pgfpointxy{6}{12}}{\pgfpointxy{2}{12}}{\pgfpointxy{2}{12}}
	\pgfpathclose 
	\pgfusepath{fill,stroke} 

	% Rope
	%     Rope-right
	\pgfsetstrokecolor{black}
	\pgfpathmoveto{\pgfpointxy{-4.5}{5}}
	\pgfpathcurveto{\pgfpointxy{-4.5}{5}}{\pgfpointxy{-6}{5}}{\pgfpointxy{-6}{5.5}}
	\pgfusepath{stroke}	
	%     Rope-left
	\pgfsetstrokecolor{black}
	\pgfpathmoveto{\pgfpointxy{4.5}{5}}
	\pgfpathcurveto{\pgfpointxy{4.5}{5}}{\pgfpointxy{6}{5}}{\pgfpointxy{6}{5.5}}
	\pgfusepath{stroke}

	\node[color=black] at (0,0) {{\scalefont{10.0}STAGE}};

	%% Just kinda a pretty path...
	%% \pgfpathmoveto{\pgfpointxy{-2}{1.5}}
	%% \pgfpathcurveto{\pgfpointxy{-2}{1.5}}{\pgfpointxy{-6}{1.5}}{\pgfpointxy{-6}{2.5}}
	%% \pgfpathcurveto{\pgfpointxy{-5}{4.5}}{\pgfpointxy{-2}{2.5}}{\pgfpointxy{-2}{3.35}}
	%% \pgfpathcurveto{\pgfpointxy{-1}{3.5}}{\pgfpointxy{1}{3.5}}{\pgfpointxy{2}{3.35}}
	%% \pgfpathcurveto{\pgfpointxy{2}{2.5}}{\pgfpointxy{5}{4.5}}{\pgfpointxy{6}{2.5}}
	%% \pgfpathcurveto{\pgfpointxy{6}{1.5}}{\pgfpointxy{2}{1.5}}{\pgfpointxy{2}{1.5}}
	%% \pgfpathclose 
\end{tikzpicture}
}
    \vspace{-4em}
\end{figure*}

\maketitle

\begin{abstract}
AHOY is an event-based simulation environment used to compare the effectiveness of different combinations of software agents, network configurations, and sensor data in real-world environments.  It is comprised of a distributed simulation engine, visualizer, and programming interface through which developers create agent software and network topologies.  Communication between virtual nodes is also simulated, providing highly realistic scenarios.
\end{abstract}

\setcounter{tocdepth}{4}
\tableofcontents
\label{toc}
\pagebreak
\pagenumbering{arabic}

\section{Introduction}
\label{sec:introduction}
\subsection{Purpose}
\label{sec:purpose}
%TODO Add number of modules or parts of the system
%TODO Add short list of modules or parts of the system
This document lays out the integration test plan for AHOY. It defines the test cases for the interaction between the 6 modules of the system: Simulation Components, Networking Components, Event Components, Agent Components, Sensor Components, and Utility Components. The integration test plan for this system is based on the project requirements set forth in the Software Requirements Specification document while taking into consideration the system architecture as laid out in the Software Design Specification. The information presented here is intended for the development team, as well as the advisor and external stakeholders, which are currently Dr.~William Regli, Joseph Macker of the Naval Research Laboratory, and Dr.~Michal P\v{e}chou\v{e}k of Czech Technical University. 

\subsection{Scope}
The goal of the AHOY project is to provide a system for testing multiple agents across varying scenarios and topologies, in a distributed, event-driven way. AHOY gives the user the ability to quantitatively examine the effectiveness of specific agent designs as well as a focus on additional factors relevant to the network, including network connectivity, connection fidelity, and the agent's ability to process and transmit data.

Users of AHOY are researchers looking to improve a specific agent's performance on a network through testing across varying combinations of topologies and scenarios.

\subsection{References%
  \label{references}%
}

These documents have been used as reference materials for various technologies involved with this project.
%
\begin{itemize*}
	\item SPEYES: Sensing and Patrolling Enablers Yielding Effective SASO: \url{http://ieeexplore.ieee.org/xpls/abs\_all.jsp?arnumber=1559616}
	\item Service Sniffer Requirements Document: \url{http://servicesniffer.net/documents/requirements.html}
    \item Developing an Agent Systems Reference Architecture: \url{www.cs.drexel.edu/~dn53/papers/paper\_cameraready.pdf}
\end{itemize*}

\section{Simulation Components}
\subsection{Simulation}

\subsubsection{Simulation Acknowledges Startup Daemons}

\testcase{\texttt{Simulation} and \texttt{StartupDaemon} objects are instantiated. At least one \texttt{StartupDaemon}
is created. The \texttt{start()} operation has been called on the \texttt{Simulation}. }{Publish an \texttt{AckStartupEvent} from each \texttt{StartupDaemon} using their \texttt{EventAPI}.}{\texttt{\_on\_ack\_startup()} is called for every \texttt{AckStartupEvent} published.  There are ID's in the \texttt{\_startup\_acks} list for every \texttt{StartupDaemon} publishing the \texttt{Event}.}

\subsection{StartupDaemon}
\subsubsection{Daemon Acknowledges StartupEvent}
\testcase{Set up a \texttt{Network} configuration with at least one daemon.  Configure at least one agent to run.  Confirm that the \texttt{Simulation} sends out a \texttt{StartupEvent} to the multicast channel upon startup, and that \texttt{StartupDaemon} listens to \texttt{StartupEvent}.}{Start the \texttt{Simulation}.}{The \texttt{Simulation} sends out a \texttt{StartupEvent} to the multicast channel, and \texttt{StartupDaemon} receives such a \texttt{StartupEvent}.}

\subsubsection{Daemon Publishes AckStartupEvent}
\testcase{Set up a \texttt{Network} configuration with at least one daemon.  Confirm that Simulation publishes \texttt{StartupEvent} upon startup, that \texttt{StartupDaemon} listens to a \texttt{StartupEvent}, and that \texttt{StartupDaemon} publishes an \texttt{AckStartupEvent}.}{Start the \texttt{Simulation}.}{Once the \texttt{Simulation} is started, \texttt{Daemon} publishes an \texttt{AckStartupEvent} after it receives the \texttt{StartupEvent}.}

\subsubsection{Daemon Starts on StartSimulationEvent}
\testcase{Configure a \texttt{Network} with at least one daemon running, and configure at least one agent to run during the \texttt{Simulation}.  Confirm that \texttt{Daemon} listens to \texttt{StartSimulationEvent}.}{Start the \texttt{Simulation}.}{For each \texttt{StartupDaemon} that is run, that \texttt{StartupDaemon}'s \texttt{start()} method is called upon receiving a \texttt{StartSimulationEvent}.}

\subsubsection{Daemon Starts all Processes at StartSimulationEvent}
\testcase{Configure at least one process to be run during the \texttt{Simulation}.  In \texttt{StartupDaemon}'s \texttt{\_on\_sim\_start()} function, add a confirmation message stating number of \texttt{local\_entities} found to be started.}{Start the \texttt{Simulation}.}{For each daemon that runs, the number of processes it starts is equal to the number it was given to start.  For each process a \texttt{Daemon} has to run, the \texttt{\_start\_entity\_process()} function is called.}

\subsubsection{Daemon Terminates all Processes on StopSimulationEvent}
\testcase{Configure a \texttt{Simulation} with at least one \texttt{Entity} running.}{Start the \texttt{Simulation}. Invoke \texttt{StopSimulationEvent} to be published to the Event Channel.}{Each daemon that is running has 0 processes running after receiving the \texttt{StopSimulationEvent}.}

\subsection{World}
\subsubsection{Entities are Added Successfully}
\testcase{Configure the \texttt{Simulation} to have more than one \texttt{Entity}.  Ensure that AHOY adds \texttt{Entity}s by calling the \texttt{World}'s \texttt{add\_entity()} function.}{Start the \texttt{Simulation}.}{For each \texttt{Entity} configured, \texttt{World}'s \texttt{add\_entity()} function is called.  Each time the function is called, the number of total \texttt{\_entities} of \texttt{World} is incremented by one.}

\subsubsection{World Returns Entity by UID}
\testcase{Configure a world that contains at least one \texttt{Entity}. In the startup script, call \texttt{get\_entity()} on the world object.}{Start the \texttt{Simulation}.}{The UID returned by the \texttt{get\_entity()} function is equal to the UID assigned to the \texttt{Entity} when it was first created.}

\subsubsection{Networks are Added Successfully}
\testcase{Configure the \texttt{World} to contain at least one \texttt{Network}.  Confirm that the networks are added to the \texttt{World} by calling the \texttt{add\_network()} function of \texttt{World}.}{Start the \texttt{Simulation}.}{For each of the \texttt{Network}s configured, the \texttt{add\_network()} function is called, and the total length of \texttt{self.\_networks} increments by one as each \texttt{Network} is added.  The name of the network at \texttt{self.\_networks[-1]} is the name of the network last added, if there has been at least one network added.}

\subsubsection{World Returns Networks by Name}
\testcase{Configure at least one \texttt{Network}.  In the startup script, after the \texttt{Simulation} is started, call \texttt{get\_network()} on the world object.}{Start the \texttt{Simulation} by running the startup script, and starting any necessary daemons.}{The name of the \texttt{Network} that is returned by the startup script calling the \texttt{get\_network()} is the same as the name of \texttt{Network} that was configured.}

\subsection{EventAPI}
\subsubsection{Events are Published}
\testcase{An \texttt{EventAPI} has been instantiated. An \texttt{Event} has been created.}{Call the \texttt{EventAPI}'s \texttt{publish()} operation using the created \texttt{Event} and a delay time of 0 seconds.}{The \texttt{Event} is sent over the Event Channel.}

\subsubsection{Events can be Delayed}
\testcase{An \texttt{EventAPI} has been instantiated. An \texttt{Event} has been created.}{Call the \texttt{EventAPI}'s publish operation using the created \texttt{Event} and any time $t$ greater than 0 for the second argument.}{The \texttt{Event} is sent over the Event Channel $t$ seconds later.}

\subsubsection{Raw Data is successfully passed}
\testcase{\texttt{EventAPI} has been instantiated. Raw data has been created.}{Pass the raw data to the \texttt{EventAPI}'s \texttt{push\_raw()} operation.}{The data can be seen and picked up over the current connection.}

\subsubsection{Subscribed Events Trigger Callbacks}
\testcase{\texttt{EventAPI} has been instantiated.  An \texttt{Event} has been instantiated. A function exists that can be called by the \texttt{EventAPI}.}{Call the \texttt{EventAPI}'s \texttt{subscribe()}.}{When the \texttt{Event} subscribed to occurs, the function is called with all arguments supplied. }

\subsubsection{Entity can Unsubscribe from an Event}
\testcase{An \texttt{EventAPI} has subscribed to an \texttt{Event} with a callback function.}{Call the \texttt{EventAPI}'s \texttt{unsubscribe\_all()} method with the same \texttt{Event} type subscribed to.}{Publish an \texttt{Event} of the same type. The callback function is not executed.}

\subsection{TCPForward}
\subsubsection{Data Received Successfully}
\testcase{A \texttt{TCPForward} object has been instantiated with a valid port number.}{Call the \texttt{start()} method of the \texttt{TCPForward} object. Send an \texttt{Event} over the socket to the same port. }{The \texttt{\_listener()} method is called soon after the message is sent. The raw \texttt{Event} is sent over the Event Channel. }

\subsubsection{Events Forwarded to Clients}
\testcase{A \texttt{TCPForward} object has been created with at least 1 client.}{Publish an \texttt{Event} over the Event Channel.}{All clients receive the \texttt{Event}.}

\subsection{Entity}
\subsubsection{Setting/Retrieving Sensors}
\testcase{In a startup script, configure the world to contain at least one \texttt{Entity}. In the script, after starting the \texttt{Simulation}, create a sensor and call the \texttt{add\_sensor()} method on an \texttt{Entity}.  On the same \texttt{Entity}, then call \texttt{get\_sensor()}.}{Start the \texttt{Simulation} through the script.}{The name the \texttt{get\_sensor()} function returns is the same as the name of the sensor the script tried to add to the \texttt{Entity} at the beginning of the test.}

\subsubsection{Setting/Retrieving Cross-sectional Area}
\testcase{In a startup script, configure the world to contain at least one \texttt{Entity}. In the script, after starting the \texttt{Simulation}, call the \texttt{set\_rsc()} method on an \texttt{Entity}.  On the same \texttt{Entity}, then call \texttt{get\_rcs()}.}{Start the \texttt{Simulation} through the script.}{The value \texttt{get\_rcs()} returns is equal to the value the script tried to set the cross-sectional area to at the beginning of the test.}

\subsubsection{Setting/Retrieving Linear Velocity}
\testcase{In a startup script, configure the world to contain at least one \texttt{Entity}. In the script, after starting the \texttt{Simulation}, call \texttt{set\_lin\_velocity()}. Then, call \texttt{get\_lin\_velocity()} on the same \texttt{Entity}.}{Start the \texttt{Simulation} through the script.}{The value returned by \texttt{get\_lin\_velocity()} is equal to the value passed to the \texttt{set\_lin\_velocity()} function.}

\subsubsection{Setting/Retrieving Forward Velocity}
\testcase{In a startup script, configure the world to contain at least one \texttt{Entity}. In the script, after starting the \texttt{Simulation}, call \texttt{set\_forward\_velocity()}. Then, call \texttt{get\_forward\_velocity()} on the same \texttt{Entity}.}{Start the \texttt{Simulation} through the script.}{The value returned by \texttt{get\_forward\_velocity()} is equal to the value passed to the \texttt{set\_forward\_velocity()} function.}

\subsubsection{Setting/Retrieving World}
\testcase{In a startup script, configure the world to contain at least one \texttt{Entity}. In the script, after starting the \texttt{Simulation}, call \texttt{set\_world()}. Then, call \texttt{get\_world()} on the same \texttt{Entity}.}{Start the \texttt{Simulation} through the script.}{The memory address of the object returned by \texttt{get\_world()} is the same as the address of the world passed to the \texttt{set\_world()} function.}

\subsubsection{Setting/Retrieving Position}
\testcase{In a startup script, configure the world to contain at least one \texttt{Entity}. In the script, after starting the \texttt{Simulation}, call \texttt{set\_position()}. Then, call \texttt{get\_position()} on the same \texttt{Entity}.}{Start the \texttt{Simulation} through the script.}{The value returned by \texttt{get\_position()} is equal to the value passed to the \texttt{set\_position()} function.}

\subsubsection{Moving to a New Position}
\testcase{In a startup script, configure the world to contain an \texttt{Entity}.  In the script, after starting the \texttt{Simulation}, call \texttt{move()} on the \texttt{Entity}. Then, call \texttt{get\_position()} on the same \texttt{Entity}.}{Start the \texttt{Simulation} through the script.}{Each time the \texttt{Entity} makes a move towards its destination, it decides its next position in its \texttt{\_iterate\_move()} function. A series of \texttt{EntityMoveEvent} published by the \texttt{Entity} indicate that the \texttt{Entity} in the \texttt{Simulation} is moving towards the location that was passed into the \texttt{move()} function of that \texttt{Entity}. It eventually reaches this destination.}

\subsubsection{Pickle/Deserialize Properly}
\testcase{In a startup script, configure the world to contain at least one \texttt{Entity}. In the script, after starting the \texttt{Simulation}, call \texttt{pickle()} on an \texttt{Entity}. Then, call \texttt{from\_pickle()} on the object that was returned by \texttt{pickle()}.}{Start the \texttt{Simulation} through the script.}{The \texttt{Entity} returned by \texttt{from\_pickle()} has the same values for all of its properties as the \texttt{Entity} passed to the \texttt{pickle()} function.}

\subsection{Node}
\subsubsection{Add and Get an Interface}
\testcase{In a startup script, create a \texttt{Node}. Then, create an \texttt{Interface} and pass it to the node's \texttt{add\_interface()} function. Finally, call the \texttt{get\_interface()} on the \texttt{Node}.}{Run the startup script.}{The total length of \texttt{self.\_interfaces} of \texttt{Node} increments by one after the \texttt{add\_interface()} function is called.}

\subsubsection{Remove an Interface}
\testcase{In a startup script, create a \texttt{Node}. Then, create an \texttt{Interface} and pass it to the node's \texttt{add\_interface()} function. Finally, call the \texttt{remove\_interface()} on the \texttt{Node}.}{Run the startup script.}{The total length of \texttt{self.\_interfaces} of \texttt{Node} decrements by one after the \texttt{remove\_interface()} function is called.}

\subsubsection{Add an Agent}
\testcase{In a startup script, create a \texttt{Node}.  Then, create an \texttt{Agent} and pass it to the \texttt{Node}'s \texttt{add\_agent()} function.}{Run the startup script.}{The \texttt{Agent} that was passed to the \texttt{add\_agent()} function is in \texttt{Node}'s \texttt{\_agents}. The total length of \texttt{self.\_agents.keys()} is incremented by one for each \texttt{Agent} added.}

\subsubsection{Send a Message}
\testcase{In a script, create an \texttt{Agent}, a \texttt{Node}, and an \texttt{Interface}.  Add the \texttt{Interface} to the \texttt{Node} by calling its \texttt{add\_interface()} method. Create a message to be sent.  Call the \texttt{Node}'s \texttt{send()} function.}{Run the startup script.}{The message that was passed to \texttt{Node}'s \texttt{send()} function is sent over its \texttt{Interface}.}

\subsection{ScriptedEntity}
\subsubsection{Scripted Entity Runs}
\testcase{Configure a \texttt{ScriptedEntity} to run in simulation.}{Start the simulation.}{The output messages the configured \texttt{ScriptedEntity} outputs by default indicate that it is moving as it was configured.}

\section{Networking Components}
\subsection{Interface}
\subsubsection{Messages Sent}
\testcase{An \texttt{Interface} has been instantiated with a \texttt{Node} that contains at least one \texttt{Agent}.}{Call the \texttt{connect()} method. Pass a \texttt{Message} to the \texttt{send\_message()} method using a valid \texttt{Agent} ID as the second argument.}{The message appears on the Event Channel as a \texttt{CommunicationSendEvent}.}

\subsubsection{Proper Function Called When Message Received}
\testcase{An \texttt{Interface} has been instantiated with a proper \texttt{Network} name and a \texttt{Node} containing at least one \texttt{Agent}. The function \texttt{set\_recv\_callback()} has been called on the \texttt{Interface} with a valid function.}{Call the \texttt{connect()} method of the \texttt{Interface}. Publish an \texttt{Event} over the Event Channel that contains the same \texttt{network\_name} as the \texttt{Interface} and a destination \texttt{Agent} ID. The ID belongs to an \texttt{Agent} on the owner \texttt{Node}.}{The function callback specified in the \texttt{\_recv\_callback} of the \texttt{Interface} is called.}

\subsection{CommsEngine}
\subsubsection{Node Returned by Agent UID}
\testcase{A \texttt{Simulation} has been created with at least one \texttt{Agent} running on a \texttt{Node}.}{Call the \texttt{get\_node\_from\_agent()} method with a valid \texttt{Agent} ID.}{The \texttt{Node} running that \texttt{Agent} with the given UID is returned.}

\subsection{LogLossCommsEngine}
\subsubsection{Pathloss Calculated Correctly}
\testcase{A \texttt{Simulation} is running with at least two \texttt{Node}s and a \texttt{LogLossCommsEngine}. }{Call \texttt{\_get\_rx\_power()} with appropriate source \texttt{Node} and destination \texttt{Node} IDs. A valid number is passed as the third argument for the source power.}{The function returns a number equivalent to what is expected by the equation in 4.2.3.1 of the AHOY Design Document.}

\subsubsection{Determines if a Message can be Delivered}
\testcase{A \texttt{Simulation} is running with at least two \texttt{Node}s and a \texttt{LogLossCommsEngine}.}{Call \texttt{\_should\_deliver()} with appropriate source \texttt{Node} and destination \texttt{Node} IDs. A valid number is passed as the third argument for the source power. The fourth argument is a number representing the minimum power sensitivity required to send the message.}{The function returns \texttt{True} if the equation in section 4.2.3.1 of the AHOY Design Document, using the \texttt{Node} positions and the source power as input, yields a result greater than or equal to the minimum power required. Otherwise it returns \texttt{False}.}

\subsubsection{Links Updated After Movement}
\testcase{A \texttt{Simulation} is running with at least two \texttt{Node}s and a \texttt{LogLossCommsEngine}.}{Send an \texttt{EntityMoveEvent}.}{The \texttt{LogLossCommsEngine} checks all links between the \texttt{Node}s. If any links are created or nodes separate beyond a linkable distance, a \texttt{LinkEvent} is published by the \texttt{LogLossCommsEngine}.}

\subsection{EthernetCommsEngine}
\subsubsection{Proper Action Taken on CommunicationSendEvent}
\testcase{A \texttt{Simulation} is running with multiple \texttt{Node}s and an \texttt{EthernetCommsEngine}.}{Publish a \texttt{CommunicationSendEvent}.}{The action assigned to the \texttt{EthernetCommsEngine} take occurs.}

\section{Event Components}
\subsection{Event}
\subsubsection{Event is Serialized/Deserialized}
\testcase{In a startup script, create an \texttt{Event} of a type that is extended from the \texttt{Event} class. In the script, call \texttt{pickle()} on the \texttt{Event}. Then, call \texttt{from\_pickle()} on the object that was returned by \texttt{pickle()}.}{Run the startup script. Do this for each type of \texttt{Event}.}{The \texttt{Event} returned by \texttt{from\_pickle()} has the same values for all of its properties as the \texttt{Event} passed to the \texttt{pickle()} function.}

\subsection{LinkEvent}
\subsubsection{Maintains All Assigned Properties}
\testcase{In a startup script, create a \texttt{Network} and two nodes, and configure them to run in the \texttt{Simulation}. In the script, after starting the \texttt{Simulation}, publish a \texttt{LinkEvent}.}{Start the \texttt{Simulation} through the script.}{Whenever a \texttt{LinkEvent} occurs, the values returned by  \texttt{get\_up()}, \texttt{get\_uid1()}, \texttt{get\_uid2()}, \texttt{get\_network\_name()}, \texttt{get\_pathloss()} on the \texttt{LinkEvent} that occurred are equivalent to their corresponding values of the \texttt{LinkEvent} that was published by the script.}

\subsection{CommunicationSendEvent}
\subsubsection{Maintains All Assigned Properties}
\testcase{In a startup script, create a \texttt{Network} and two nodes, and configure them to run in the \texttt{Simulation}.  In the script, after starting the \texttt{Simulation}, publish a \texttt{CommunicationSendEvent}.}{Start the \texttt{Simulation} through the script.}{Whenever a \texttt{CommunicationSendEvent} occurs, the values returned by \texttt{get\_src\_agent\_uid()}, \texttt{get\_src\_iface\_name()}, \texttt{get\_message()}, \texttt{get\_network()} on the \texttt{CommunicationSendEvent} that occurred are equivalent to their corresponding values of the \texttt{CommunicationSendEvent} that was published by the script.}

\subsection{CommunicationRecvEvent}
\subsubsection{Maintains All Assigned Properties}
\testcase{In a startup script, create a \texttt{Network} and two nodes, and configure them to run in the \texttt{Simulation}. In the script, after starting the \texttt{Simulation}, publish a \texttt{CommunicationRecvEvent}.}{Start the \texttt{Simulation} through the script.}{Whenever a \texttt{CommunicationRecvEvent} occurs, the values returned by \texttt{get\_src\_agent\_uid()}, \texttt{get\_src\_iface\_name()}, \texttt{get\_message()}, \texttt{get\_network()} on the \texttt{CommunicationRecvEvent} that occurred are equivalent to their corresponding values of the \texttt{CommunicationRecvEvent} that was published by the script.}

\subsection{EntityMoveEvent}
\subsubsection{Maintains All Assigned Properties}
\testcase{In a startup script, create an \texttt{Entity} to run in the \texttt{Simulation}.  In the script, after starting the \texttt{Simulation}, publish a \texttt{EntityMoveEvent}.}{Start the \texttt{Simulation} through the script.}{Whenever a \texttt{EntityMoveEvent} occurs, the values returned by \texttt{get\_uid()}, \texttt{get\_lat()}, \texttt{get\_lon()}, \texttt{get\_agl()}, \texttt{get\_velocity()} on the \texttt{EntityMoveEvent} that occurred are equivalent to their corresponding values of the \texttt{EntityMoveEvent} that was published by the script.}

\subsection{StartupEvent}
\subsubsection{Maintains World Property}
\testcase{In a startup script, create an \texttt{Entity} to run in the \texttt{Simulation}.  Verify that a \texttt{StartupEvent} will be called when the \texttt{Simulation} is started.}{Start the \texttt{Simulation} through the script.}{When a \texttt{StartupEvent} occurs, the memory location returned by the \texttt{get\_world()} function is the same as the memory location of the world object being passed to the \texttt{StartupEvent}. Also, the object returned by the function is of type \texttt{World}.}

\subsection{AckStartupEvent}
\subsubsection{Daemon ID Maintained}
\testcase{In a startup script, after starting the \texttt{Simulation}, publish an \texttt{AckStartupEvent}.}{Start the \texttt{Simulation} through the script.}{Whenever a \texttt{AckStartupEvent} occurs, the value returned by the \texttt{get\_daemon\_id()} function is equal to the value passed by the script when publishing the \texttt{EntityMoveEvent}.}

\subsection{StartSimulationEvent}
\subsubsection{Daemon to Entity Mapping Maintained}
\testcase{In a startup script, create a dictionary containing integers as both keys and values. Publish a \texttt{StartSimulationEvent}, and pass it this dictionary.}{Run the startup script.}{Whenever a \texttt{StartSimulationEvent} occurs, the dictionary returned by the \texttt{get\_mapping()} function is identical to the dictionary passed by the script when publishing the \texttt{StartSimulationEvent}.}

\subsection{StopSimulationEvent}
\subsubsection{Capable of Being Invoked}
\testcase{In the script, publish a \texttt{StopSimulationEvent}.}{Run the startup script.}{A \texttt{StopSimulationEvent} occurs.}

\section{Agent Components}
\subsection{Agent}
\subsubsection{Behaviors Initialized}
\testcase{An \texttt{Agent} has been created in a startup script.}{Add multiple behavior tuples to the \texttt{Agent}. Start the \texttt{Simulation}.}{The \texttt{\_behaviors} dictionary contains a key for each \texttt{Event}, and at least one \texttt{Condition} and \texttt{Action} pair for each \texttt{Event}.}

\subsubsection{Action Triggered Appropriately}
\testcase{A Simulation is running with at least one \texttt{Agent} with assigned behaviors.}{Publish an \texttt{Event} that the \texttt{Agent} uses as a key in its behavior dictionary.}{For each \texttt{Action} / \texttt{Behavior} pair associated in the \texttt{Agent}'s behavior map, the \texttt{Condition} is tested.  If it returns true, the \texttt{Action} is performed.  If it returns \texttt{False}, the \texttt{Action} is not performed.}

\subsubsection{Proper Behavior Removed}
\testcase{A \texttt{Simulation} is running with at least one \texttt{Agent} with assigned behaviors.}{Call the \texttt{Agent}'s \texttt{remove\_behavior()} method with a behavior tuple.}{If the Agent contains a behavior set containing the same \texttt{Event}, \texttt{Action}, and \texttt{Condition} pairs, that behavior is removed from the \texttt{Agent}'s behavior map.}

\subsection{MoveAction}
\subsubsection{Proper Entity is Moved}
\testcase{A \texttt{MoveAction} has been instantiated.}{Call the perform method of the \texttt{MoveAction}.}{The \texttt{Entity} in the \texttt{\_entity} attribute changes its position.}

\section{Sensor Components}
\subsection{Sensor}
\subsubsection{Event Data Passed to Callbacks}
\testcase{A \texttt{Simulation} is running with a \texttt{Sensor} running on a \texttt{Node}. The \texttt{Sensor} has subscribed at least one callback function.}{The \texttt{\_publish\_data()} method is called during an \texttt{Event}.}{The \texttt{Event} is passed to all callback functions subscribed to by the \texttt{Sensor}. If \texttt{\_use\_event\_channel} attribute of the \texttt{Sensor} is \texttt{True}, then the \texttt{EventAPI} of the \texttt{Sensor}'s owner publishes the same \texttt{Event}.}

\subsection{RadarSensor}
\subsubsection{Objects Properly Detected}
\testcase{A \texttt{Simulation} is running with multiple \texttt{Node}s and a \texttt{RadarSensor} running on one of the \texttt{Node}s.}{The run method of the \texttt{RadarSensor} is called.}{For each object in the \texttt{Simulation}, if the pathloss of an object relative to the radar antenna is below the sensitivity of an antenna, the object is not detected.  Otherwise, its distance is calculated for each \texttt{Entity} is the same as the result of the distance equation in section 4.5.2.1 of AHOY's Design Document. The velocity of the \texttt{Entity} being detected is equivalent to that produced by the Doppler equation, also found in section 4.5.2.1 of the Design Document.}

\section{Utility Components}
\subsection{Geo}
\subsubsection{Haversine Distance Calculated Correctly}
\testcase{None}{Call \texttt{haver\_distance()} with latitude and longitude arguments.}{A correct Haversine distance is returned.}

\subsubsection{Linear Distance Calculated Correctly}
\testcase{None}{Call \texttt{linear\_distance()} with latitude and longitude arguments}{The correct linear distance between the points is returned.}

\subsubsection{Lat/Lon/Agl Converted to Cartesian Coords}
\testcase{None}{Call \texttt{sph\_to\_lin()} with a latitude, longitude, and above ground location in that order.}{A correct conversion into Cartesian coordinates is returned.}

\subsubsection{Cartesian Coords Converted to Lat/Lon/Agl}
\testcase{None}{Call \texttt{lin\_to\_sph()} with Cartesian x, y, and z coordinates.}{The input coordinates are correctly converted to latitude, longitude, and above aground location points.}

\subsubsection{Kilometers converted to radians Correctly}
\testcase{None}{Call \texttt{loc\_from\_bearing\_dist()} with appropriate latitude and longitude inputs.}{The kilometers along the latitude and longitude lines are correctly converted to radians and returned.}

\subsubsection{Location and Bearing Calculated Correctly}
\testcase{None}{Call \texttt{loc\_from\_bearing\_dist()} with valid longitude/latitude points along with a bearing and distance.}{The correct location of an object after moving from the given lat/lon at the given bearing for the given distance is returned.}

\pagebreak
\appendix
\appendixpage

\section{Definitions, Acronyms, and Abbreviations}
\label{sec:glossary}
\begin{description}
\item[Agent]
	Agents are simulated pieces of software that run on nodes in the network. They consist of different algorithms that are relevant for the user to test on different scenarios and topologies.   
\item[Distribution]
	Distribution refers to the process of distributing the simulation across a multi-platform physical cluster.  This allows the system to exceed the number of nodes per platform for a single simulation at the system's discretion.  A framework is provided to allow the user to distribute their simulation. 	
\item[Node]
	Nodes are virtual or physical machines that consist of agents and network interfaces.  If nodes are virtual, many nodes may run on one physical machine.  
\item[Scenario]
	Scenario is comprised of a scripted language indicating the location simulated nodes within the virtual world. These nodes consist of agents (see definition of `Agent') and non-agent world objects such as planes, boats, ground vehicles, etc. 
\item[Multicast]
    A method of simultaniously delivering a message to many computers.
\item[TCP]
    Transmission Control Protocol.  A reliable, connection-oriented, communication method guaranteeing in-order, error-free delivery of all messages.
\item[Terrain]
	Terrain refers to the simulated landscape.  This includes such geography as the slope of the land, the tree density, water v.s. land surfaces, etc.
\item[Topology]
	Topology describes time-dependent connections between nodes and their characteristics (e.g. radio model). It is described with a scripting language which specifies the details of network interfaces on each simulated node, including radio models and throughput characteristics.  It describes any physical or wireless links that connect these interfaces.  Further, it indicates changes in linkage over time such as a wireless interface switches, wireless LANs, or a physical link being created or severed. 
\item[UDP]
    User Datagram Protocol.  A stateless, non-connection-oriented, communication method which does not guarantee in-order, error-free delivery of messages.
\item[Visualizer]
	The Visualizer allows the simulations to be superimposed over real-world topography.  This permits the user to examine the behavior of the agents.  It also allows for overlays such as link quality, traffic rates, and other metrics deemed important to specific components.
\end{description}

\end{document}
