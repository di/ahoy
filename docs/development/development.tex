\RequirePackage{fix-cm}
\documentclass[titlepage]{article}

\usepackage[utf8]{inputenc}
\usepackage{fullpage}    % Use the whole page
\usepackage{fancyhdr}    % Nice headers/footers
\usepackage{graphicx}    % Importing graphics
\usepackage{hyperref}    % Hyperlink references and URLs
\usepackage[figure,table]{hypcap} % Hyperlink points to the top of figures
\usepackage[usenames,dvipsnames]{xcolor}	% Logo
\usepackage{tikz,ifthen}			% Logo
\usepackage{pgf}				% Logo
\usepackage{scalefnt}				% Logo
\usepgfmodule{shapes}				% Logo
\usepgfmodule{plot}				% Logo
\usetikzlibrary{shapes,arrows,shadows,fit}
\usepackage{pgf-umlsd}
\usepackage{multirow}
\usepackage{mdwlist}
\usepackage{colortbl}
\usepackage{calc}
\usepackage{float}
\usepackage{longtable}
\usepackage{amsmath}
\usepackage{appendix}
\usepackage{listings}   %for displaying code
\usepackage{rotating}	% for sideways column headings in table

\renewenvironment{itemize*}
    {\begin{itemize}
        \setlength{\itemsep}{0pt}%
        \setlength{\parskip}{0pt}%
        \setlength{\partopsep}{0pt}%
        \setlength{\topsep}{0pt}}%
    {\end{itemize}}

\newcommand{\testcase}[3]{
    \begin{center}
    \begin{tabular}{| l | p{0.7\textwidth}|}
        \hline
        \rowcolor[gray]{0.8}\textbf{Pre-Condition:} & #1 \\ \hline
        \textbf{Action:} & #2 \\ \hline
        \rowcolor[gray]{0.8}\textbf{Post-Condition:} & #3 \\ \hline
    \end{tabular}
    \end{center}
}

% Just so we don't have to specify this twice
\newcommand\mytitle{Application Development Review Document}
\newcommand\mydate{\today}
\newcommand\myversion{1}

\hypersetup{
    colorlinks=true,
    linkcolor=blue,
    urlcolor=blue,
    pdftitle={AHOY \mytitle V\myversion},
    pdfauthor={Dustin Ingram, Aaron Rosenfeld, Maria Kolakowska, Frank Clark}
}

% To make referencing sections more biggerer and clickyer 
\newcommand{\rrref}[2]{\hyperref[#2]{#1}}
\newcommand{\sref}[1]{\hyperref[#1]{Section~\ref*{#1}}}
\newcommand{\fref}[1]{\hyperref[#1]{Figure~\ref*{#1}}}
\newcommand{\aref}[1]{\hyperref[#1]{Appendix~\ref*{#1}}}

% So we can number subsubsections too
\setcounter{secnumdepth}{5}

% For headers and footers
\setlength{\headheight}{15pt}
\setlength{\headsep}{25pt}
\pagestyle{fancy}
	
% Page style for the title page
\fancypagestyle{plain}{
    \fancyhf{}
    \renewcommand{\headrulewidth}{0pt}
    \renewcommand{\footrulewidth}{0pt}
}

% Page style for every other page
\fancyhf{} % clear all header and footer fields
\fancyhead[L]{AHOY}
\fancyhead[C]{\mytitle}
\fancyhead[R]{\mydate}
\fancyfoot[C]{\thepage}
\renewcommand{\headrulewidth}{0.4pt}
\renewcommand{\footrulewidth}{0.4pt}

\title{\textbf{\mytitle}}
\author{
	Frank Clark \\\url{francis.j.clark@drexel.edu}
    \and Dustin Ingram \\\url{dustin.s.ingram@drexel.edu}
	\and Maria Kolakowska \\\url{maria.j.kolakowska@drexel.edu}
    \and Aaron Rosenfeld \\\url{aaron.rosenfeld@drexel.edu}
}
\date{\mydate\\Version \myversion}
\begin{document}
\pagenumbering{roman}

\begin{figure*}
   % \vspace{-2em}
    \centering
    \scalebox{0.8}{
\begin{tikzpicture}[scale=1]
	
	\pgfsetlinewidth{3pt}

	% Background
	\color{cyan!70!black}
	\pgfpathmoveto{\pgfpointxy{-5}{2}}
	\pgfpathlineto{\pgfpointxy{-5}{11}}
	\pgfpathlineto{\pgfpointxy{-2}{11.9}}	
	\pgfpathlineto{\pgfpointxy{2}{11.9}}	
	\pgfpathlineto{\pgfpointxy{5}{11}}
	\pgfpathlineto{\pgfpointxy{5}{2}}
	\pgfpathclose 
	\pgfusepath{fill,stroke} 

	% Base
	\color{green!70!black}
	\pgfsetstrokecolor{black}
	\pgfpathmoveto{\pgfpointxy{-2}{1.5}}
	\pgfpathcurveto{\pgfpointxy{-2}{1.5}}{\pgfpointxy{-6}{1.5}}{\pgfpointxy{-6}{2.5}}
	\pgfpathlineto{\pgfpointxy{-6}{4}}
	\pgfpathlineto{\pgfpointxy{6}{4}}
	\pgfpathlineto{\pgfpointxy{6}{2.5}}
	\pgfpathcurveto{\pgfpointxy{6}{1.5}}{\pgfpointxy{2}{1.5}}{\pgfpointxy{2}{1.5}}
	\pgfpathclose 
	\pgfusepath{fill,stroke} 

	% Curtains
	\color{red!70!black}
	\pgfsetstrokecolor{black}

	% Left
	\pgfpathmoveto{\pgfpointxy{-6}{11}}
	\pgfpathlineto{\pgfpointxy{-6}{2.5}}
	\pgfpathcurveto{\pgfpointxy{-6}{2.2}}{\pgfpointxy{-3.5}{2.2}}{\pgfpointxy{-3.5}{2.5}}
	\pgfpathcurveto{\pgfpointxy{-3.5}{3}}{\pgfpointxy{-3.5}{4}}{\pgfpointxy{-4.5}{5}}
	\pgfpathcurveto{\pgfpointxy{-2.5}{7}}{\pgfpointxy{-4}{11}}{\pgfpointxy{-3}{11.5}}
	\pgfpathcurveto{\pgfpointxy{-4}{11}}{\pgfpointxy{-2.5}{7}}{\pgfpointxy{-4.5}{5}}
	\pgfpathcurveto{\pgfpointxy{-2.5}{7}}{\pgfpointxy{-6}{11}}{\pgfpointxy{-3}{11.5}}
	\pgfpathcurveto{\pgfpointxy{-6}{11}}{\pgfpointxy{-2.5}{7}}{\pgfpointxy{-4.5}{5}}
	\pgfpathcurveto{\pgfpointxy{-2.5}{7}}{\pgfpointxy{-8}{11}}{\pgfpointxy{-3}{11.5}}
	\pgfpathcurveto{\pgfpointxy{-8}{11}}{\pgfpointxy{-2.5}{7}}{\pgfpointxy{-4.5}{5}}
	\pgfpathcurveto{\pgfpointxy{-2.5}{7}}{\pgfpointxy{-2.5}{11}}{\pgfpointxy{-3}{11.5}}
	\pgfusepath{fill,stroke}

	% Right
	\pgfsetlinewidth{3pt}
	\pgfpathmoveto{\pgfpointxy{6}{11}}
	\pgfpathlineto{\pgfpointxy{6}{2.5}}
	\pgfpathcurveto{\pgfpointxy{6}{2.2}}{\pgfpointxy{3.5}{2.2}}{\pgfpointxy{3.5}{2.5}}
	\pgfpathcurveto{\pgfpointxy{3.5}{3}}{\pgfpointxy{3.5}{4}}{\pgfpointxy{4.5}{5}}
	\pgfpathcurveto{\pgfpointxy{2.5}{7}}{\pgfpointxy{4}{11}}{\pgfpointxy{3}{11.5}}
	\pgfpathcurveto{\pgfpointxy{4}{11}}{\pgfpointxy{2.5}{7}}{\pgfpointxy{4.5}{5}}
	\pgfpathcurveto{\pgfpointxy{2.5}{7}}{\pgfpointxy{6}{11}}{\pgfpointxy{3}{11.5}}
	\pgfpathcurveto{\pgfpointxy{6}{11}}{\pgfpointxy{2.5}{7}}{\pgfpointxy{4.5}{5}}
	\pgfpathcurveto{\pgfpointxy{2.5}{7}}{\pgfpointxy{8}{11}}{\pgfpointxy{3}{11.5}}
	\pgfpathcurveto{\pgfpointxy{8}{11}}{\pgfpointxy{2.5}{7}}{\pgfpointxy{4.5}{5}}
	\pgfpathcurveto{\pgfpointxy{2.5}{7}}{\pgfpointxy{2.5}{11}}{\pgfpointxy{3}{11.5}}
	\pgfusepath{fill,stroke}

	% Top
	%     Top-left
	\pgfpathmoveto{\pgfpointxy{-2}{12}}
	\pgfpathcurveto{\pgfpointxy{-2}{12}}{\pgfpointxy{-6}{12}}{\pgfpointxy{-6}{11}}
	\pgfpathcurveto{\pgfpointxy{-5}{9}}{\pgfpointxy{-2}{11}}{\pgfpointxy{-2}{11.85}}
	\pgfpathcurveto{\pgfpointxy{-2}{11.5}}{\pgfpointxy{-4.5}{9.5}}{\pgfpointxy{-6}{11}}
	\pgfpathcurveto{\pgfpointxy{-4.5}{9.5}}{\pgfpointxy{-2}{11.5}}{\pgfpointxy{-2}{11.85}}
	\pgfpathcurveto{\pgfpointxy{-2}{12}}{\pgfpointxy{-3.5}{10.4}}{\pgfpointxy{-6}{11}}
	\pgfpathcurveto{\pgfpointxy{-3.5}{10.4}}{\pgfpointxy{-2}{12}}{\pgfpointxy{-2}{11.85}}

	%    Top-middle
	\pgfpathcurveto{\pgfpointxy{-1}{10.5}}{\pgfpointxy{1}{10.5}}{\pgfpointxy{2}{11.85}}
	\pgfpathcurveto{\pgfpointxy{1}{10.5}}{\pgfpointxy{-1}{10.5}}{\pgfpointxy{-2}{11.85}}	
	\pgfpathcurveto{\pgfpointxy{-1}{11}}{\pgfpointxy{1}{11}}{\pgfpointxy{2}{11.85}}
	\pgfpathcurveto{\pgfpointxy{1}{11}}{\pgfpointxy{-1}{11}}{\pgfpointxy{-2}{11.85}}	
	\pgfpathcurveto{\pgfpointxy{-1}{10}}{\pgfpointxy{1}{10}}{\pgfpointxy{2}{11.85}}

	%    Top-right
	\pgfpathcurveto{\pgfpointxy{2}{11.5}}{\pgfpointxy{4.5}{9.5}}{\pgfpointxy{6}{11}}
	\pgfpathcurveto{\pgfpointxy{4.5}{9.5}}{\pgfpointxy{2}{11.5}}{\pgfpointxy{2}{11.85}}
	\pgfpathcurveto{\pgfpointxy{2}{12}}{\pgfpointxy{3.5}{10.4}}{\pgfpointxy{6}{11}}
	\pgfpathcurveto{\pgfpointxy{3.5}{10.4}}{\pgfpointxy{2}{12}}{\pgfpointxy{2}{11.85}}
	\pgfpathcurveto{\pgfpointxy{2}{11}}{\pgfpointxy{5}{9}}{\pgfpointxy{6}{11}}
	\pgfpathcurveto{\pgfpointxy{6}{12}}{\pgfpointxy{2}{12}}{\pgfpointxy{2}{12}}
	\pgfpathclose 
	\pgfusepath{fill,stroke} 

	% Rope
	%     Rope-right
	\pgfsetstrokecolor{black}
	\pgfpathmoveto{\pgfpointxy{-4.5}{5}}
	\pgfpathcurveto{\pgfpointxy{-4.5}{5}}{\pgfpointxy{-6}{5}}{\pgfpointxy{-6}{5.5}}
	\pgfusepath{stroke}	
	%     Rope-left
	\pgfsetstrokecolor{black}
	\pgfpathmoveto{\pgfpointxy{4.5}{5}}
	\pgfpathcurveto{\pgfpointxy{4.5}{5}}{\pgfpointxy{6}{5}}{\pgfpointxy{6}{5.5}}
	\pgfusepath{stroke}

	\node[color=black] at (0,0) {{\scalefont{10.0}STAGE}};

	%% Just kinda a pretty path...
	%% \pgfpathmoveto{\pgfpointxy{-2}{1.5}}
	%% \pgfpathcurveto{\pgfpointxy{-2}{1.5}}{\pgfpointxy{-6}{1.5}}{\pgfpointxy{-6}{2.5}}
	%% \pgfpathcurveto{\pgfpointxy{-5}{4.5}}{\pgfpointxy{-2}{2.5}}{\pgfpointxy{-2}{3.35}}
	%% \pgfpathcurveto{\pgfpointxy{-1}{3.5}}{\pgfpointxy{1}{3.5}}{\pgfpointxy{2}{3.35}}
	%% \pgfpathcurveto{\pgfpointxy{2}{2.5}}{\pgfpointxy{5}{4.5}}{\pgfpointxy{6}{2.5}}
	%% \pgfpathcurveto{\pgfpointxy{6}{1.5}}{\pgfpointxy{2}{1.5}}{\pgfpointxy{2}{1.5}}
	%% \pgfpathclose 
\end{tikzpicture}
}
    \vspace{-4em}
\end{figure*}

\maketitle
\pagenumbering{arabic}

\section{Introduction}
This document gives a brief overview of the current AHOY application development to date, discusses issues encountered and resolved, and outlines future work to be completed. 

\subsection{Short Status}
The AHOY development team has completed the second prototype and has finished outlining features which must be included to complete the final iteration. The team has also begun preparing their final presentations.

\section{Completed Work}
The completion of the second prototype gives a basic, yet functioning, simulation environment. 

\subsection{Event Channel API}
The Event Channel API decouples the simulator from any type of visualizer or data collection application. It is robust, detailed, and can be accessed within the network of distributed simulation nodes, or outside it.

\subsection{Simulation Distribution}
In tandem with the Event Channel API, AHOY can now be fully distributed across nodes on a local network. 

\subsection{Agent, Sensor, and Network Framework}
AHOY now has rudimentary methods to define an agent, node, sensor, and their corresponding networks.

\subsection{RADAR Sensor}
A sample RADAR sensor, using an actual sensor model, is currently implemented.

\section{Issues Encountered}
The main issues encountered thus far are not immediately related to the current prototypes, but rather in anticipation of the final presentations. 

\subsection{Capabilities of the Demo Visualizer}
The demonstration visualizer is not an inherent part of the project, but is essential to effectively convey the scenario to the audience. The current visualizer being used is SDT, a spinoff of WorldWind. While this visualizer has been trivial to integrate with AHOY, it lacks some features (3D building models, 3D node models, etc.) which will be crucial to a complete understanding. The team has decided to switch visualizers to Google Earth as a result of some preliminary investigation.

\subsection{Issues with Video Presentation}
In the initial project presentation, the team encountered issues presenting video of the prototype. These issues have since been resolved with different hardware and software.

\section{Future Work}
This section outlines future work needed to support the final scenario and presentation. It mostly consists of the creation of various types of `example' agents, sensors and nodes, as work on the core simulator is nearing completion.

\subsection{Boat Traffic Models}
These models for boat agents will be used to generate random paths from real data for a ship node.

\subsubsection{Probabilistic AIS ships}
Using a probabilistic model to generate movement patterns for large cargo ships based on real-world AIS data allows us to scale the number of AIS ships in a scenario without duplication.

\subsubsection{Non-AIS ships}
Because there is no available data for the movement of non-AIS ships, this will need to be nondeterministically generated, but without a probabilistic model.

\subsection{Additional Sensor Models}
AHOY should be able to handle a variety of sensor models. The following outline several sensors which need to be developed to support the final scenario.

\subsubsection{SONAR Sensor}
Completing initial work on a SONAR sensor is nearing completion, but needs testing.

\subsubsection{Camera Sensor}
Work needs to be done to determine whether developing a camera sensor will be useful in the final scenario. Such a sensor may require more work than it is worth.

\subsubsection{Chemical Spill Sensor}
A basic model of a chemical spill sensor will add another variety of ``threats'' to the final scenario.

\subsection{``Threat Detector'' Agent Models}
Developing agents which can monitor a wide array of sensors, and intelligently detect and report events to a human operator, will demonstrate an agent's use of the event channel, both to monitor events, and produce events of it's own.

\subsection{Visualization Improvements}
Improving the visualization will add a new level to the presentations demonstration.

\subsubsection{Full switch from SDT to GE}
The final transition from SDT to Google Earth needs to be completed.

\subsubsection{Performance Optimization of GE}
Google Earth requires some performance optimization before it will be worth using.

\subsection{Scenario Definition}
The scenario is currently roughly defined. Once the relevant agent and sensor models are completed, it can be refined.

\subsection{Scenario Collection \& Replay}
To conserve time and add smoothness to the final presentation, the final demonstration will not be live, but pre-recorded. The ability to ``collect'' all the events from an executing scenario and replay them to create different recordings of the same scenario will allow for this. 

\end{document}
