\RequirePackage{fix-cm}
\documentclass[titlepage]{article}

\usepackage[utf8]{inputenc}
\usepackage{fullpage}    % Use the whole page
\usepackage{fancyhdr}    % Nice headers/footers
%\usepackage{mdwlist}     % For itemize* and enumerate*
\usepackage{graphicx}    % Importing graphics
\usepackage{hyperref}    % Hyperlink references and URLs
\usepackage[figure,table]{hypcap} % Hyperlink points to the top of figures
\usepackage[usenames,dvipsnames]{xcolor}	% Logo
\usepackage{tikz,ifthen}			% Logo
\usepackage{pgf}				% Logo
\usepackage{scalefnt}				% Logo
\usepgfmodule{shapes}				% Logo
\usepgfmodule{plot}				% Logo
\usetikzlibrary{shapes,arrows,shadows,fit}
\usepackage{pgf-umlsd}
\usepackage{multirow}
\usepackage{mdwlist}
\usepackage{colortbl}
\usepackage{calc}
\usepackage{float}
\usepackage{longtable}

\renewenvironment{itemize*}%
    {\begin{itemize}%
        \setlength{\itemsep}{0pt}%
        \setlength{\parskip}{0pt}%
        \setlength{\partopsep}{0pt}%
        \setlength{\topsep}{0pt}}%
    {\end{itemize}}

\newcommand{\operations}[1]{
\begin{center}
    \begin{longtable}{|p{4cm}|p{9cm + 2.0\tabcolsep}|}
    \hline
    \multicolumn{2}{|l|}{\cellcolor[gray]{0.5}{\textbf{Operations}}} \\ \hline
#1
    \end{longtable}
\end{center}
}
\newcommand{\operation}[4]{
    \hline
    \multicolumn{2}{|l|}{\cellcolor[gray]{0.8}{\texttt{#1}}} \\ \hline
    \hspace{7pt}\textbf{Input:} & #2 \\ \hline
    \hspace{7pt}\textbf{Output:} & #3 \\ \hline
    \hspace{7pt}\textbf{Description:} & #4 \\ \hline
}

\newcommand{\attributes}[1]{
    \begin{center}
        \begin{tabular}{|p{3cm}|p{3cm}|p{7cm}|}
            \multicolumn{3}{|l|}{\cellcolor[gray]{0.5}{\textbf{Attributes}}} \\ \hline
            \rowcolor[gray]{0.8} Name & Type & Description \\ \hline 
            #1
        \end{tabular}
    \end{center}
}

\newcommand{\attribute}[3]{
    \texttt{#1} & \texttt{#2} & #3 \\ \hline
}

% Just so we don't have to specify this twice
\newcommand\mytitle{Software Design Specification}
\newcommand\mydate{\today}
\newcommand\myversion{1}

\hypersetup{
    colorlinks=true,
    linkcolor=blue,
    urlcolor=blue,
    pdftitle={AHOY Software Design Specification V\myversion},
    pdfauthor={Dustin Ingram, Aaron Rosenfeld, Maria Kolakowska, Frank Clark}
}

% So we can number subsubsections too
\setcounter{secnumdepth}{5}

% For headers and footers
\setlength{\headheight}{15pt}
\setlength{\headsep}{25pt}
\pagestyle{fancy}
	
% Page style for the title page
\fancypagestyle{plain}{
    \fancyhf{}
    \renewcommand{\headrulewidth}{0pt}
    \renewcommand{\footrulewidth}{0pt}
}

% Page style for every other page
\fancyhf{} % clear all header and footer fields
\fancyhead[L]{AHOY}
\fancyhead[C]{\mytitle}
\fancyhead[R]{\mydate}
\fancyfoot[C]{\thepage}
\renewcommand{\headrulewidth}{0.4pt}
\renewcommand{\footrulewidth}{0.4pt}

\title{\textbf{\mytitle}}
\author{
	Frank Clark \\\url{francis.j.clark@drexel.edu}
    \and Dustin Ingram \\\url{dustin.s.ingram@drexel.edu}
	\and Maria Kolakowska \\\url{maria.j.kolakowska@drexel.edu}
    \and Aaron Rosenfeld \\\url{aaron.rosenfeld@drexel.edu}
}
\date{\mydate\\Version \myversion}

%%% Tikz stuff %%%
% Style for UML Class with attributes and methods
%BEGIN IMAGE
\tikzstyle{umlclass} = [
    rectangle, rectangle split, rectangle split parts=3,
    % This makes a nice gradient
    top color=white, bottom color=blue!30, draw=blue!50!black!100,
    drop shadow, rounded corners,
    node distance = 5cm, text width = 10cm, minimum width=10cm]
\tikzstyle{component} = [umlclass, rectangle split parts=0, text width =, node distance=3.14159cm, minimum width=]
% Line styles
\tikzstyle{hasa} = [draw, <-, >=open diamond]
\tikzstyle{ownsa} = [draw, ->, >=diamond]
\tikzstyle{isa} = [draw, ->, >=open triangle 45]

\newcommand{\umlclass}[5][,]{
    \node [umlclass,#1] (#2) {
        \textbf{#3}
        \nodepart{second}
        \begin{description*}
        #4
        \end{description*}
        \nodepart{third}
        \begin{description*}
        #5
        \end{description*}
    };
}

\newcommand{\eumlclass}[4][,]{
    \node [umlclass,#1] (#2) {
        \textbf{#3}
        \nodepart{second}
        \nodepart{third}
        \begin{description*}
        #4
        \end{description*}
    };
}

\newcommand{\umlattr}[3]{
    \item[$#1$]\textbf{#2}: \textit{#3}
}
\newcommand{\umlmethod}[4]{
    \item[$#1$]\textbf{#2}({#3}): \textit{#4}
}
\newcommand{\umlarg}[2]{\textit{#1} #2}

\newcommand{\umlrelation}[6][-|]{
    \path [#3] (#2) #1 (#4)
        node [very near start, auto=right] {#5}
        node [very near end, auto=right] {#6};
}
%END IMAGE
%%% End tikz stuff %%%

\begin{document}
\pagenumbering{roman}

\begin{figure*}
   % \vspace{-2em}
    \centering
    \scalebox{0.8}{\RequirePackage{fix-cm}
\documentclass{article}
\usepackage[usenames,dvipsnames]{xcolor}	% Logo
\usepackage{tikz,ifthen}			% Logo
\usepackage{pgf}				% Logo
\usepackage{scalefnt}				% Logo
\usepgfmodule{shapes}				% Logo
\usepgfmodule{plot}				% Logo
\usetikzlibrary{shapes,arrows,shadows,fit}

\usepackage[graphics,tightpage,active]{preview}
\usetikzlibrary{mindmap,trees}
\PreviewEnvironment{tikzpicture}
\begin{document}
\pagestyle{empty}
\begin{tikzpicture}[scale=1]
    \node[color=black] at (0,-2) {{\scalefont{11.0}\textbf{AHOY}}};
    \clip[] (0,6.5) circle (6.8cm);
    
    \coordinate (A) at (0,20);
    \coordinate (B) at (-12,-.4);
    \coordinate (C) at (12,-.4);
    \foreach \density in {20,30,...,160}{%
        \draw[fill=MidnightBlue!\density] (A)--(B)--(C)--cycle;
        \path
             (A) coordinate (X)
          -- (B) coordinate[pos=.15](A)
          -- (C) coordinate[pos=.15](B)
          -- (X) coordinate[pos=.15](C);
    }


    \draw[draw,black,line width=10pt] (0,6.5) circle (6.8cm);
    \draw[shorten >=2pt,line width=39pt,->] (0,1) -- (0,0);
    \draw[line width=30pt,o-] (0,13) -- (0,3);
    \draw[line width=30pt,o-o] (5,9) -- (-5,9);

    \draw[line width=20pt,line cap=round,-] (0,5) .. controls +(0,-3.5) and +(0,0) .. (3,2.5);
    \draw[line width=20pt,line cap=round,-] (0,5) .. controls +(0,-3.5) and +(0,0) .. (-3,2.5);
    \draw[line width=40pt,-] (0,10.5) -- (0,2);

    \draw[shorten <=15pt,line width=30pt,line cap=round,-latex] (0,0) .. controls +(0,3.5) and +(0,-6) .. (-4,7);
    \draw[shorten <=15pt,line width=30pt,line cap=round,-latex] (0,0) .. controls +(0,3.5) and +(0,-6) .. (4,7);
	
\end{tikzpicture}
\end{document}
}
    \vspace{-4em}
\end{figure*}

\maketitle

\begin{abstract}
AHOY is an event-based simulation environment used to compare the effectiveness of different combinations of software agents, network configurations, and sensor data in real-world environments.  It is comprised of a distributed simulation engine, visualizer, and programming interface through which developers create agent software and network topologies.  Communication between virtual nodes is also simulated, providing highly realistic scenarios.
\end{abstract}

\setcounter{tocdepth}{4}
\tableofcontents
\pagebreak
\listoffigures
\pagebreak
\pagenumbering{arabic}

\section{Introduction}
\subsection{Purpose}
This design document defines the architecture of the AHOY application. AHOY consists of several software components, including those which make up the simulator, the networking engine, event controller, and the agent and sensor frameworks. The information presented here is intended for the development team, as well as the advisor and external stakeholders, which are currently Dr.~William Regli, Joseph Macker of the Naval Research Laboratory, and Dr.~Michal P\v{e}chou\v{e}k of Czech Technical University. 

\subsection{Scope}
The goal of the AHOY project is to provide a system for testing multiple agents across varying scenarios and topologies, in a distributed, event-driven way. AHOY gives the user the ability to quantitatively examine the effectiveness of specific agent designs as well as a focus on additional factors relevant to the network, including network connectivity, connection fidelity, and the agent's ability to process and transmit data.

Users of AHOY are researchers looking to improve a specific agent's performance on a network through testing across varying combinations of topologies and scenarios.

\subsection{Design Goals}
AHOY is designed to be used as a testbed for sensors and respective algorithms in the laboratory against virtual operations using data from simulated sensors and nodes. It provides the ability to independently and quickly vary the network topology, application suites, environment, or any other variable is essential to collect data which would be cost-prohibitive to produce in a real-world scenario. 

The goals of AHOY's core design process is to provide a framework upon which researches can build custom agents, scenarios and topologies, and quickly and easily run an experiment using them. A modular distributed architecture allows for a wide scale of simulation size. Most importantly, the Event API allows developer to easily tailor custom user interfaces for existing applications or visualizers by providing a common interface to monitor every action within an executing experiment.

\subsection{Definitions, Acronyms, and Abbreviations}
\begin{description}
\item[Agent]
	Agents are simulated pieces of software that run on nodes in the network. They consist of different algorithms that are relevant for the user to test on different scenarios and topologies.   

\item[Distribution]
	Distribution refers to the process of distributing the simulation across a multi-platform physical cluster.  This allows the system to exceed the number of nodes per platform for a single simulation at the system's discretion.  A framework is provided to allow the user to distribute their simulation. 	

\item[Node]
	Nodes are virtual or physical machines that consist of agents and network interfaces.  If nodes are virtual, many nodes may run on one physical machine.  

\item[Scenario]
	Scenario is comprised of a scripted language indicating the location simulated nodes within the virtual world. These nodes consist of agents (see definition of `Agent') and non-agent world objects such as planes, boats, ground vehicles, etc. 

\item[Terrain]
	Terrain refers to the simulated landscape.  This includes such geography as the slope of the land, the tree density, water v.s. land surfaces, etc.

\item[Topology]
	Topology describes time-dependent connections between nodes and their characteristics (e.g. radio model). It is described with a scripting language which specifies the details of network interfaces on each simulated node, including radio models and throughput characteristics.  It describes any physical or wireless links that connect these interfaces.  Further, it indicates changes in linkage over time such as a wireless interface switches, wireless LANs, or a physical link being created or severed. 

\item[Visualizer]
	The Visualizer allows the simulations to be superimposed over real-world topography.  This permits the user to examine the behavior of the agents.  It also allows for overlays such as link quality, traffic rates, and other metrics deemed important to specific components.
\end{description}

\subsection{References%
  \label{references}%
}

These documents have been used as reference materials for various technologies involved with this project.
%
\begin{itemize*}
	\item SPEYES: Sensing and Patrolling Enablers Yielding Effective SASO: \url{http://ieeexplore.ieee.org/xpls/abs_all.jsp?arnumber=1559616}
	\item Service Sniffer Requirements Document: \url{http://servicesniffer.net/documents/requirements.html}
    \item Developing an Agent Systems Reference Architecture: \url{www.cs.drexel.edu/~dn53/papers/paper_cameraready.pdf}
\end{itemize*}

\section{Design Overview}
\subsection{Description of Problem}
The AHOY application bridges network and agent simulators. It must provide an entirely event-driven, real-time system to ultimately allow for human-in-the-loop interaction with agents. It must enable scenarios and network topologies to be easily defined, and similarly make writing custom agents extremely simple. The Event API must be comprehensive to provide flexibility for the user in terms of visualization and data collection. Finally, AHOY must be able to effortlessly distribute an experiment across physical machines, so that scaling limitations are not a user issue.

\subsection{Subsystem Diagram}
\begin{figure}%[!ht]
    \centering
    \begin{tikzpicture}[node distance=.5cm]
        \node [component] (Entity) {\textbf{Entities}};
        \node [component, right of=Entity] (World) {\textbf{World}};
        \node [component, right of=World] (Simulation) {\textbf{Simulation}};
        \node [component, below left of=Entity] (Agent) {\textbf{Agents}};
        \node [component, below right of=Entity] (Sensor) {\textbf{Sensors}};
        \node [component, below of=Entity] (Interface) {\textbf{Interfaces}};
        \draw [draw, <-, >=open diamond] (Simulation) -- (World);
        \draw [draw, ->, >=open diamond] (Entity) -- (World);
        \draw [draw, <-, >=open diamond] (Entity) -- (Agent);
        \draw [draw, <-, >=open diamond] (Entity) -- (Sensor);
        \draw [draw, <-, >=open diamond] (Entity) -- (Interface);
    \end{tikzpicture}
    \caption{Subsystem Diagram}
    \label{fig-subsystem}
\end{figure}

\subsection{Distribution Diagram}
\begin{figure}%[!ht]
    \centering
    \includegraphics[scale=.6]{../initial-pres/arch.pdf}
    \caption{Distribution Diagram}
    \label{fig-distribution}
\end{figure}

\subsection{User Interface Overview}
AHOY is intentionally designed to provide no specific user interface in the traditional sense. Instead, it provides an extensive and comprehensive API, for creating a simulation, interacting with an experiment as it is running, monitoring events from a global viewpoint, and recording the results of an experiment for analysis. This ultimately offers flexibility for the end user, as separate, pre-existing interfaces can be modified to support AHOY's software API, or simply created from scratch to explicitly satisfy the user's needs. 

\subsection{Technologies Used}
The AHOY project is written entirely in Python, an interpreted, general-purpose high-level programming language. It does not depend on any external or libraries or packages, aside from those provided by the standard library. These include: the \texttt{sys} and \texttt{time} modules for standard system functions, the \texttt{math} library for advanced calculations, the \texttt{threading}, \texttt{signal} and \texttt{subprocess} for managing the distributed system, and the \texttt{pickle}, \texttt{socket} and \texttt{struct} modules for organizing the multicast event channel.

\subsection{Design Motivations}
\paragraph{Communication}{The method by which distributed nodes communicate must provide fast, reliable, in-order delivery of event information.  Many other architecutures use pair-wise TCP connections to guarantee reliability and ordered delivery, but result in significant overhead from the $n \cdot (n-1)$ connections with $n$ distributed nodes.  This results in poor scaling with large numbers of nodes.  In AHOY, a multicast event channel is instead used.  This results in no additional overhead as new distributed nodes are added.  Further, connections need not be maintained between nodes.  Due to the nature of multicast, it is possible that messages will be dropped or may arrive out of order.  However, many other frameworks, including EMANE\footnote{http://labs.cengen.com/emane/} utilize multicast event channels with success.  Further, AHOY assumes all distributed nodes are running on the same subnet. In this situation, packet loss and delay is generally minimal.  In the event that messages are dropped or arrive out of order, AHOY can utilize NRL's NORM\footnote{http://cs.itd.nrl.navy.mil/work/norm/} protocol which provides the same QoS guarantees as TCP.}

\section{Detailed System Design}

\subsection{Simulator Components}
\subsubsection{Simulation}
{ The Simulation starts the entire process of sending and receiving events.  It maintains the list of startup daemons running on each node. In addition, it initializes the world within the simulation along with the
entities running in it. }


\attributes{
    \attribute{\_event\_api}{EventAPI}{Instance of the Event API }
    \attribute{\_startup\_acks}{list(int)}{A list of startup daemon ID's}
    \attribute{\_comms\_module}{CommsEngine}{Communication Engine used to send and receive messages}
    \attribute{\_world}{World}{Instance of the world maintained by the simulation}
}


\operations{
    \operation{\_\_init\_\_(world\_inst, comms\_module}
    {
        \begin{itemize*}
            \item \texttt{world\_inst}: Instance of the world. 
            \item \texttt{comms\_module}: Communication Engine for sending/receiving messages.
        \end{itemize*}
    }{None}{Instantiates the Simulation}
    \operation{get\_world()}{None}{World}{Gets the World object being maintained}
    \operation{get\_on\_ack\_startup(event)}{Event containing the startup daemon id.}{None}{Adds a startup daemon id to the \_startup\_acks list}
    \operation{start(wait)}{Number of seconds to wait before starting.}{None}{Starts the Simulation. Startup daemons are alerted and the World and the entities within it are initialized}
    \operation{stop()}{None}{None}{Stops the simulation. A StopSimulationEvent event is published.}
}


\subsubsection{StartupDaemon}
{Startup Daemons maintain processes running on distributed nodes. It is responsible for starting and terminating these processes when appropriate. }

\attributes{
    \attribute{\_event\_api}{EventAPI}{Instance of the Event API. }
    \attribute{\_phys\_id}{int}{Integer ID assigned to the daemon.}
    \attribute{\_running\_pids}{int}{Process id's maintained by the daemon.}
    \attribute{\_running}{boolean}{True if the daemon is running. }
    \attribute{\_world}{World}{Instance of the World}
}

\operations{
    \operation{\_\_init\_\_(phys\_id)}{The integer id of the daemon.}{None}{Instantiates the StartupDaemon.}
    \operation{\_on\_startup(event)}{Event passed to the daemon on startup}{None}{Subscribes the daemon to the StartSimulationEvent and publishes an AckStartupEvent.  It also initializes the \_world attribute. }
    \operation{terminate\_all()}{None}{None}{Kills all processes maintained in its list \_running\_pids.}
    \operation{\_start\_entity\_processes(entity)}{Entity to start and maintain.}{None}{Starts the entity passed in and adds its process id to the list \_running\_pids.}
    \operation{\_on\_sim\_start(event)}{StartSimulationEvent published at the start of the simulation.}{None}{Grabs the list of entities to maintain from event and calls \_start\_entity\_process with each one.}
    \operation{start()}{None}{None}{Starts the daemon. Subscribes to StartupEvent and StopSimulationEvent events.}
    \operation{\_restart(event)}{Event that triggered the call to \_restart.}{None}{Resets the state of the daemon. All processes maintained are cleared.}
    \operation{stop()}{None}{None}{Stops the daemon and sets \_running to false.}
}

\subsubsection{World}
{The World maintains the entities running in the simulation.  It also maintains the networks being using within the simulation.}

\attributes{
    \attribute{\_entities}{dict(int : Entity)}{Maps entity ids to entities}
    \attribute{\_networks}{dict(String : Network)}{Maps network names to the network}
    \attribute{\_event\_api}{EventAPI}{Event API for subscribing to and publish events}
}

\operations{
    \operation{\_\_init\_\_()}{None}{None}{Instantiates the World.}
    \operation{add\_entity(entity)}
    { \texttt{entity}: Entity to be added to list of entities. }
    {None}{Adds an entity to \_entities.}
    \operation{get\_entities()}{None}{dict(int : Entity)}{Returns the dictionary of entities}
    \operation{get\_entity(uid)}
    {
        \texttt{uid}: Unique entity id. 
    }{Entity}{Returns the entity that has id uid.}
    \operation{add\_network(network)}
    {
        \texttt{network}: Network to add to the dictionary of networks.
    }{None}{Adds the given network to the dictionary of networks. }
    \operation{get\_networks()}{None}{dict(String : Network)}{Returns the dictionary of networks}
    \operation{get\_network(name)}
    {
        \texttt{name}: Name of the network. 
    }{Entity}{Returns the network whose name is the same as the parameter}
    \operation{initialize()}{None}{None}{Initializes the World. Instantiates \_event\_api, starts it, and subscribes the World to events.}
    \operation{\_on\_entity\_move(event)}
    {
        \texttt{event}: EntityMoveEvent event with information on new entity values. 
    }{None}{Finds the entity in \_entities given by the event and updates that entities position, forward velocity, and linear velocity.}
    \operation{pickle()}{None}{Serialized World}{Serializes the World with base 64 encoding.}
    \operation{from\_pickle(pickled)}
    {
        \texttt{pickled}: Serialized version of the World. 
    }{World}{Deserializes a pickled verson of the World.}
}
    
\subsubsection{EventAPI} % TODO: Create UML for this
{The EventAPI is used to communicate events across the event channel.  It allows entities to subscribe to certain events or publish the occurrence of events. When an event occurs, it will handle any function calls an entity wishes to make during that event. }

\attributes{
    \attribute{\_subscriptions}{dictionary}{Holds event subscriptions for an entity. Keyed by event type. The values are method calls to be made on an event. }
    \attribute{\_running}{boolean}{True if the EventAPI is running.}
    \attribute{\_tcp\_conn}{Socket}{A socket, designated by the user, for EventAPI to communication over}
    \attribute{\_sock}{Socket}{A socket for EventAPI to communicate over}
    \attribute{\_ip}{string}{IP address to communicate over}
    \attribute{\_port}{int}{port number to communicate over}
}

\operations{
    \operation{\_\_init\_\_(tcp\_conn=None)}{tcp\_conn: A socket for the EventAPI to communicate over}{None}{Instantiates an EventAPI}
    \operation{\_setup\_mc()}{None}{None}{Initializes the multicast channel to communicate on.}
    \operation{publish(event,dealy\_sec=None)}
    {
        \begin{itemize*}
            \item \texttt{event}: The event to be published. 
            \item \texttt{delay}: Number of seconds to wait before publishing. 
        \end{itemize*}
    }{None}{Publishes the occurrence of an event over the event channel.}
    \operation{\_delay(event, delay\_sec=None)}
    {
        \begin{itemize*}
            \item \texttt{event}: The event to delay.
            \item \texttt{delay\_sec}: Number of seconds to delay the event. 
        \end{itemize*}
    }{None}{Delays an event for a given number of seconds.}
    \operation{push\_raw(raw)}{\texttt{raw}: Data to send over a connection. }{None}{Sends raw data over the current connection. }
    \operation{subscribe(event\_type, callback, **args)}
    {
        \begin{itemize*}
            \item \texttt{event\_type}: Event type being subscribed to. 
            \item \texttt{callback}: Method to call on event event\_type.
            \item \texttt{**args}: arguments to pass to callback.
        \end{itemize*}
    }{None}{Sets a callback function to be made at a given event type.}
    \operation{unsubscribe\_all(event\_type)}{\texttt{event\_type}: The type of event to remove all subscriptions from.}
    {None}{Removes all subscriptions to the given event type.}
    \operation{clear\_subscriptions()}{None}{None}{Removes all subscriptions.}
    \operation{\_process(data)}{ \texttt{data}: Serialized event information.}
    {None}{Processes which callbacks need to be made based on the given data.}
    \operation{run()}{None}{None}{Listens for events over the socket and processes them.}
    \operation{\_tcp\_assemble()}{None}{Packet}{Builds a packet to be sent over the network.}
    \operation{start()}{None}{Thread}{Runs the EventAPI in a thread and returns the thread.}
    \operation{stop()}{None}{None}{Stops the EventAPI from running. }
}
    
\subsubsection{Entity}
\attributes{
    \attribute{MAX\_DISTANCE}{const float}{Maximum distance the entity can move in kilometers before publishing a move event.}
    \attribute{\_uid}{int}{Unique user id of the entity.}
    \attribute{\_lat}{float}{Current latitude.}
    \attribute{\_long}{float}{Current longitude.}
    \attribute{\_agl}{float}{Distance above ground level.}
    \attribute{\_rcs}{float}{Approximate cross-sectional area.}
    \attribute{\_event\_api}{EventAPI}{The EventAPI for an entity.}
    \attribute{\_world}{World}{The World instance.}
    \attribute{\_velocity}{list(float)}{The vector velocity of the entity.}
    \attribute{\_forward\_velocity}{float}{The forward velocity of the entity.}
}

\operations{
    \operation{set\_rcs(rcs)}{\texttt{rcs}: Cross sectional-area}{None}{Sets the cross-sectional area.}
    \operation{get\_rcs()}{None}{Cross-sectional area}{The cross-sectional area.}
    \operation{get\_lin\_velocity()}{None}{list(float)}{Returns the velocity as a vector of three numbers.}
    \operation{get\_forward\_velocity()}{None}{float}{Returns the forward velocity.}
    \operation{set\_lin\_velocity(v)}{\texttt{v}: A vector of three floating point numbers describing the velocity.}{None}{Sets the vector velocity of the entity.}
    \operation{set\_forward\_velocity(v)}{\texttt{v}: Floating point number to assign to the forward velocity.}{None}{Sets the forward velocity of the entity.}
    \operation{set\_world(world)}{\texttt{world}: World instance}{None}{Sets the World instance for the entity}
    \operation{get\_world()}{None}{World}{Returns the World instance.}
    \operation{set\_position(lat,long,agl)}
    {
        \begin{itemize*}
            \item \texttt{lat}: Latitude to assign the entity. 
            \item \texttt{long}: Longitude to assign the entity. 
            \item \texttt{agl}: Above ground location to assign the entity. 
        \end{itemize*}
    }{None}{Sets the latitude, longitude, and above ground location for the entity.}
    \operation{move(lat,lon,agl,forward\_vel,vert\_vel,block=False)}
    {
        \begin{itemize*}
            \item \texttt{lat}: Latitude to move to.
            \item \texttt{lon}: Longitude to move to.
            \item \texttt{agl}: Above ground location to move to. 
            \item \texttt{forward\_vel}: Forward velocity to move with.
            \item \texttt{vert\_vel}: Vertical velocity to move with. 
            \item \texttt{block}: If true, moves the entity in a new thread.
        \end{itemize*}
    }{None}{Moves the entity to the given coordinates.}
    \operation{\_iterative\_move(lat,lon,agl,forward\_vel,vert\_vel)}
    {
        \begin{itemize*}
            \item \texttt{lat}: Latitude to move to.
            \item \texttt{lon}: Longitude to move to.
            \item \texttt{agl}: Above ground location to move to. 
            \item \texttt{forward\_vel}: Forward velocity to move with.
            \item \texttt{vert\_vel}: Vertical velocity to move with. 
        \end{itemize*}
    }{None}{Moves the entity to the given coordinates in steps by smaller distances using its velocity and MAX\_DISTANCE.}
    \operation{get\_position()}{None}{list(float)}{Returns a list of the latitude, longitude, and above ground location in that order.}
    \operation{pickle()}{None}{Serialized Entity}{Serializes the entity.}
    \operation{from\_pickle(pickled)}{\texttt{pickled}:Serialized entity.}{Entity}{Deserializes an entity.}
    \operation{get\_event\_api()}{None}{EventAPI}{Returns the entities EventAPI.}
    \operation{get\_uid()}{None}{int}{Returns the entities unique id.}
    \operation{initialize()}{None}{None}{Initializes the entity.}
}

\subsubsection{Built-in Entities}
\paragraph{Node}{Nodes are processes that aggregate agents. Each one has an interface that handles communication to the outside world (i.e. other nodes or agents). They inherit properties from the Entity class.}

\attributes{
    \attribute{\_interfaces}{dict(String : Interface)}{A dictionary mapping the name of an network interface to the interface itself. }
    \attribute{\_agents}{dict(int : Agent)}{A dictionary mapping agent ids to the agents themselves.}
}

\operations{
    \operation{\_\_init\_\_(uid)}{\texttt{uid}: Unique id assigned to the node.}{None}{Instantiates the Node.}
    \operation{get\_agent\_uids()}{None}{list(int)}{Returns a list of all agent ids belonging to that node.}
    \operation{add\_interface(interface)}{\texttt{interface}: Network interface to add to the node.}{None}{Adds a network interface to the node.}
    \operation{remove\_interface(name)}{\texttt{name}: The name of the interface to remove.}{None}{Removes an network interface from the node.}
    \operation{get\_interface(name)}{\texttt{name}: The name of the interface to return.}{Interface}{Returns a network interface with the given name.}
    \operation{get\_interfaces()}{None}{list(Interface)}{Returns a list of all network interfaces belonging to the node.}
    \operation{add\_agent(agent\_inst)}{\texttt{agent\_inst}: Agent to add to the node.}{None}{Adds an agent to the list of agents owned by the node.}
    \operation{get\_interface\_on\_net(network\_name)}{\texttt{network\_name}:Network name associated with an interface.}{None}{Returns the interface associated with the given network name.}
    \operation{send(message,src\_agent)}
    {
        \begin{itemize*}
            \item \texttt{message}: Message to send. 
            \item \texttt{src\_agent} Agent sending the message.
        \end{itemize*}
    }{None}{Sends a message from a specified agent over a node interface.}
    \operation{\_on\_message(event,**kwds)}
    {
        \begin{itemize*}
            \item \texttt{event}: An event that has occurred. 
            \item \texttt{**kwds}: Arguments associated with the callback for an event.
        \end{itemize*}
    }{None}{Passes an event on to the agent it is meant to reach.}
    \operation{run()}{None}{None}{Starts the node. Interfaces and agents are all initialized}
}

\paragraph{ScriptedEntity}
\subsubsection{Startup Sequence}
\begin{figure}[!htb]
  \centering

  \begin{sequencediagram}{3.5cm}
    \newthread{sim}{Simulation}{Simulation}
    \newthread{mc}{Event Channel}{EventChannel}
    \newthread{startup1}{StartupDaemon 1}{StartupDaemon}
    \newthread{startupn}{StartupDaemon $n$}{StartupDaemon}

    \mess{sim}{StartupEvent}{mc}
    \begin{callself}{sim}{time.sleep()}{}
        \mess{mc}{StartupEvent}{startup1}
        \mess{mc}{\hspace*{3cm}StartupEvent}{startupn}
        \mess{startup1}{AckStartupEvent}{mc}
        \mess{mc}{AckStartupEvent}{sim}
        \mess{startupn}{\hspace*{3cm}AckStartupEvent}{mc}
        \mess{mc}{AckStartupEvent}{sim}
    \end{callself}
    \begin{callself}{sim}{allocate\_entities()}{}
    \end{callself}
    \mess{sim}{StartSimulationEvent}{mc}
    \mess{mc}{StartSimulationEvent}{startup1}
    \begin{callself}{startup1}{start\_entities()}{}
    \end{callself}
    \mess{mc}{\hspace*{3.7cm}StartSimulationEvent}{startupn}
    \begin{callself}{startupn}{start\_entities()}{}
    \end{callself}
  \end{sequencediagram}

  \caption{Startup sequence}
  \label{fig-startupseq}
\end{figure}
\subsection{Networking Components}
\begin{figure}[!htb]
    \centering

    \begin{sequencediagram}{4cm}
        \newthread{node}{Node}{Node}
        \newthread{mc}{Event Channel}{Event Channel}
        \newthread{ce}{CommsEngine}{CommsEngine}

        \mess{node}{CommunicationSendEvent}{mc}
        \mess{mc}{CommunicationSendEvent}{ce}
        \begin{callself}{ce}{\_on\_send}{}
        \end{callself}
        \begin{sdloop}{alt [send successful]}
            \mess{ce}{CommunicationRecvEvent}{mc}
            \mess{mc}{CommunicationRecvEvent}{node}
        \end{sdloop}
    \end{sequencediagram}

    \caption{Messaging sequence diagram}
    \label{fig-messageseq}
\end{figure}
\subsubsection{Interface}
\subsubsection{CommsEngine}

\subsubsection{Built-in CommsEngines}
\paragraph{LogLossCommsEngine}
\paragraph{EthernetCommsEngine}

\subsection{Event Components}
\subsubsection{Event}

\subsubsection{Built-in Events}
Built-in events are specific Event classes provided by default to any emulation. These embody the basic events for a network simulation, but users are encouraged to develop and add their own as well.
\paragraph{LinkEvent}{Published when a network link between two entities changes within the simulation.}

\attributes{
    \attribute{\_up}{bool}{Whether the link is currently up (connected)}
    \attribute{\_uid\_1}{int}{UID of the first node which makes up the link}
    \attribute{\_uid\_2}{int}{UID of the second node which makes up the link}
    \attribute{\_network\_name}{string}{The network that the link exists on}
    \attribute{\_pathloss}{float}{Pathloss value for wireless links}
}

\operations{
    \operation{\_\_init\_\_(up, uid\_1, uid\_2, network\_name, pathloss)}
    {
        \begin{itemize*}
            \item \texttt{up}: Whether the link is up
            \item \texttt{uid\_1}: The UID of the first node
            \item \texttt{uid\_2}: The UID of the second node
            \item \texttt{network\_name}: The name of the network
            \item \texttt{pathloss}: The pathloss value, if relevant
        \end{itemize*}
    }{None}{Instantiates a new link action}
    \operation{get\_up()}{None}{Link State}{Gets whether or not the link is connected}
    \operation{get\_uid1()}{None}{First Agent UID}{Gets the UID of the first node in the link}
    \operation{get\_uid2()}{None}{Second Agent UID}{Gets the UID of the second node in the link}
    \operation{get\_network\_name()}{Network Name}{None}{Gets the name of the network}
    \operation{get\_pathloss()}{None}{Pathloss Value}{Gets the pathloss value for wireless links}
}

\paragraph{CommunicationSendEvent}{Published when an entity sends a message within the simulation.}

\attributes{
    \attribute{\_src\_agent\_uid}{int}{UID of the entity that is the source of the message}
    \attribute{\_src\_iface\_name}{string}{Interface name that the message is sent on}
    \attribute{\_message}{Message}{An instance of the message that is sent}
    \attribute{\_network}{string}{The network that the interface is a part of}
}

\operations{
    \operation{\_\_init\_\_(src\_agent\_uid, src\_iface\_name, message\_inst, network)}
    {
        \begin{itemize*}
            \item \texttt{src\_agent\_uid}: UID of the sending agent
            \item \texttt{src\_iface\_name}: Name of the interface
            \item \texttt{message\_inst}: Instance of the message being sent
            \item \texttt{network}: Name of the network
        \end{itemize*}
    }{None}{Instantiates a new communication send action}
    \operation{get\_src\_agent\_uid()}{None}{Source Agent UID}{Gets the UID of the source entity that sent the message}
    \operation{get\_src\_iface\_name()}{None}{Source Interface Name}{Gets the name of the interface the message was sent on}
    \operation{get\_message()}{None}{Message}{Gets the entirety of the message}
    \operation{get\_network()}{None}{Network Name}{Gets the network the message was sent on}
}

\paragraph{CommunicationRecvEvent}{Published when an entity receives a message within the simulation.}

\attributes{
    \attribute{\_src\_agent\_uid}{int}{UID of the entity that is the source of the message}
    \attribute{\_src\_iface\_name}{string}{Interface name that the message is sent on}
    \attribute{\_message}{Message}{An instance of the message that is sent}
    \attribute{\_network}{string}{The network that the interface is a part of}
}

\operations{
    \operation{\_\_init\_\_(src\_agent\_uid, src\_iface\_name, message\_inst, network)}
    {
        \begin{itemize*}
            \item \texttt{src\_agent\_uid}: UID of the sending agent
            \item \texttt{src\_iface\_name}: Name of the interface
            \item \texttt{message\_inst}: Instance of the message being sent
            \item \texttt{network}: Name of the network
        \end{itemize*}
    }{None}{Instantiates a new communication receive action}
    \operation{get\_src\_agent\_uid()}{None}{Source Agent UID}{Gets the UID of the source entity that sent the message}
    \operation{get\_src\_iface\_name()}{None}{Source Interface Name}{Gets the name of the interface the message was sent on}
    \operation{get\_message()}{None}{Message}{Gets the entirety of the message}
    \operation{get\_network()}{None}{Network Name}{Gets the network the message was sent on}
}

\paragraph{EntityMoveEvent}{Published when an entity moves within the simulation.}

\attributes{
    \attribute{\_entity\_uid}{int}{UID of the entity that moved}
    \attribute{\_lat}{float}{Latitude of new location}
    \attribute{\_lon}{float}{Longitude of new location}
    \attribute{\_agl}{float}{AGL of new location}
    \attribute{\_velocity}{float}{Velocity of movement}
}

\operations
{
    \operation{\_\_init\_\_(entity, lat, lon, agl, velocity)}
    {
        \begin{itemize*}
            \item \texttt{entity}: Instance of entity that moved
            \item \texttt{lat}: Latitude of new location
            \item \texttt{lon}: Longitude new location
            \item \texttt{agl}: AGL of new location
            \item \texttt{velocity}: Velocity of movement
        \end{itemize*}
    }{None}{Instantiates a new movement action}
    \operation{get\_uid()}{None}{UID}{Gets the UID of the entity that moved}
    \operation{get\_lat()}{None}{Latitude}{Gets the new latitude of the entity that moved}
    \operation{get\_lon()}{None}{Longitude}{Gets the new longitude of the entity that moved}
    \operation{get\_agl()}{None}{AGL}{Gets the new AGL of the entity that moved}
    \operation{get\_velocity()}{None}{Velocity}{Gets the velocity of the entity that moved}
}

\paragraph{StartupEvent}
\paragraph{AckStartupEvent}
\paragraph{StartSimulationEvent}
\paragraph{StopSimulationEvent}

\subsection{Agent Components}
\subsubsection{Agent}

\subsubsection{Condition}

\subsubsection{Action}
\subsubsection{Built-in Actions}
\paragraph{MoveAction}

\subsection{Sensor Components}
\subsubsection{Sensor}

\subsubsection{Built-in Sensors}
\paragraph{RadarSensor}

\subsection{Utility Components}
\subsubsection{Geo}
\operations{
    \operation{haver\_distance(lat1, lon1, lat2, lon2)}
    {
        \begin{itemize*}
            \item \texttt{lat1}: Latitude of first point
            \item \texttt{lon1}: Longitude of first point
            \item \texttt{lat2}: Latitude of second point
            \item \texttt{lon2}: Longitude of second point
        \end{itemize*}
    }{Haversine distance between two points}{Determines the ``as-the-bird-flies'' distance between to points.}
    \operation{linear\_distance(lat1, lon1, lat2, lon2)}
    {
        \begin{itemize*}
            \item \texttt{lat1}: Latitude of first point
            \item \texttt{lon1}: Longitude of first point
            \item \texttt{lat2}: Latitude of second point
            \item \texttt{lon2}: Longitude of second point
        \end{itemize*}
    }{Linear distance between two points}{Determines the linear distance between two points.}
    \operation{sph\_to\_lin(lat, lon, agl)}
    {
        \begin{itemize*}
            \item \texttt{lat}: Latitude of point
            \item \texttt{lon}: Longitude of point
            \item \texttt{agl}: AGL of point
        \end{itemize*}
    }{\texttt{(x,y,z)} tuple of location}{Converts latitude, longitude, and altitude into Cartesian coordinates.}
    \operation{lin\_to\_sph(x, y, z)}
    {
        \begin{itemize*}
            \item \texttt{x}: $x$-coordinate of point
            \item \texttt{y}: $y$-coordinate of point
            \item \texttt{z}: $z$-coordinate of point
        \end{itemize*}
    }{\texttt{(lat, lon, agl)} tuple of location}{Converts Cartesian coordinates into latitude, longitude, and altitude.}
    \operation{lin\_to\_degree(lat, lon, lat\_km, lon\_km)}
    {
        \begin{itemize*}
            \item \texttt{lat}: Latitude of point
            \item \texttt{lon}: Longitude of point
            \item \texttt{lat\_km}: Kilometers along latitude line
            \item \texttt{lon\_km}: Kilometers along longitude line
        \end{itemize*}
    }{Radians of movement}{Converts distance in kilometers to radians at a given point on the Earth.}
    \operation{loc\_from\_bearing\_dist(lat, lon, bearing, dist)}
    {
        \begin{itemize*}
            \item \texttt{lat}: Latitude of point
            \item \texttt{lon}: Longitude of point
            \item \texttt{bearing}: Bearing of movement
            \item \texttt{dist}: Distance of movement in kilometers
        \end{itemize*}
    }{\texttt{(lat, lon)} tuple of new location}{Determines location after moving from a point, along a bearing, for a specific distance.}
}

\section{Appendix}
\begin{figure}[!H]
    \centering
    \begin{tikzpicture}
        \umlclass[xshift=-2cm]{simulation}{Simulation}{
            \umlattr{\#}{\_event\_api}{EventAPI}
            \umlattr{\#}{\_startup\_acks}{set(int)}
            \umlattr{\#}{\_comms\_engine}{CommsEngine}
            \umlattr{\#}{\_world}{World}
        }{
            \umlmethod{+}{\_\_init\_\_}{world\_inst : World, comms\_engine : CommsEngine}{void}
            \umlmethod{\#}{\_on\_ack\_startup}{event : Event}{void}
            \umlmethod{+}{get\_world}{}{World}
            \umlmethod{+}{start}{wait : int}{void}
            \umlmethod{+}{stop}{}{void}
        }
        \umlclass[below of=simulation, yshift=-2cm]{world}{World}{
            \umlattr{\#}{\_entities}{dict(String : Entity)}
            \umlattr{\#}{\_networks}{dict(String : Network)}
            \umlattr{\#}{\_event\_api}{EventAPI}
            \umlattr{\#}{\_agent\_mapping}{dict(int : int)}
        }{
            \umlmethod{+}{\_\_init\_\_}{}{void}
            \umlmethod{+}{\underline{from\_pickle}}{pickled : String}{World}
            \umlmethod{\#}{\_on\_entity\_move}{event : Event}{void}
            \umlmethod{+}{add\_entity}{entity : Entity}{void}
            \umlmethod{+}{get\_entity}{}{Entity}
            \umlmethod{+}{get\_entities}{}{list(Entity)}
            \umlmethod{+}{add\_network}{}{void}
            \umlmethod{+}{get\_network}{String}{Network}
            \umlmethod{+}{get\_networks}{}{list(Network)}
            \umlmethod{+}{get\_agent\_mapping}{}{dict(int : int)}
            \umlmethod{+}{initialize}{}{void}
            \umlmethod{+}{pickle}{}{String}
        }
        \path [draw, <-, >=open diamond] (simulation) -- (world)
            node [very near start, auto=right] {}
            node [very near end, auto=right] {};
    \end{tikzpicture}
    \caption{Class diagram for simulation}
    \label{fig-simulation}
\end{figure}

\begin{figure}
    \centering
    \begin{tikzpicture}
        \umlclass[xshift=-2cm]{startup}{StartupDaemon}{
            \umlattr{\#}{\_event\_api}{EventAPI}
            \umlattr{\#}{\_phys\_id}{int}
            \umlattr{\#}{\_running\_pids}{set(int)}
            \umlattr{\#}{\_running}{bool}
        }{
            \umlmethod{+}{\_\_init\_\_}{phys\_id : int}{void}
            \umlmethod{\#}{\_on\_startup}{event : Event}{void}
            \umlmethod{\#}{\_start\_entity\_process}{entity : Entity}{void}
            \umlmethod{\#}{\_on\_sim\_start}{event : Event}{void}
            \umlmethod{\#}{\_restart}{event : Event}{void}
            \umlmethod{+}{terminate\_all}{}{void}
            \umlmethod{+}{start}{}{void}
            \umlmethod{+}{stop}{}{void}
        }
    \end{tikzpicture}
    \caption{Class diagram for startup daemon}
    \label{fig-startupdaemon}
\end{figure}

\begin{figure}
    \centering
    \begin{tikzpicture}
        \umlclass[xshift=-2cm]{entity}{Entity \textit{$<<$Abstract$>>$}}{
            \umlattr{\#}{\_uid}{int}
            \umlattr{\#}{\_lat}{float}
            \umlattr{\#}{\_lon}{float}
            \umlattr{\#}{\_agl}{float}
            \umlattr{\#}{\_rcs}{float}
            \umlattr{\#}{\_event\_api}{EventAPI}
            \umlattr{\#}{\_world}{World}
            \umlattr{\#}{\_velocity}{(float, float, float)}
            \umlattr{\#}{\_sensors}{dict(String : Sensor)}
        }{
            \umlmethod{+}{\_\_init\_\_}{phys\_id : int}{void}
            \umlmethod{+}{\underline{from\_pickle}}{String}{Entity}
            \umlmethod{\#}{\_iterate\_move}{pos : (float, float, float), vel : float}{void}
            \umlmethod{+}{add\_sensor}{name : String, sensor : Sensor}{void}
            \umlmethod{+}{get\_sensor}{name : String}{Sensor}
            \umlmethod{+}{get\_rcs}{}{float}
            \umlmethod{+}{set\_rcs}{float}{void}
            \umlmethod{+}{get\_velocity}{}{(float, float, float)}
            \umlmethod{+}{set\_velocity}{vel : (float, float, float)}{void}
            \umlmethod{+}{get\_world}{}{World}
            \umlmethod{+}{set\_world}{world : World}{}
            \umlmethod{+}{get\_position}{}{(float, float, float)}
            \umlmethod{+}{set\_position}{pos : (float, float, float)}{}
            \umlmethod{+}{move}{pos : (float, float, float)}{}
            \umlmethod{+}{pickle}{}{String}
            \umlmethod{+}{get\_event\_api}{}{EventAPI}
            \umlmethod{+}{get\_uid}{}{int}
            \umlmethod{+}{initialize}{}{void}
            \umlmethod{+}{\textit{run}}{}{void}
        }

        \umlclass[minimum width=5cm, text width=8cm, yshift=-7cm, below left of=entity]{node}{Node}{
            \umlattr{\#}{\_interfaces}{list(Interface)}
            \umlattr{\#}{\_agents}{list(Agent)}
        }{
            \umlmethod{+}{\_\_init\_\_}{uid : int}{void}
            \umlmethod{\#}{\_on\_message}{event : Event, **kwds : dict}{void}
            \umlmethod{+}{get\_agent\_uids}{}{list(int)}
            \umlmethod{+}{add\_interface}{interface : Interface}{void}
            \umlmethod{+}{remove\_interface}{name : String}{void}
            \umlmethod{+}{get\_interface}{name : String}{Interface}
            \umlmethod{+}{get\_interfaces}{}{list(Interface)}
            \umlmethod{+}{add\_agent}{agent : Agent}{void}
            \umlmethod{+}{get\_interface\_on\_net}{network\_name : String}{Interface}
            \umlmethod{+}{send}{message : Message, src\_agent : Agent}{void}
            \umlmethod{+}{run}{}{void}
        }
        \umlclass[minimum width=5cm, text width=5cm, yshift=-6cm, below right of=entity]{scripted}{ScriptedEntity}{
            \umlattr{\#}{\_waypoints}{list((float, float, float))}
            \umlattr{\#}{\_vel}{float}
        }{
            \umlmethod{+}{\_\_init\_\_}{uid : int, waypoints : list((float, float, float)), vel : float}{void}
            \umlmethod{+}{run}{}{void}
        }
        \path [draw, <-, >=open triangle 45] (entity) -- (node)
            node [very near start, auto=right] {}
            node [very near end, auto=right] {};
        \path [draw, <-, >=open triangle 45] (entity) -- (scripted)
            node [very near start, auto=right] {}
            node [very near end, auto=right] {};
    \end{tikzpicture}
    \caption{Class diagram for entities}
    \label{fig-entities}
\end{figure}
\begin{figure}
    \centering
    \begin{tikzpicture}
        \umlclass[]{interface}{Interface}{
            \umlattr{\#}{\_name}{String}
            \umlattr{\#}{\_owner\_node}{Node}
            \umlattr{\#}{\_recv\_callback}{function}
            \umlattr{\#}{\_network\_name}{String}
            \umlattr{\#}{\_power}{float}
        }{
            \umlmethod{+}{\_\_init\_\_}{name : String, owner\_node : Node, network : Network, power : float}{void}
            \umlmethod{\#}{\_on\_communication}{event : Event}{void}
            \umlmethod{+}{connect}{}{void}
            \umlmethod{+}{send}{message\_inst : Message, src\_agent : int}{void}
            \umlmethod{+}{set\_recv\_callback}{cb : function}{void}
            \umlmethod{+}{get\_owner}{}{Node}
            \umlmethod{+}{get\_network\_name}{}{String}
            \umlmethod{+}{get\_name}{}{String}
            \umlmethod{+}{get\_power}{}{float}
        }
    \end{tikzpicture}
    \caption{Class diagram for interfaces}
    \label{fig-interface}
\end{figure}
\begin{figure}
    \centering
    \begin{tikzpicture}
        \umlclass[]{comms}{CommsEngine \textit{$<<$Abstract$>>$}}{
            \umlattr{\#}{\_event\_api}{EventAPI}
            \umlattr{\#}{\_simulation}{Simulation}
        }{
            \umlmethod{+}{\_\_init\_\_}{}{void}
            \umlmethod{\#}{\textit{\_on\_send}}{event : Event}{void}
            \umlmethod{+}{get\_node\_from\_agent}{agent\_uid : int}{int}
            \umlmethod{+}{set\_simulation}{simulation : Simulation}{void}
            \umlmethod{+}{get\_event\_api}{}{EventAPI}
            \umlmethod{+}{get\_simulation}{}{Simulation}
        }
        \eumlclass[minimum width=5cm, text width=6cm, below left of=comms, yshift=-2cm]{logloss}{LogLossCommsEngine}
        {
            \umlmethod{+}{\_\_init\_\_}{}{void}
            \umlmethod{\#}{\_on\_send}{event : Event}{void}
            \umlmethod{\#}{\_on\_movement}{event : Event}{void}
            \umlmethod{\#}{\_get\_rx\_power}{src\_uid : int, dest\_uid : int, src\_power : float}{float}
            \umlmethod{\#}{\_should\_deliver}{src\_uid : int, dest\_uid : int, src\_power : float, dest\_sensitivity : float}{bool}
        }
        \eumlclass[minimum width=5cm, text width=6cm, below right of=comms, yshift=-2cm]{ethernet}{EthernetCommsEngine}
        {
            \umlmethod{+}{\_\_init\_\_}{}{void}
            \umlmethod{\#}{\_on\_send}{event : Event}{void}
            \umlmethod{\#}{\_on\_movement}{event : Event}{void}
        }
        \path [draw, <-, >=open triangle 45] (comms) -- (logloss)
            node [very near start, auto=right] {}
            node [very near end, auto=right] {};
        \path [draw, <-, >=open triangle 45] (comms) -- (ethernet)
            node [very near start, auto=right] {}
            node [very near end, auto=right] {};
    \end{tikzpicture}
    \caption{Class diagram for communication engines}
    \label{fig-interface}
\end{figure}

\begin{figure}
    \centering
    \scalebox{0.8}{
    \begin{tikzpicture}
        \eumlclass[minimum width=6cm, text width=6cm]{event}{Event \textit{$<<$Abstract$>>$}}
        {
            \umlmethod{+}{\_\_init\_\_}{}{void}
            \umlmethod{+}{\underline{from\_pickle}}{pickled : String}{Event}
            \umlmethod{+}{pickle}{}{String}
        }
        \umlclass[minimum width=6cm, text width=6cm, xshift=-7cm, below of=event]{send}{CommunicationSendEvent}{
            \umlattr{\#}{\_src\_agent\_uid}{int}
            \umlattr{\#}{\_src\_iface\_name}{String}
            \umlattr{\#}{\_message}{Message}
            \umlattr{\#}{\_network}{String}
        }
        {
            \umlmethod{+}{\_\_init\_\_}{src\_agent\_uid : int, src\_iface\_name : String, message\_inst : Message, network : String}{void}
            \umlmethod{+}{get\_src\_agent\_uid}{}{int}
            \umlmethod{+}{get\_src\_iface\_name}{}{String}
            \umlmethod{+}{get\_message}{}{Message}
            \umlmethod{+}{get\_network}{}{String}
        }
        \umlclass[minimum width=6cm, text width=6cm, xshift=7cm, below of=event]{recv}{CommunicationRecvEvent}{
            \umlattr{\#}{\_src\_agent\_uid}{int}
            \umlattr{\#}{\_src\_iface\_name}{String}
            \umlattr{\#}{\_message}{Message}
            \umlattr{\#}{\_network}{String}
        }
        {
            \umlmethod{+}{\_\_init\_\_}{src\_agent\_uid : int, src\_iface\_name : String, message\_inst : Message, network : String}{void}
            \umlmethod{+}{get\_src\_agent\_uid}{}{int}
            \umlmethod{+}{get\_src\_iface\_name}{}{String}
            \umlmethod{+}{get\_message}{}{Message}
            \umlmethod{+}{get\_network}{}{String}
        }
        \umlclass[minimum width=6cm, text width=6cm, xshift=7cm, above of=event]{link}{LinkEvent}{
            \umlattr{\#}{\_up}{bool}
            \umlattr{\#}{\_uid\_1}{int}
            \umlattr{\#}{\_uid\_2}{int}
            \umlattr{\#}{\_network\_name}{String}
            \umlattr{\#}{\_pathloss}{float}
        }
        {
            \umlmethod{+}{\_\_init\_\_}{up : bool, uid\_1 : int, uid\_2 : int, network\_name : String, pathloss : float}{void}
            \umlmethod{+}{get\_up}{}{bool}
            \umlmethod{+}{get\_uid\_1}{}{int}
            \umlmethod{+}{get\_uid\_2}{}{int}
            \umlmethod{+}{get\_network\_name}{}{String}
            \umlmethod{+}{get\_pathloss}{}{float}
        }
        \umlclass[minimum width=6cm, text width=6cm, xshift=-7cm, above of=event]{move}{MoveEvent}{
            \umlattr{\#}{\_entity\_uid}{int}
            \umlattr{\#}{\_lat}{float}
            \umlattr{\#}{\_lon}{float}
            \umlattr{\#}{\_agl}{float}
            \umlattr{\#}{\_velocity}{float}
        }
        {
            \umlmethod{+}{\_\_init\_\_}{entity : Entity, lat : float, lon : float, agl : float, vel : float}{void}
            \umlmethod{+}{get\_uid}{}{int}
            \umlmethod{+}{get\_lat}{}{float}
            \umlmethod{+}{get\_lon}{}{float}
            \umlmethod{+}{get\_agl}{}{float}
            \umlmethod{+}{get\_velocity}{}{float}
        }
        \umlclass[minimum width=6cm, text width=6cm, xshift=2cm, right of=event]{start}{StartupEvent}{
            \umlattr{\#}{\_world}{World}
        }
        {
            \umlmethod{+}{\_\_init\_\_}{world : World}{void}
            \umlmethod{+}{get\_world}{}{World}
        }
        \umlclass[minimum width=6cm, text width=6cm, below of=event]{ackstart}{AckStartupEvent}{
            \umlattr{\#}{\_daemon\_id}{int}
        }
        {
            \umlmethod{+}{\_\_init\_\_}{daemon\_id : int}{void}
            \umlmethod{+}{get\_daemon\_id}{}{int}
        }
        \umlclass[minimum width=6cm, text width=6cm, xshift=-2cm, left of=event]{startsim}{StartSimulationEvent}{
            \umlattr{\#}{\_entity\_mapping}{dict(int : int)}
        }
        {
            \umlmethod{+}{\_\_init\_\_}{daemon\_id : int}{void}
            \umlmethod{+}{get\_mapping}{}{dict(int : int)}
        }
        \eumlclass[minimum width=6cm, text width=6cm, above of=event]{stop}{StopSimulationEvent}
        {
            \umlmethod{+}{\_\_init\_\_}{}{void}
        }

        \path [draw, <-, >=open triangle 45] (event) -- (send);
        \path [draw, <-, >=open triangle 45] (event) -- (recv);
        \path [draw, <-, >=open triangle 45] (event) -- (link);
        \path [draw, <-, >=open triangle 45] (event) -- (move);
        \path [draw, <-, >=open triangle 45] (event) -- (start);
        \path [draw, <-, >=open triangle 45] (event) -- (ackstart);
        \path [draw, <-, >=open triangle 45] (event) -- (startsim);
        \path [draw, <-, >=open triangle 45] (event) -- (stop);
    \end{tikzpicture}}
    \caption{Class diagram for events}
    \label{fig-events}
\end{figure}

\begin{figure}
    \centering
    \begin{tikzpicture}
        \umlclass[xshift=-2cm]{action}{Agent \textit{$<<$Abstract$>>$}}{
            \umlattr{\#}{\_owner\_node}{Node}
            \umlattr{\#}{\_uid}{int}
            \umlattr{\#}{\_behaviors}{dict(Event : list([precondition, action]))}
        }{
            \umlmethod{+}{\_\_init\_\_}{owner\_uid, uid}{void}
            \umlmethod{\#}{\_on\_event}{event : Event}{void}
            \umlmethod{\#}{\_init\_behavior}{void}{void}
            \umlmethod{+}{get\_uid}{}{int}
            \umlmethod{+}{get\_owner\_node}{}{Node}
            \umlmethod{+}{on\_message}{event : Event}{void}
            \umlmethod{+}{start}{}{void}
            \umlmethod{+}{add\_behavior}{behavior : (precond, event, action)}{void}
            \umlmethod{+}{remove\_behavior}{behavior : (precond, event, action)}{void}
            \umlmethod{+}{\textit{run}}{}{void}
        }
    \end{tikzpicture}
    \caption{Class diagram for agents}
    \label{fig-agents}
\end{figure}
\begin{figure}
    \centering
    \begin{tikzpicture}
        \eumlclass[xshift=-2cm]{condition}{Condition \textit{$<<$Abstract$>>$}}
        {
            \umlmethod{+}{\_\_init\_\_}{}{void}
            \umlmethod{+}{\textit{is\_met}}{event : Event}{bool}
        }
    \end{tikzpicture}
    \caption{Class diagram for conditions}
    \label{fig-condition}
\end{figure}

\begin{figure}
    \centering
    \begin{tikzpicture}
        \eumlclass[]{action}{Action \textit{$<<$Abstract$>>$}}
        {
            \umlmethod{+}{\_\_init\_\_}{}{void}
            \umlmethod{+}{\textit{perform}}{}{void}
        }
        \umlclass[below of=action]{move}{Move}{
            \umlattr{\#}{\_entity}{int}
            \umlattr{\#}{\_lat}{float}
            \umlattr{\#}{\_lon}{float}
            \umlattr{\#}{\_agl}{float}
        }
        {
            \umlmethod{+}{\_\_init\_\_}{entity : int, lat : float, lon : float, agl : float}{void}
            \umlmethod{+}{perform}{}{void}
        }
        \path [draw, <-, >=open triangle 45] (action) -- (move)
            node [very near start, auto=right] {}
            node [very near end, auto=right] {};
    \end{tikzpicture}
    \caption{Class diagram for actions}
    \label{fig-actions}
\end{figure}

\begin{figure}
    \centering
    \begin{tikzpicture}
        \umlclass[]{sensor}{Sensor \textit{$<<$Abstract$>>$}}{
            \umlattr{\#}{\_subscribers}{list(Entity)}
        }
        {
            \umlmethod{+}{\_\_init\_\_}{}{void}
            \umlmethod{\#}{\_publish\_data}{event : Event}{void}
            \umlmethod{+}{subscribe}{callback : function}{void}
            \umlmethod{+}{\textit{run}}{}{void}
        }
        \umlclass[below of=sensor]{radar}{RadarSensor}{
            \umlattr{\#}{\_trans\_power}{float}
            \umlattr{\#}{\_trans\_freq}{float}
            \umlattr{\#}{\_gain}{float}
            \umlattr{\#}{\_aperture}{float}
            \umlattr{\#}{\_prop\_fac}{float}
            \umlattr{\#}{\_dwell\_time}{float}
            \umlattr{\#}{\_angle}{float}
        }
        {
            \umlmethod{+}{\_\_init\_\_}{}{void}
            \umlmethod{\#}{\_publish\_data}{event : Event}{void}
            \umlmethod{+}{subscribe}{callback : function}{void}
            \umlmethod{+}{run}{}{void}
        }

        \path [draw, <-, >=open triangle 45] (sensor) -- (radar)
            node [very near start, auto=right] {}
            node [very near end, auto=right] {};
    \end{tikzpicture}
    \caption{Class diagram for sensors}
    \label{fig-sensors}
\end{figure}
\begin{figure}
    \centering
    \begin{tikzpicture}
        \eumlclass[xshift=-2cm]{geo}{Geo}
        {
            \umlmethod{+}{\underline{haver\_distance}}{lat1 : float, lon1 : float, lat2 : float, lon2 : float}{float}
            \umlmethod{+}{\underline{linear\_distance}}{lat1 : float, lon1 : float, lat2 : float, lon2 : float}{float}
            \umlmethod{+}{\underline{sph\_to\_lin}}{lat : float, lon : float, agl : float}{(float, float, float)}
            \umlmethod{+}{\underline{lin\_to\_sph}}{x : float, y : float, z : float}{(float, float, float)}
            \umlmethod{+}{\underline{linear\_to\_degree}}{lat : float, lon : float, lat\_km : float, lon\_km : float}{float}
            \umlmethod{+}{\underline{loc\_from\_bearing\_dist}}{lat : float, lon : float, bearing : float, dist : float}{float}
        }
    \end{tikzpicture}
    \caption{Class diagram for geographic utilities}
    \label{fig-condition}
\end{figure}

\end{document}
