\RequirePackage{fix-cm}
\documentclass[titlepage]{article}

\usepackage{fullpage}    % Use the whole page
\usepackage{fancyhdr}    % Nice headers/footers
\usepackage{mdwlist}     % For itemize* and enumerate*
\usepackage{graphicx}    % Importing graphics
\usepackage{hyperref}    % Hyperlink references and URLs
\usepackage[figure,table]{hypcap} % Hyperlink points to the top of figures
\usepackage[usenames,dvipsnames]{xcolor}	% Logo
\usepackage{tikz,ifthen}			% Logo
\usepackage{pgf}				% Logo
\usepackage{scalefnt}				% Logo
\usepgfmodule{shapes}				% Logo
\usepgfmodule{plot}				% Logo

% Tikz packages
%\usepackage[svgnames]{xcolor}
\usepackage[utf8]{inputenc}
%\usepackage{tikz}
%BEGIN IMAGE
\usetikzlibrary{shapes,arrows,shadows,fit}
%END IMAGE
\usepackage{pgf-umlsd}

% Just so we don't have to specify this twice
\newcommand\mytitle{Software Requirements Specification}
\newcommand\mydate{\today}
\newcommand\myversion{1}

\hypersetup{
    colorlinks=true,
    linkcolor=blue,
    urlcolor=blue,
    pdftitle={AHOY Software Requirements Specification V6},
    pdfauthor={Dustin Ingram, Aaron Rosenfeld, Maria Kolakowska, Frank Clark}
}

% So we can number paragraphs too
\setcounter{secnumdepth}{5}

% For headers and footers
\setlength{\headheight}{15pt}
\setlength{\headsep}{25pt}
\pagestyle{fancy}
	
% Page style for the title page
\fancypagestyle{plain}{
    \fancyhf{}
    \renewcommand{\headrulewidth}{0pt}
    \renewcommand{\footrulewidth}{0pt}
}

% Page style for every other page
\fancyhf{} % clear all header and footer fields
\fancyhead[L]{AHOY}
\fancyhead[C]{\mytitle}
\fancyhead[R]{\mydate}
\fancyfoot[C]{\thepage}
\renewcommand{\headrulewidth}{0.4pt}
\renewcommand{\footrulewidth}{0.4pt}

\title{\textbf{\mytitle}}
\author{
	Frank Clark \\\url{francis.j.clark@drexel.edu}
    \and Dustin Ingram \\\url{dustin.s.ingram@drexel.edu}
	\and Maria Kolakowska \\\url{maria.j.kolakowska@drexel.edu}
    \and Aaron Rosenfeld \\\url{aaron.rosenfeld@drexel.edu}
}
\date{\mydate\\Version \myversion}

%%% Tikz stuff %%%
% Style for UML Class with attributes and methods
%BEGIN IMAGE
\tikzstyle{umlclass} = [
    rectangle, rectangle split, rectangle split parts=3,
    % This makes a nice gradient
    top color=white, bottom color=blue!30, draw=blue!50!black!100,
    drop shadow, rounded corners,
    node distance = 5cm, text width = 5cm]

% Line styles
\tikzstyle{hasa} = [draw, ->, >=open diamond]
\tikzstyle{ownsa} = [draw, ->, >=diamond]
\tikzstyle{isa} = [draw, ->, >=open triangle 45]

\newcommand{\umlclass}[5][,]{
    \node [umlclass,#1] (#2) {
        \textbf{#3}
        \nodepart{second}
        \begin{description*}
        #4
        \end{description*}
        \nodepart{third}
        \begin{description*}
        #5
        \end{description*}
    };
}

\newcommand{\umlattr}[3]{
    \item[$#1$]\textbf{#2}: \textit{#3}
}
\newcommand{\umlmethod}[4]{
    \item[$#1$]\textbf{#2}({#3}): \textit{#4}
}
\newcommand{\umlarg}[2]{\textit{#1} #2}

\newcommand{\umlrelation}[6][-|]{
    \path [#3] (#2) #1 (#4)
        node [very near start, auto=right] {#5}
        node [very near end, auto=right] {#6};
}
%END IMAGE
%%% End tikz stuff %%%

\begin{document}
\pagenumbering{roman}

\begin{figure*}
   % \vspace{-2em}
    \centering
    \scalebox{0.8}{
\begin{tikzpicture}[scale=1]
	
	\pgfsetlinewidth{3pt}

	% Background
	\color{cyan!70!black}
	\pgfpathmoveto{\pgfpointxy{-5}{2}}
	\pgfpathlineto{\pgfpointxy{-5}{11}}
	\pgfpathlineto{\pgfpointxy{-2}{11.9}}	
	\pgfpathlineto{\pgfpointxy{2}{11.9}}	
	\pgfpathlineto{\pgfpointxy{5}{11}}
	\pgfpathlineto{\pgfpointxy{5}{2}}
	\pgfpathclose 
	\pgfusepath{fill,stroke} 

	% Base
	\color{green!70!black}
	\pgfsetstrokecolor{black}
	\pgfpathmoveto{\pgfpointxy{-2}{1.5}}
	\pgfpathcurveto{\pgfpointxy{-2}{1.5}}{\pgfpointxy{-6}{1.5}}{\pgfpointxy{-6}{2.5}}
	\pgfpathlineto{\pgfpointxy{-6}{4}}
	\pgfpathlineto{\pgfpointxy{6}{4}}
	\pgfpathlineto{\pgfpointxy{6}{2.5}}
	\pgfpathcurveto{\pgfpointxy{6}{1.5}}{\pgfpointxy{2}{1.5}}{\pgfpointxy{2}{1.5}}
	\pgfpathclose 
	\pgfusepath{fill,stroke} 

	% Curtains
	\color{red!70!black}
	\pgfsetstrokecolor{black}

	% Left
	\pgfpathmoveto{\pgfpointxy{-6}{11}}
	\pgfpathlineto{\pgfpointxy{-6}{2.5}}
	\pgfpathcurveto{\pgfpointxy{-6}{2.2}}{\pgfpointxy{-3.5}{2.2}}{\pgfpointxy{-3.5}{2.5}}
	\pgfpathcurveto{\pgfpointxy{-3.5}{3}}{\pgfpointxy{-3.5}{4}}{\pgfpointxy{-4.5}{5}}
	\pgfpathcurveto{\pgfpointxy{-2.5}{7}}{\pgfpointxy{-4}{11}}{\pgfpointxy{-3}{11.5}}
	\pgfpathcurveto{\pgfpointxy{-4}{11}}{\pgfpointxy{-2.5}{7}}{\pgfpointxy{-4.5}{5}}
	\pgfpathcurveto{\pgfpointxy{-2.5}{7}}{\pgfpointxy{-6}{11}}{\pgfpointxy{-3}{11.5}}
	\pgfpathcurveto{\pgfpointxy{-6}{11}}{\pgfpointxy{-2.5}{7}}{\pgfpointxy{-4.5}{5}}
	\pgfpathcurveto{\pgfpointxy{-2.5}{7}}{\pgfpointxy{-8}{11}}{\pgfpointxy{-3}{11.5}}
	\pgfpathcurveto{\pgfpointxy{-8}{11}}{\pgfpointxy{-2.5}{7}}{\pgfpointxy{-4.5}{5}}
	\pgfpathcurveto{\pgfpointxy{-2.5}{7}}{\pgfpointxy{-2.5}{11}}{\pgfpointxy{-3}{11.5}}
	\pgfusepath{fill,stroke}

	% Right
	\pgfsetlinewidth{3pt}
	\pgfpathmoveto{\pgfpointxy{6}{11}}
	\pgfpathlineto{\pgfpointxy{6}{2.5}}
	\pgfpathcurveto{\pgfpointxy{6}{2.2}}{\pgfpointxy{3.5}{2.2}}{\pgfpointxy{3.5}{2.5}}
	\pgfpathcurveto{\pgfpointxy{3.5}{3}}{\pgfpointxy{3.5}{4}}{\pgfpointxy{4.5}{5}}
	\pgfpathcurveto{\pgfpointxy{2.5}{7}}{\pgfpointxy{4}{11}}{\pgfpointxy{3}{11.5}}
	\pgfpathcurveto{\pgfpointxy{4}{11}}{\pgfpointxy{2.5}{7}}{\pgfpointxy{4.5}{5}}
	\pgfpathcurveto{\pgfpointxy{2.5}{7}}{\pgfpointxy{6}{11}}{\pgfpointxy{3}{11.5}}
	\pgfpathcurveto{\pgfpointxy{6}{11}}{\pgfpointxy{2.5}{7}}{\pgfpointxy{4.5}{5}}
	\pgfpathcurveto{\pgfpointxy{2.5}{7}}{\pgfpointxy{8}{11}}{\pgfpointxy{3}{11.5}}
	\pgfpathcurveto{\pgfpointxy{8}{11}}{\pgfpointxy{2.5}{7}}{\pgfpointxy{4.5}{5}}
	\pgfpathcurveto{\pgfpointxy{2.5}{7}}{\pgfpointxy{2.5}{11}}{\pgfpointxy{3}{11.5}}
	\pgfusepath{fill,stroke}

	% Top
	%     Top-left
	\pgfpathmoveto{\pgfpointxy{-2}{12}}
	\pgfpathcurveto{\pgfpointxy{-2}{12}}{\pgfpointxy{-6}{12}}{\pgfpointxy{-6}{11}}
	\pgfpathcurveto{\pgfpointxy{-5}{9}}{\pgfpointxy{-2}{11}}{\pgfpointxy{-2}{11.85}}
	\pgfpathcurveto{\pgfpointxy{-2}{11.5}}{\pgfpointxy{-4.5}{9.5}}{\pgfpointxy{-6}{11}}
	\pgfpathcurveto{\pgfpointxy{-4.5}{9.5}}{\pgfpointxy{-2}{11.5}}{\pgfpointxy{-2}{11.85}}
	\pgfpathcurveto{\pgfpointxy{-2}{12}}{\pgfpointxy{-3.5}{10.4}}{\pgfpointxy{-6}{11}}
	\pgfpathcurveto{\pgfpointxy{-3.5}{10.4}}{\pgfpointxy{-2}{12}}{\pgfpointxy{-2}{11.85}}

	%    Top-middle
	\pgfpathcurveto{\pgfpointxy{-1}{10.5}}{\pgfpointxy{1}{10.5}}{\pgfpointxy{2}{11.85}}
	\pgfpathcurveto{\pgfpointxy{1}{10.5}}{\pgfpointxy{-1}{10.5}}{\pgfpointxy{-2}{11.85}}	
	\pgfpathcurveto{\pgfpointxy{-1}{11}}{\pgfpointxy{1}{11}}{\pgfpointxy{2}{11.85}}
	\pgfpathcurveto{\pgfpointxy{1}{11}}{\pgfpointxy{-1}{11}}{\pgfpointxy{-2}{11.85}}	
	\pgfpathcurveto{\pgfpointxy{-1}{10}}{\pgfpointxy{1}{10}}{\pgfpointxy{2}{11.85}}

	%    Top-right
	\pgfpathcurveto{\pgfpointxy{2}{11.5}}{\pgfpointxy{4.5}{9.5}}{\pgfpointxy{6}{11}}
	\pgfpathcurveto{\pgfpointxy{4.5}{9.5}}{\pgfpointxy{2}{11.5}}{\pgfpointxy{2}{11.85}}
	\pgfpathcurveto{\pgfpointxy{2}{12}}{\pgfpointxy{3.5}{10.4}}{\pgfpointxy{6}{11}}
	\pgfpathcurveto{\pgfpointxy{3.5}{10.4}}{\pgfpointxy{2}{12}}{\pgfpointxy{2}{11.85}}
	\pgfpathcurveto{\pgfpointxy{2}{11}}{\pgfpointxy{5}{9}}{\pgfpointxy{6}{11}}
	\pgfpathcurveto{\pgfpointxy{6}{12}}{\pgfpointxy{2}{12}}{\pgfpointxy{2}{12}}
	\pgfpathclose 
	\pgfusepath{fill,stroke} 

	% Rope
	%     Rope-right
	\pgfsetstrokecolor{black}
	\pgfpathmoveto{\pgfpointxy{-4.5}{5}}
	\pgfpathcurveto{\pgfpointxy{-4.5}{5}}{\pgfpointxy{-6}{5}}{\pgfpointxy{-6}{5.5}}
	\pgfusepath{stroke}	
	%     Rope-left
	\pgfsetstrokecolor{black}
	\pgfpathmoveto{\pgfpointxy{4.5}{5}}
	\pgfpathcurveto{\pgfpointxy{4.5}{5}}{\pgfpointxy{6}{5}}{\pgfpointxy{6}{5.5}}
	\pgfusepath{stroke}

	\node[color=black] at (0,0) {{\scalefont{10.0}STAGE}};

	%% Just kinda a pretty path...
	%% \pgfpathmoveto{\pgfpointxy{-2}{1.5}}
	%% \pgfpathcurveto{\pgfpointxy{-2}{1.5}}{\pgfpointxy{-6}{1.5}}{\pgfpointxy{-6}{2.5}}
	%% \pgfpathcurveto{\pgfpointxy{-5}{4.5}}{\pgfpointxy{-2}{2.5}}{\pgfpointxy{-2}{3.35}}
	%% \pgfpathcurveto{\pgfpointxy{-1}{3.5}}{\pgfpointxy{1}{3.5}}{\pgfpointxy{2}{3.35}}
	%% \pgfpathcurveto{\pgfpointxy{2}{2.5}}{\pgfpointxy{5}{4.5}}{\pgfpointxy{6}{2.5}}
	%% \pgfpathcurveto{\pgfpointxy{6}{1.5}}{\pgfpointxy{2}{1.5}}{\pgfpointxy{2}{1.5}}
	%% \pgfpathclose 
\end{tikzpicture}
}
    \vspace{-4em}
\end{figure*}

\maketitle

\begin{abstract}
AHOY is an event-based simulation environment used to compare the effectiveness of different combinations of software agents, network configurations, and sensor data in real-world environments.  It is comprised of a distributed simulation engine, visualizer, and programming interface through which developers create agent software and network topologies.  Communication between virtual nodes is also simulated, providing highly realistic scenarios.
\end{abstract}

\setcounter{tocdepth}{4}
\tableofcontents
\pagebreak
\listoffigures
\pagebreak
\pagenumbering{arabic}

%------------------------------------------------------------------------------

%
% What this document should do:
%
% The SDD shows how the software system will be structured to satisfy the
% requirements identified in the software requirements specification IEEE Std
% 830-1998. It is a translation of requirements into a description of the
% software structure, software components, interfaces, and data necessary for the
% implementation phase. In essence, the SDD becomes a detailed blueprint for the
% implementation activity. In a complete SDD, each requirement must be traceable
% to one or more design entities.
%

%------------------------------------------------------------------------------

\section{Introduction
    \label{introduction}
}
%     6.1 Introduction 
%         Each design description user may have a different view of what are
%         considered the essential aspects of a software design. All other
%         information is extraneous to that user. The proportion of useful
%         information for a specific user will decrease with the size and
%         complexity of a software project. The needed information then becomes
%         difficult or impractical to extract from the description and impossible
%         to assimilate. Hence, a practical organization of the necessary design
%         information is essential to its use. 
% 
%         This clause introduces the notion of design views to aid in organizing
%         the design attribute information defined in Clause 5. It does not
%         supplement Clause 5 by providing additional design information nor does
%         it prescribe the format or documentation practice for design views. 
% 
%         A recommended organization of design entities and their associated
%         attributes are presented in this clause to facilitate the access of
%         design information from various technical viewpoints. This recommended
%         organization is flexible and can be implemented through different media
%         such as paper documentation, design languages, or database management
%         systems with automated report generation, and query language access. A
%         sample table of contents is given in Annex A to illustrate how an
%         access structure to a design description may be prepared. 
% 

%------------------------------------------------------------------------------

\subsection{Purpose
    \label{purpose}
}

This design document defines the architecture of the ServiceSniffer program.
ServiceSniffer is built modularly with the major chunks being data capture,
data filtering and processing, central system kernel, and user interfaces.  The
information presented here is intended for the development team and the
external stakeholders, which are currently Dr. William Regli, Dr.  Dara Kusic,
and Raytheon Company.

%------------------------------------------------------------------------------

\subsection{Scope
    \label{scope}
}

The goal of the ServiceSniffer project is to automate the passive discovery,
recognition, and consumption of semantic web services.  ServiceSniffer does not
require any central registry of services (and, in fact, is designed to be a
substitute for one) but will examine network traffic patterns to identify
specific services, how to invoke them, and report specific, user chosen
information.  As well as running with known service definitions, the system
should be able to identify new services and analyze relevant network traffic to
determine how to consume the service.  This application will be constructed of
an extensible core framework upon which future modules can be developed and
integrated.

ServiceSniffer is split into 4 general pieces: the data input, data processing,
system kernel, and user interface modules.  The data input module is
responsible for collecting traffic off the network or using a Pcap file as a
source instead of a live network.  The data processing module, which is
constructed from various filters and processors, applies filters to the network
traffic from the data input module and converts it into various and
configurable reports and views depending on what the user wants to see.  Some
examples of these reports will be described later.  The system kernel acts as a
central broker for all other modules to talk to each other, as well as being
responsible for the creation/deletion of filter and processor chains.  Lastly,
ServiceSniffer has both a CLI and a GUI for viewing the various data returned
by the data processors and filters.  These interfaces will allow users to
interact with the data returned.  For example, a user could apply additional
display filters, such as filtering by a specific destination IP.  Both
interfaces will offer to users a report of what web services appear to be
available on the network and supply a report - an ad-hoc registry of sorts -
which users can use to consume the web services.  Users may specify additional
types of filters or processors to generate different types of reports via an
extension system.

%------------------------------------------------------------------------------

\subsection{Definitions, Acronyms, and Abbreviations%
  \label{definitions}%
}

\begin{description}
\item[Consumption of a Service]
    To become a user/client of the web service

\item[Filter]
    A tool that returns a subset of its input, based on some parameters

\item[JSON]
    JavaScript Object Notation

\item[libpcap]
    Multiplatform packet capture library

\item[PCAP]
    Packet CAPture - an API for capturing and processing packets from live
    network devices, and files

\item[Processor]
    A tool that transforms its input from one data type to another

\item[Semantic Service]
    A web service which defines a context for the meaning of the data which it
    returns

\item[Web Service]
    A software system designed to support interoperable machine-to-machine
    interaction over a network

\item[WSDL]
    Web Services Description Language - a format for describing web services
\end{description}

%------------------------------------------------------------------------------

\subsection{References%
  \label{references}%
}

These documents explain in-depth about the various technologies involved in
this project.
%
\begin{itemize*}
\item WSDL Specification: \url{http://www.w3.org/TR/wsdl20/}
\item SOAP Specification: \url{http://www.w3.org/TR/soap/}
\item libpcap Homepage: \url{http://www.tcpdump.org/}
\item MVP-E Project (Team 10): \url{http://www.cs.drexel.edu/SeniorDesign/2010Material/Projects2010.html}
\end{itemize*}

%------------------------------------------------------------------------------

\section{Design Views
    \label{design-views}
}
%     6.2 Design views 
%         Entity attribute information can be organized in several ways to reveal
%         all of the essential aspects of a design. In so doing, the user is able
%         to focus on design details from a different perspective or viewpoint. A
%         design view is a subset of design entity attribute information that is
%         specifically suited to the needs of a software project activity. 
% 
%         Each design view represents a separate concern about a software system.
%         Together, these views provide a comprehensive description of the design
%         in a concise and usable form that simplifies information access and
%         assimilation. 
% 
%         A recommended organization of the SDD into separate design views to
%         facilitate information access and assimilation is given in Table 1.
%         Each of these views, their use, and representation are discussed in
%         detail. 
% 
%         (Table 1 - "Recommended design views" here)
% 
%         6.2.1 Decomposition description 
%             6.2.1.1 Scope 
%                 The decomposition description records the division of the
%                 software system into design entities. It describes the way the
%                 system has been structured and the purpose and function of each
%                 entity. For each entity, it provides a reference to the
%                 detailed description via the identification attribute. 
% 
%                 The attribute descriptions for identification, type, purpose,
%                 function, and subordinates should be included in this design
%                 view. This attribute information should be provided for all
%                 design entities. 
% 
%             6.2.1.2 Use 
%                 The decomposition description can be used by designers and
%                 maintainers to identify the major design entities of the system
%                 for purposes such as determining which entity is responsible
%                 for performing specific functions and tracing requirements to
%                 design entities. Design entities can be grouped into major
%                 classes to assist in locating a particular type of information
%                 and to assist in reviewing the decomposition for completeness.
%                 For example, a module decomposition may exist separately from a
%                 data decomposition. 
% 
%                 The information in the decomposition description can be used by
%                 project management for planning, monitoring, and control of a
%                 software project. They can identify each software component,
%                 its purpose, and basic functionality. This design information
%                 together with other project information can be used in
%                 estimating cost, staff, and schedule for the development
%                 effort. 
% 
%                 Configuration management may use the information to establish
%                 the organization, tracking, and change management of emerging
%                 work products; see IEEE Std 828-1998. Metrics developers may
%                 also use this information for initial complexity, sizing,
%                 staffing, and development time parameters. The software quality
%                 assurance staff can use the decomposition description to
%                 construct a requirements traceability matrix. 
% 

%------------------------------------------------------------------------------

ServiceSniffer will be built modularly.  The modules being used are data
capture, filter, kernel, and user interface.  Each module communicates to other
modules using a defined interface.  Using a modular approach allows for
individual modules to be replaced or extended as the user requires without
altering other modules in the system, as each task is logically distinct from
the others

\subsection{Data Capture}

The data capture module (defined in Section \ref{capture_classes}) handles the
collection, storage, and reuse of network traffic.  Using libpcap, network
traffic is collected from the network device and a first analysis is done to
remove all non-service related traffic.  The remaining data will then be
optionally stored and passed on to the kernel module, which will in turn hand
it off to the filter/processor module.  If the data is also stored, it is saved
in a PCap file in a locally accessible location specified by the user.  Lastly,
the user can load a previously saved PCap file for analysis instead of
collecting and analyzing live data.

\subsection{Filters \& Processors}

The filter and processor module (defined in Section \ref{processor_classes})
receives the PCap-formatted packets from the data capture module and turns them
into something useful to users.  While filters and processors are the same in
terms of class interfaces, there is a logical difference between the two.
Filters strictly reduce the number of items from their input returning a subset
of its input based on some criteria and perform no translation of them into
other forms of data (e.g.  filter out all non-SOAP packets), while processors
analyze the input they receive to transform them into different forms (e.g.
determine and report patterns in invocations of services).  Filters and
processors can be linked together to form processing chains as the user
requires.

\subsection{Kernel}

The kernel module (defined in Section \ref{kernel_classes}) acts as the central
communication bus for all modules to interoperate.  It's also responsible for
the handling of the filter module's processes and their inter-process
communication (IPC).  As there will likely be a high volume of traffic, the
data sent multiple times needs to be reduced to a minimum.  Using a central
kernel and bus architecture makes this happen.

\subsection{Interface}

The last piece of the system is the user interface (UI) module (defined in
Section \ref{interfaces}).  The UI module is, as expected, the part of the
system that will be used most frequently by users.  There are two interfaces
that will be developed for the first release: a command-line interface (CLI)
and a graphical user interface (GUI).  The CLI will allow ServiceSniffer to be
scripted, which adds functionality such as exporting the UDDI registry for
systematic importing into another system.  The GUI allows a user to interact
with the data being collected and analyzed, as well as control how filters and
processors are used, and let the user run and export reports.

%------------------------------------------------------------------------------

\section{Architecture
    \label{architecture}
}

ServiceSniffer adopts a process-based message passing architecture, with each
module operating in its own operating system process.  This architecture allows
us to easily take advantage of multi-core or multi-processor systems, without
the synchronization issues involved with threading.  On UNIX based systems the
cost of starting a process is low, so this is not an issue on modern machines.

The various processes in our system communicate to each other using message
passing over UNIX domain sockets.  UNIX domain sockets provide a fast and
fairly high level way of sending data to different processes.

This section defines the interfaces each module uses, and the APIs used to
communicate with all of the modules in our system.

%------------------------------------------------------------------------------

\subsection{Communication Classes
    \label{communication_classes}
}

\begin{figure}
    \centering
%BEGIN IMAGE
    \begin{tikzpicture}
        \umlclass[text width = 6cm,xshift=-2cm]{socketserver}{SocketServer}{
            \umlattr{\#}{name}{String}
            \umlattr{\#}{callback}{Callable(SocketMessage)}
            \umlattr{-}{sockets}{list(socket)}
        }{
            \umlmethod{+}{\_\_init\_\_}{
                \umlarg{String}{input\_socket\_path}
                \umlarg{Callable(SocketMessage)}{receive\_callback}
            }{None}
            \umlmethod{\#}{send\_message}{
                \umlarg{SocketMessage}{message}
            }{void}
        }
        \umlclass[text width = 6cm,right of=socketserver,xshift=2cm]{socketclient}{SocketClient}{
            \umlattr{\#}{name}{String}
            \umlattr{\#}{callback}{Callable(SocketMessage)}
            \umlattr{-}{socket}{socket}
        }{
            \umlmethod{+}{\_\_init\_\_}{
                \umlarg{String}{input\_socket\_path}
                \umlarg{Callable(SocketMessage)}{receive\_callback}
            }{None}
            \umlmethod{\#}{send\_message}{
                \umlarg{SocketMessage}{message}
            }{void}
        }
        \umlclass[below right of=socketserver]{socketmessage}{SocketMessage}{
            \umlattr{\#}{data}{Dictionary}
        }{
            \umlmethod{+}{\_\_init\_\_}{
                    **kwargs        
            }{void}
            \umlmethod{+}{\_\_str\_\_}{}{String}
        }
        
        \umlrelation[|-]{socketserver}{hasa}{socketmessage}{}{}
        \umlrelation[|-]{socketclient}{hasa}{socketmessage}{}{}
    \end{tikzpicture}
%END IMAGE
%HEVEA\imageflush
    \caption{Class diagram for common classes across multiple modules}
    \label{fig-commonclasses}
\end{figure}

The communication classes are the base for the interprocess communication
system.  Figure~\ref{fig-commonclasses} shows the main classes that deal with
communication, the \nameref{class-socket-server},
\nameref{class-socket-client}, and \nameref{class-socket-message} classes.
These classes will be reused for communication purposes between the other
modules.

%------------------------------------------------------------------------------

\subsubsection{SocketServer
    \label{class-socket-server}
}

The SocketServer is used by a producer of data to facilitate message passing to
one or many consumers.  The SocketServer handles the creation of the UNIX
domain socket, and listening for connections on that socket.  When the owner of
the SocketServer wants to send a message (a \nameref{class-socket-client}), the
instance handles distributing the message to everyone connected to it.  Also,
it provides a callback-based method for receiving messages sent to it by a
client.

\paragraph{Attributes}
\begin{itemize}
    \item \textbf{name}: A string of the path to the socket file to create.
        This must be a path the process running this has access to and is not
        already taken by another socket.
    \item \textbf{callback}: A callable object to be called when the server
        receives a message from a client.  If None, messages from clients are
        silently ignored.
    \item \textbf{sockets}: A list of socket objects connected to all clients
        currently listening for messages from this server.
\end{itemize}

\paragraph{Methods}
\begin{itemize}
    \item \textbf{\_\_init\_\_}: The constructor to this class.  A mandatory
        string path is required to create the initial socket.  An optional
        callback Callable can be given to be notified of message reception.
    \item \textbf{send\_message}: This is what the owner of the SocketServer
        uses to send a message.  Given a \nameref{class-socket-message} (or a
        subclass), the message is serialized and sent over the socket to all
        listening clients.
\end{itemize}

%------------------------------------------------------------------------------

\subsubsection{SocketClient
    \label{class-socket-client}
}

The SocketClient is used by consumers of data to facilitate connecting to a
\nameref{class-socket-server} and sending and receiving messages from that
producer of data.  It allows for a callback-based method of incoming message
notification.

\paragraph{Attributes}
\begin{itemize}
    \item \textbf{name}: A string of the path to the socket file to connect to.  This
        must be a path that the process running this has access to, and that is
        connected to a server.
    \item \textbf{callback}: A callable object to be called when the client receives a
        message from a server.
    \item \textbf{socket}: The socket connected to the server.
\end{itemize}

\paragraph{Methods}
\begin{itemize}
    \item \textbf{\_\_init\_\_}: The constructor to this class.  A mandatory
        string path is required to connect to the server socket.  A mandatory
        callback Callable can be given to be notified of message reception.
    \item \textbf{send\_message}: This is what the owner of the SocketClient
        uses to send a message.  Given a \nameref{class-socket-message} (or a
        subclass), the message is serialized and sent over the socket to the
        connected server.
\end{itemize}

%------------------------------------------------------------------------------

\subsubsection{SocketMessage
    \label{class-socket-message}
}

SocketMessage is the base class for all messages going between sockets.  Each
module can have an implementation of this SocketMessage in order to send
different types of data.  The SocketMessage interface is a dictionary,
providing easy access to the message contents.  When the message is sent, it is
serialized and sent to the socket.

\paragraph{Attributes}
\begin{itemize}
    \item \textbf{data}: A simple dictionary containing all the data in this
        message in a key-value store.
\end{itemize}

\paragraph{Methods}
\begin{itemize}
    \item \textbf{\_\_init\_\_}: The initializer for the data.  This can accept
        any keword arguments, which will be loaded into the message.
    \item \textbf{\_\_str\_\_}: This will return the message as a serialized
        string ready for transmission over a socket.
\end{itemize}

%------------------------------------------------------------------------------

\clearpage

\subsection{Socket Messaging Classes
    \label{socket_messaging_classes}
}

\begin{figure}
    \centering
    \caption{Class diagram for socket messages}
    \label{fig-socketmessages}
%BEGIN IMAGE
    \begin{tikzpicture}
        \umlclass{socketmessage}{SocketMessage}{
            \umlattr{\#}{data}{Dictionary}
        }{
            \umlmethod{+}{\_\_init\_\_}{
                    **kwargs        
            }{void}
            \umlmethod{+}{\_\_str\_\_}{}{String}
        }
        \umlclass[below right of=socketmessage,yshift=1cm]{interfacesocketmessage}{InterfaceSocketMessage}{
            \umlattr{\#}{data}{Dictionary}
        }{
            \umlmethod{+}{\_\_init\_\_}{
                    **kwargs        
            }{void}
        }
        \umlclass[below of=socketmessage,yshift=0.5cm]{protocolsocketmessage}{ProtocolSocketMessage}{
            \umlattr{\#}{data}{Dictionary}
        }{
            \umlmethod{+}{\_\_init\_\_}{
                    **kwargs        
            }{void}
        }  
        \umlclass[below left of=socketmessage,yshift=1cm]{invocationsocketmessage}{InvocationSocketMessage}{
            \umlattr{\#}{data}{Dictionary}
        }{
            \umlmethod{+}{\_\_init\_\_}{
                    **kwargs        
            }{void}
        }      
        \umlrelation[|-]{invocationsocketmessage}{isa}{socketmessage}{}{}
        \umlrelation[|-]{interfacesocketmessage}{isa}{socketmessage}{}{}
        \umlrelation[--]{protocolsocketmessage}{isa}{socketmessage}{}{}
    \end{tikzpicture}
%END IMAGE
%HEVEA\imageflush
\end{figure}

The Socket Messaging Classes are specializations of the
\nameref{class-socket-message} class used for input/output types of various
processors and filters.  The objects themselves take only keyword arguments
with a \_\_str\_\_ method for serialization.  Figure~\ref{fig-socketmessages}
shows the relationship between these messages.

%------------------------------------------------------------------------------

\subsubsection{ProtocolSocketMessage}

Class inheriting from a SocketMessage that contains meta data about a protocol
invocation.

%------------------------------------------------------------------------------

\subsubsection{InvocationSocketMessage}

Class inheriting from a SocketMessage that contains a full invocation of a protocol.

%------------------------------------------------------------------------------

\subsubsection{InterfaceSocketMessage}

Class from a processor or filter that outputs only to the input socket of an
interface module.

%------------------------------------------------------------------------------

\clearpage

\subsection{Kernel Module Classes
    \label{kernel_classes}
}

The Kernel Module is the core of the system.  It is in charge if spawning other
modules and coordinating messaging between them.  It is made up of the
\nameref{class-process-manager}, \nameref{class-history},
\nameref{class-uddi-registry} classes.

%------------------------------------------------------------------------------

\begin{figure}
    \centering
%BEGIN IMAGE
    \begin{tikzpicture}
        % --  ProcessManager
        \umlclass{ProcessManager}{ProcessManager}{
            \umlattr{-}{processes}{list(Process)}
        }{
            \umlmethod{+}{\_\_init\_\_}{}{void}
            \umlmethod{+}{create\_process}{
                String name, list(Process inputs), \{options\}
            }{Process}
            \umlmethod{+}{stop\_process}{Process proc}{None}
            \umlmethod{+}{get\_running\_processes}{}{list(Process)}
            \umlmethod{+}{register\_process}{String name, Process p}{}
        }

        % --  Process
        \umlclass[below right of=ProcessManager,xshift=2cm,yshift=-1cm]{Process}{Process}{
            \umlattr{-}{path}{String}
            \umlattr{-}{out\_socket\_path}{String}
            \umlattr{-}{input}{SocketMessage}
            \umlattr{-}{output}{SocketMessage}
        }{
            \umlmethod{+}{\_\_init\_\_}{
                    path, list(SocketMessage input), list(SocketMessage output)
            }{void}
            \umlmethod{+}{get\_input}{}{list(SocketMessage)}
            \umlmethod{+}{get\_output}{}{list(SocketMessage)}
            \umlmethod{+}{create}{list(Process in), String output\_socket\_path, dict(options)}{Popen}
        }
        \umlrelation{ProcessManager}{hasa}{Process}{}{}
    \end{tikzpicture}
%END IMAGE
%HEVEA\imageflush
    \caption{Class diagram for Process Management}
    \label{fig-kernel_process_mgmt}
\end{figure}

\subsubsection{ProcessManager
    \label{class-process-manager}
}

The ProcessManager class is the main class in the Kernel module.  It is in
charge of spawning other processes (\nameref{class-process}), setting their
options, and connecting their inputs and outputs.
Figure~\ref{fig-kernel_process_mgmt} shows its relation to the
\nameref{class-process} class.

\paragraph{Attributes}
\begin{itemize}
    \item \textbf{processes}: A list of \nameref{class-process} objects that
        the ProcessManager is maintaining.
\end{itemize}

\paragraph{Methods}
\begin{itemize}
    \item \textbf{\_\_init\_\_}: Default constructor with no arguments.
        Initializes all data structures.
    \item \textbf{create\_process}: Given a name of a registered process, start
        the \nameref{class-process}, set its inputs to read from the list of
        input processes, and set the options specified.
    \item \textbf{stop\_process}: Given an instance of an already running
        \nameref{class-process}, stop the process and ensure its sockets are
        cleaned up.
    \item \textbf{get\_running\_processes}: Returns a list of
        \nameref{class-process} that are running and being managed by the
        ProcessManager.
    \item \textbf{register\_process}: Given a string name, and a new
        \nameref{class-process} instance, makes this known to the
        ProcessManager, so that it can be started by name.
\end{itemize}

%------------------------------------------------------------------------------

\subsubsection{Process
    \label{class-process}
}

The Process class encapsulates the information about a single process available
to be started.  It contains the path to the binary to launch when this process
is started and has information on the \nameref{class-socket-message} subtypes
that the process expects as input, and will provide as output.

\paragraph{Attributes}
\begin{itemize}
    \item \textbf{path}: A string pointing to the filesystem path of the binary
        to launch when an instance of this process is started.
    \item \textbf{out\_socket\_path}: A string pointing to the filesystem path of the socket
        this process will use to distribute its output.
    \item \textbf{input}: A \nameref{class-socket-message} subclass specifying what
        this process expects to receive as input.
    \item \textbf{output}: A \nameref{class-socket-message} subclass specifying what
        this process will produce as output.
\end{itemize}

\paragraph{Methods}
\begin{itemize}
    \item \textbf{\_\_init\_\_}: The constructor specifying initial values for
        all of the attributes.
    \item \textbf{get\_input}: Returns the types that this process expects as input.
    \item \textbf{get\_output}: Returns the types that this process produces as output.
    \item \textbf{create}: Given a list of processes to use as input, a path
        for the output socket, and options, this spawns a new instance of the
        process pointed to by the \textbf{path} attribute.
\end{itemize}

%------------------------------------------------------------------------------

\subsubsection{History
    \label{class-history}
}

\begin{figure}
    \centering
%BEGIN IMAGE
    \begin{tikzpicture}
        % --  History
        \umlclass[text width = 7cm]{History}{History}{
            \umlattr{-}{history}{list(SocketMessages)}
        }{
            \umlmethod{+}{\_\_init\_\_}{}{void}
            \umlmethod{+}{log\_message}{SocketMessage message}{None}
            \umlmethod{+}{get\_all\_messages}{SocketMessage *type}{None}
            \umlmethod{+}{dump}{File f}{}
        }
    \end{tikzpicture}
%END IMAGE
%HEVEA\imageflush
    \caption{Class diagram for History.  Asterisk denotes an optional variable.}
    \label{fig-kernel_history}
\end{figure}

The History class is used as a data cache for information received from other
modules.  This is to reduce redundant processing in cases where a filter is
applied and then removed.  Figure~\ref{fig-kernel_history} shows the attributes
and methods of this class.

\paragraph{Attributes}
\begin{itemize}
    \item \textbf{history}: A list of \nameref{class-socket-message} instances
        of messages sent during system execution.  This serves as a local cache
        of all PCap data previously sent during this execution cycle in case
        new filter/processors are created that need to run against data already
        processed.
\end{itemize}

\paragraph{Methods}
\begin{itemize}
    \item \textbf{\_\_init\_\_}: Default constructor that initializes data
        structures.
    \item \textbf{log\_message}: Takes a \nameref{class-socket-message} instance and
        stores it in the cache.
    \item \textbf{get\_all\_messages}: If the optional ``type'' field is None,
        then return all \nameref{class-socket-message} instances in the cache.
        If ``type'' is a valid \nameref{class-socket-message} instance, only
        return items in the history list of that type.
    \item \textbf{dump}: Given a File (or File-like object), dump all messages
        cached to a file while leaving the history object unchanged.
    \item \textbf{clear}: Remove all previously stored PCap data from the
        history object.
\end{itemize}

%------------------------------------------------------------------------------

\subsubsection{UDDIRegistry
    \label{class-uddi-registry}
}

\begin{figure}
    \centering
%BEGIN IMAGE
    \begin{tikzpicture}
        % --  UDDI Registry
        \umlclass{UDDIRegistry}{UDDIRegistry}{
            \umlattr{-}{registry}{list(Service)}
        }{
            \umlmethod{+}{\_\_init\_\_}{}{void}
            \umlmethod{+}{register\_service}{Service service}{}
            \umlmethod{+}{unregister\_service}{Service service}{}
            \umlmethod{+}{get\_UDDI}{Service}{String definition}
        }
    \end{tikzpicture}
%END IMAGE
%HEVEA\imageflush
    \caption{Class diagram for UDDI Registry.}
    \label{fig-kernel_uddi_registry}
\end{figure}

The UDDIRegistry class provides an interface to an external UDDI registry
system.  It facilitates registering and unregistering services from the
registry.  Also, it provides a method of receiving the UDDI XML definitions
from a registry given a service.  Figure~\ref{fig-kernel_uddi_registry} shows a
diagram of the UDDIRegistry class structure.

\paragraph{Attributes}
\begin{itemize}
    \item \textbf{registry}: A list of services in the registry.
\end{itemize}

\paragraph{Methods}
\begin{itemize}
    \item \textbf{\_\_init\_\_}: Default constructor that initializes data
        structures.
    \item \textbf{register\_service}: Given a service instance, add that
        service to the registry.
    \item \textbf{unregister\_service}: Given a service instance, remove that
        service from the registry.  If the service is not in the registry,
        silently ignore it.
    \item \textbf{get\_UDDI}: Given a service instance, return the UDDI XML
        definitions of that service from the registry.
\end{itemize}

%------------------------------------------------------------------------------

\clearpage

\subsection{Capture Module Classes
    \label{capture_classes}
}

\begin{figure}
    \centering
%BEGIN IMAGE
    \begin{tikzpicture}
        \umlclass{capturemodule}{CaptureModule}{
            \umlattr{\#}{pcap}{libPcap}
            \umlattr{\#}{output\_socket}{SocketServer}
            \umlattr{-}{output\_file}{File}
            \umlattr{-}{input\_file}{File}
            \umlattr{-}{is\_loading\_from\_file}{Boolean}
        }{
            \umlmethod{+}{\_\_init\_\_}{
                  \umlarg{ String }{output\_socket\_path},
                  \umlarg{ String }{output\_filepath},
                  \umlarg{ String }{input\_filepath},
                  \umlarg{ Boolean }{load\_from\_file}
            }{void}
            \umlmethod{+}{get\_input\_file}{}{String}
            \umlmethod{+}{get\_output\_file}{}{String}
            \umlmethod{+}{is\_loading\_from\_file}{}{Boolean}
        }
        \node [umlclass, below of=capturemodule, yshift=-5mm] (socketserver) {\textbf{SocketServer}};
        \umlrelation[--]{capturemodule}{hasa}{socketserver}{}{}
    \end{tikzpicture}
%END IMAGE
%HEVEA\imageflush
    \caption{Class diagram for capture module.  Note:
        \nameref{class-socket-server}previously defined.}
    \label{fig-capturemodule}
\end{figure}

The CaptureModule is responsible for reading packets from the network, or from
a file, and sending them to the kernel for further processing.  The
CaptureModule uses libpcap in order to accomplish this.
Figure~\ref{fig-capturemodule} shows the interface to this module.  This module
has a \nameref{class-socket-server} instance in order to communicate its
output.

\paragraph{Attributes}
\begin{itemize}
    \item \textbf{pcap}: An instance of the libpcap library.
    \item \textbf{output}: An instance of \nameref{class-socket-server}.  Used
        for sending output to clients listening for it.
    \item \textbf{output\_file}: If not None, specifies a file to log all raw
        packets captured.
    \item \textbf{input\_file}: Specifies a file to read raw packets from, or a
        live capture device.
    \item \textbf{is\_loading\_from\_file}: Returns whether the String returned
        by ``get\_output\_file'' is a filesystem path (True) or a device name
        (False).
\end{itemize}

\paragraph{Methods}
\begin{itemize}
    \item \textbf{\_\_init\_\_}: Constructor function used to initialize
        attributes.
    \item \textbf{get\_input\_file}: Returns the String path to the current
        input file or the name of the capture device being used.
    \item \textbf{get\_output\_file}: Returns the String path of the packet
        output file, if any.
    \item \textbf{is\_loading\_from\_file}: Returns whether the String returned
        by ``get\_output\_file'' is a filesystem path (True) or a device name
        (False).
\end{itemize}

%------------------------------------------------------------------------------

\clearpage

\subsection{Processor Module Classes
    \label{processor_classes}
}

\begin{figure}
    \centering
%BEGIN IMAGE
    \begin{tikzpicture}
        \node [umlclass] (socketserver) {\textbf{SocketServer}};
        \umlclass[above right of=socketserver]{processor}{Processor}{
            \umlattr{\#}{socket}{SocketServer}
        }{
            \umlmethod{+}{run}{
                \umlarg{void}{}
            }{void}
        }
        \umlclass[below right of=processor,yshift=1cm]{servicelistprocessor}{ServicelistProcessor}{
            \umlattr{\#}{socket}{SocketServer}
        }{
            \umlmethod{+}{run}{
                \umlarg{void}{}
            }{void}
        }
        \umlclass[below of=processor,xshift=4cm]{otherprocessor}{OtherProcessors}{
            \umlattr{\#}{socket}{SocketServer}
        }{
            \umlmethod{+}{run}{
                \umlarg{void}{}
            }{void}
        }
        \umlrelation{processor}{hasa}{socketserver}{}{}
        \umlrelation[|-]{servicelistprocessor}{isa}{processor}{}{}
        \umlrelation[-|]{otherprocessor}{isa}{processor}{}{}
    \end{tikzpicture}
%END IMAGE
%HEVEA\imageflush
    \caption{Class diagrams for processor modules.   Note:
        \nameref{class-socket-server} previously defined.}
    \label{fig-processors}
\end{figure}

The processor modules are runnable objects with sockets that perform operations
from socket messages.  See Figure~\ref{fig-processors} for the structure of
this class.


\subsubsection{Processor}
\paragraph{Attributes}
\begin{itemize}
    \item \textbf{socket}: An instance of \nameref{class-socket-server} used to
        communicate with the processor.
\end{itemize}

\paragraph{Methods}
\begin{itemize}
    \item \textbf{run}: Starts the processor with specified settings.
\end{itemize}

\paragraph{Attributes}
\begin{itemize}
    \item \textbf{pcap}: An instance of the libpcap library.
    \item \textbf{output}: An instance of \nameref{class-socket-server}.  Used
        for sending output to clients listening for it.
    \item \textbf{output\_file}: If not None, specifies a file to log all raw
        packets captured.
    \item \textbf{input\_file}: Specifies a file to read raw packets from, or a
        live capture device.
    \item \textbf{is\_loading\_from\_file}: Returns whether the String returned
        by ``get\_output\_file'' is a filesystem path (True) or a device name
        (False).
\end{itemize}

\paragraph{Methods}
\begin{itemize}
    \item \textbf{\_\_init\_\_}: Constructor function used to initialize
        attributes.
    \item \textbf{get\_input\_file}: Returns the String path to the current
        input file or the name of the capture device being used.
    \item \textbf{get\_output\_file}: Returns the String path of the packet
        output file, if any.
    \item \textbf{is\_loading\_from\_file}: Returns whether the String returned
        by ``get\_output\_file'' is a filesystem path (True) or a device name
        (False).
\end{itemize}


%------------------------------------------------------------------------------
% Its too late to write stuff for this, try again later... --Tom
%\clearpage
%
%\subsection{User Interface Module Classes
%    \label{ui_classes}
%}


%             6.2.1.3 Representation 
%                 The literature on software engineering describes a number of
%                 methods that provide consistent criteria for entity
%                 decomposition (see Freeman and Wasserman, Tutorial on Software
%                 Design Techniques ). These methods provide for designing
%                 simple, independent entities and are based on such principles
%                 as structured design and information hiding.  The primary
%                 graphical technique used to describe system decomposition is a
%                 hierarchical decomposition diagram. This diagram can be used
%                 together with natural language descriptions of purpose and
%                 function for each entity. 

%\begin{figure}
%    \centering
%    \begin{tikzpicture}
%        \umlclass{main}{Main}{
%            \umlattr{+}{foo}{String}
%            \umlattr{-}{bar}{Integer}
%        }{
%            \umlmethod{+}{DoFoo}{
%                \umlarg{void}{}
%            }{Integer}
%            \umlmethod{-}{getThing}{
%                \umlarg{String}{foo},
%                \umlarg{Integer}{bar},
%            }{void}
%        }
%        \umlclass[below left of=main]{other}{Other}{
%            \umlattr{+}{foo}{String}
%            \umlattr{-}{bar}{Integer}
%        }{
%            \umlmethod{+}{DoFoo}{
%                \umlarg{void}{}
%            }{Integer}
%            \umlmethod{-}{getThing}{
%                \umlarg{String}{foo},
%                \umlarg{Integer}{bar},
%            }{void}
%        }
%        \umlclass[below right of=main]{cool}{Cool}{
%            \umlattr{+}{foo}{String}
%            \umlattr{-}{bar}{Integer}
%        }{
%            \umlmethod{+}{DoFoo}{
%                \umlarg{void}{}
%            }{Integer}
%            \umlmethod{-}{getThing}{
%                \umlarg{String}{foo},
%                \umlarg{Integer}{bar},
%            }{void}
%        }
%        \umlrelation{main}{hasa}{other}{1..*}{1}
%        \umlrelation{main}{ownsa}{cool}{1}{1}
%        \umlrelation[--]{other}{isa}{cool}{}{}
%    \end{tikzpicture}
%    \caption{Test UML class diagram}
%\end{figure}


%\begin{figure}
%  \centering
%  \begin{sequencediagram}
%  \tikzstyle{inststyle}+=[rounded corners=3.2mm, bottom color=cyan] %% hier werden weitere Optionen angegeben
%    \newthread{ss}{simulationServer}
%    \newinst{ctr}{simControlNode}
%    \newinst{ps}{physicsServer}
%    \newinst[1]{sense}{senseServer}
%    
%    \begin{call}{ss}{Initialize()}{sense}{}
%    \end{call}
%    \begin{sdloop}{Run Loop}
%      \begin{call}{ss}{StartCycle()}{ctr}{}
%        \begin{call}{ctr}{ActAgent()}{sense}{}
%        \end{call}
%      \end{call}
%      \begin{call}{ss}{Update()}{ps}{}
%        \begin{call}{ps}{PrePhysicsUpdate()}{sense}{state}
%        \end{call}
%        \begin{callself}{ps}{PhysicsUpdate()}{}
%        \end{callself}
%        \begin{call}{ps}{PostPhysicsUpdate()}{sense}{}
%        \end{call}
%      \end{call}
%      \begin{call}{ss}{EndCycle()}{ctr}{}
%        \begin{call}{ctr}{SenseAgent()}{sense}{}
%        \end{call}
%      \end{call}
%    \end{sdloop}
%  \end{sequencediagram}
%
%  \caption{UML sequence diagram demo. The used style-file is part of pgf-umlsd-0.2.tar.gz
%Get it at http://code.google.com/p/pgf-umlsd/}
%\end{figure}

%------------------------------------------------------------------------------

\clearpage

\section{Dependencies
    \label{Dependencies}
}

%         6.2.2 Dependency description 
%             6.2.2.1 Scope 
%                 The dependency description specifies the relationships among
%                 entities. It identifies the dependent entities, describes their
%                 coupling, and identifies the required resources.  This design
%                 view defines the strategies for interactions among design
%                 entities and provides the information needed to easily perceive
%                 how, why, where, and at what level system actions occur. It
%                 specifies the type of relationships that exist among the
%                 entities such as shared information, prescribed order of
%                 execution, or well-defined parameter interfaces.  The attribute
%                 descriptions for identification, type, purpose, dependencies,
%                 and resources should be included in this design view. This
%                 attribute information should be provided for all design
%                 entities. 
% 
%             6.2.2.2 Use 
%                 The dependency description provides an overall picture of how
%                 the system works in order to assess the impact of requirements
%                 and design changes. It can help maintainers to isolate entities
%                 causing system failures or resource bottlenecks. It can aid in
%                 producing the system integration plan by identifying the
%                 entities that are needed by other entities and that must be
%                 developed first. This description can also be used by
%                 integration testing to aid in the production of integration
%                 test cases. 
% 
%             6.2.2.3 Representation 
%                 There are a number of methods that help minimize the
%                 relationships among entities by maximizing the relationship
%                 among elements in the same entity. These methods emphasize low
%                 module coupling and high module cohesion (see Freeman and
%                 Wasserman, Tutorial on Software Design Techniques). 
% 
%                 Formal specification languages provide for the specification of
%                 system functions and data, their interrelationships, the inputs
%                 and outputs, and other system aspects in a well-defined
%                 language. The relationship among design entities is also
%                 represented by data flow diagrams, structure charts, or
%                 transaction diagrams. 
% 

%------------------------------------------------------------------------------

While the system is highly modular, most of the modules require each other to
run.  The only module that can run on its own is the capture module, which can
be run alone to collect and save data from the network for later analysis.  The
capture, filtering, and user interface modules all only communicate to (and
therefore through) the system kernel module.  When running ServiceSniffer, if
any modules other than just the capture module need to be used, the system
kernel must also be used.  

As the system kernel is the central control point for ServiceSniffer, it is
needed to run the system. The kernel holds the system together passing
information between all of the other modules with in the system. Spawning the
processes that complete the work for the filters and processors, this module
is required for all other modules to be able to run.

The filter module is loosely coupled with the rest of the system. Filters and
processors can be put into any length chains at any point. This allows a filter
design to be used in multiple points with in the system. Also, with a
standardized method of specifying what to filter on, the interface with filters
of different types easier and more fluid.

Data flows through the ServiceSniffer system through the capture module
(regardless if it's live or previously saved data) to the system kernel.  The
system kernel then passes this data off to the filter module, which performs
its work and returns the new, processed data to the kernel.  The kernel then
sends the information to the user interface module which presents it to the
user.

%------------------------------------------------------------------------------

\clearpage

\section{Interfaces
    \label{interfaces}
}

\begin{figure}
    %\includegraphics[width=1\textwidth]{../presentation/gfx/dataflow_base}
    \caption{Dataflow}
    \label{fig-dataflow}
\end{figure}

%------------------------------------------------------------------------------

The following interfaces facilitate all communication over sockets. The sockets
are named pipes on a single system. This allows for the ability to provide a
standard interface among all the modules.

%------------------------------------------------------------------------------

\subsection{Capture Module
    \label{capture}
}
This is a one way communication from the capture module to the system kernel.
Using libpcap, the capture module takes incoming packets, from either the
network or a pcap file, and transform them into a data structure that in can
be used by the kernel and other parts of the system. A pickled version of the
packet will be sent over the socket to the kernel.

\begin{figure}
    \centering
    %\includegraphics[width=.5\textwidth]{gfx/input.png}
    \caption{Dataflow for the capture module}
    \label{fig-dataflow-input}
\end{figure}

%------------------------------------------------------------------------------

\subsection{Filter Module
    \label{filter}
}
The filter module takes in a set of data and returns a subset of the input
depending on provided criteria. Each data type will have its own filter module,
but they will all share a common filtering language.

\begin{figure}
    \centering
    %\includegraphics[width=.5\textwidth]{gfx/filter.png}
    \caption{Dataflow for a generic filter}
    \label{fig-dataflow-filter}
\end{figure}

%------------------------------------------------------------------------------

\subsection{User Interface Module
    \label{ui}
}

The user interface module will tie in with the system kernel through a two way
socket channel. This will allow for the module to be able to queries the
kernel, along with the kernel send information back to the module.
Figures~\ref{fig-gui-main} and \ref{fig-gui-uddi} show mockups of a sample
interface.


\begin{figure}
    \centering
    %\includegraphics[width=.95\textwidth]{../presentation/gfx/gui-main}
    \caption{Mockup of the main screen}
    \label{fig-gui-main}
\end{figure}

\begin{figure}
    \centering
    %\includegraphics[width=.95\textwidth]{../presentation/gfx/gui-uddi}
    \caption{Mockup of the UDDI screen}
    \label{fig-gui-uddi}
\end{figure}


%             6.2.3 Interface description 
%                 6.2.3.1 Scope 
%                     The entity interface description provides everything
%                     designers, programmers, and testers need to know to
%                     correctly use the functions provided by an entity. This
%                     description includes the details of external and internal
%                     interfaces not provided in the software requirements
%                     specification. 
% 
%                     This design view consists of a set of interface
%                     specifications for each entity. The attribute descriptions
%                     for identification, function, and interfaces should be
%                     included in this design view. This attribute information
%                     should be provided for all design entities. 
% 
%                 6.2.3.2 Use 
%                     The interface description serves as a binding contract
%                     among designers, programmers, customers, and testers. It
%                     provides them with an agreement needed before proceeding
%                     with the detailed design of entities. In addition, the
%                     interface description may be used by technical writers to
%                     produce customer documentation or may be used directly by
%                     customers. In the latter case, the interface description
%                     could result in the production of a human interface view. 
% 
%                     Designers, programmers, and testers may need to use design
%                     entities that they did not develop. These entities may be
%                     reused from earlier projects, contracted from an external
%                     source, or produced by other developers. The interface
%                     description settles the agreement among designers,
%                     programmers, and testers about how cooperating entities
%                     will interact. Each entity interface description should
%                     contain everything another designer or programmer needs to
%                     know to develop software that interacts with that entity. A
%                     clear description of entity interfaces is essential on a
%                     multiperson development for smooth integration and ease of
%                     maintenance. 
% 
%             6.2.3.3 Representation 
%                 The interface description should provide the language for
%                 communicating with each entity to include screen formats, valid
%                 inputs, and resulting outputs. For those entities that are data
%                 driven, a data dictionary should be used to describe the data
%                 characteristics. Those entities that are highly visible to a
%                 user and involve the details of how the customer should
%                 perceive the system should include a functional model,
%                 scenarios for use, detailed feature sets, and the interaction
%                 language. 
% 

%------------------------------------------------------------------------------

%\section{Detailed Design Description
%    \label{detail}
%}

%         6.2.4 Detailed design description 
%             6.2.4.1 Scope 
%                 The detailed design description contains the internal details
%                 of each design entity. These details include the attribute
%                 descriptions for identification, processing, and data. This
%                 attribute information should be provided for all design
%                 entities. 
% 
%             6.2.4.2 Use 
%                 This description contains the details needed by programmers
%                 prior to implementation. The detailed design description can
%                 also be used to aid in producing unit test plans. 
% 
%             6.2.4.3 Representation 
%                 There are many tools used to describe the details of design
%                 entities. Program design languages can be used to describe
%                 inputs, outputs, local data and the algorithm for an entity.
%                 Other common techniques for describing design entity logic
%                 include using metacode or structured English, or graphical
%                 methods such as Nassi-Schneidermann charts or flowcharts. 
% 
% 

%------------------------------------------------------------------------------

\end{document}
