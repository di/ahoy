\documentclass[titlepage]{article}

\usepackage{fullpage}   % Use the whole page
\usepackage{fancyhdr}   % Nice headers/footers
\usepackage{graphicx}   % Importing graphics
\usepackage{mdwlist}    % For itemize* and enumerate*
\usepackage{hyperref}   % Hyperlink references and URLs
\usepackage{subfig}
\usepackage{appendix}
\usepackage{listings}
%\usepackage{todonotes}

\hypersetup{
    colorlinks=true,
    linkcolor=blue,
    urlcolor=blue,
    pdftitle={STAGE Software Requirements Specification v1},
    pdfauthor={Dustin Ingram, Aaron Rosenfeld, Maria Kolakowska, Frank Clark}
}

% So we can number paragraphs too
\setcounter{secnumdepth}{5}

% For headers and footers
\setlength{\headheight}{15pt}
\setlength{\headsep}{25pt}
\pagestyle{fancy}
	
% Page style for the title page
\fancypagestyle{plain}{
    \fancyhf{}
    \renewcommand{\headrulewidth}{0pt}
    \renewcommand{\footrulewidth}{0pt}
}

% Just so we don't have to specify this twice
\newcommand\mytitle{Software Requirements Specification}
\newcommand\mydate{November 1, 2010}

% Page style for every other page
\fancyhf{} % clear all header and footer fields
\fancyhead[L]{STAGE}
\fancyhead[C]{\mytitle}
\fancyhead[R]{\mydate}
\fancyfoot[C]{\thepage}
\renewcommand{\headrulewidth}{0.4pt}
\renewcommand{\footrulewidth}{0.4pt}

\title{\textbf{\mytitle}}
\author{
    Dustin Ingram \\\url{dsi23@drexel.edu}
    \and Aaron Rosenfeld \\\url{ar374@drexel.edu}
    \and Maria Kolakowska \\\url{mjk75@drexel.edu}
    \and Frank Clark \\\url{fjc28@drexel.edu}
}
\date{\mydate\\Version 1}

\begin{document}
\pagenumbering{roman}

% \TODO %%%%%%%%%%%%%%%%%%%%%%%%%%%%%%%%%%%%%%%%%%%%
% Stick our (vectorized) logo in here...
%
%\begin{figure%
%	\label{logo}%
%}
%    \vspace{-6em}
%    \centering
%    \includegraphics[width=0.6\textwidth]{../logo}
%    \vspace{-4em}
%\end{figure}
%%%%%%%%%%%%%%%%%%%%%%%%%%%%%%%%%%%%%%%%%%%%%%%%%\ %

\maketitle

\begin{abstract}
STAGE provides the capability to test the veracity of system-level sensors and component-level algorithms in the laboratory against virtual operations using data from simulated sensors and nodes. These nodes will work with scripted or real time data, depending on the user's need. This network and sensor testbed allows users to independently vary its topology, application suites, and the environment. It also emulates the communication between these components. A visualization tool is provided to the user for demoing and analyzing.
\end{abstract}

\setcounter{tocdepth}{4}
\tableofcontents
\pagebreak
\pagenumbering{arabic}

%\listoftodos

%___________________________________________________________________________

\section{Introduction%
  \label{introduction}%
}

%___________________________________________________________________________

\subsection{Purpose%
  \label{purpose}%
}

This requirements document defines the functional and non-functional requirements for the STAGE project.  These requirements will include, but are not limited to, the performance, interfaces, and hardware goals and limitations.  The information presented here is intended for the development team and the external stakeholders, currently Dr. William Regli, Mr. Joseph B. Kopena, Mr. Joeseph P. Macker, and The U.S. Naval Research Laboratory.

%___________________________________________________________________________

\subsection{Scope%
  \label{scope}%
}

% \TODO %%%%%%%%%%%%%%%%%%%%%%%%%%%%%%%%%%%%%%%%%%%%
% Define the scope of STAGE, break it down into the components
The goal of the --------- project is to automate the passive discovery,
recognition, and consumption of semantic web services.  --------- does not
require any central registry of services but will examine network traffic
patterns to identify specific services, how to invoke them and report
specific, user chosen information.  As well as running with known service
definitions, the system will identify new services and analyze
relevant network traffic to determine how to consume the service.  This
application will be constructed of an extensible core framework upon which
future modules can be developed and integrated.

--------- is split into 3 general pieces: the data input, data processing,
and user interface modules.  The data input module is responsible for
collecting traffic off the network or using a Pcap file as a source instead of
a live network.  The data processing module, which is constructed from various
filters and processors (which will be described in more depth later), applies
filters to collected information and transforms it into
various and configurable reports and views depending on what the user wants to
see.  Some examples of these reports will be described later.  Lastly,
--------- has both a CLI and a GUI for viewing the various data returned
by the data processors and filters.  These interfaces will allow users to
interact with the data returned.  For example, a user could apply additional
display filters, such as filtering by a specific destination IP.  Both
interfaces will offer to users a report of what web services appear to be
available on the network and supply a report - an ad-hoc registry of sorts -
which users can use to consume the web service.  Users may specify a additional
types of filters or processors to generate different types of reports via an
extension system.

Users of --------- are chunked into 3 loose groups: system/network
administrators who wish to control all of the --------- pieces (data
input, data processing, and the user interface), developers who wish to or are
tasked with developing custom data filters and/or processors so users can have
new/custom reports, and UI users who only wish to analyze the data collected
and processed by ---------.
%%%%%%%%%%%%%%%%%%%%%%%%%%%%%%%%%%%%%%%%%%%%%%%%%\ %

%___________________________________________________________________________

% \TODO %%%%%%%%%%%%%%%%%%%%%%%%%%%%%%%%%%%%%%%%%%%%
% Add ~5 relevant definitions here...
\subsection{Definitions, Acronyms, and Abbreviations%
  \label{definitions}%
}

\begin{description}
\item[Foobar]
    A foobar

\end{description}
%%%%%%%%%%%%%%%%%%%%%%%%%%%%%%%%%%%%%%%%%%%%%%%%%\ %

%___________________________________________________________________________

% \TODO %%%%%%%%%%%%%%%%%%%%%%%%%%%%%%%%%%%%%%%%%%%%
% Do we have any references? Add them here
\subsection{References%
  \label{references}%
}

These documents explain in-depth about the various technologies involved in this project.
%
\begin{itemize*}
	\item WSDL Specification: \url{http://www.w3.org/TR/wsdl20/}
	\item SOAP Specification: \url{http://www.w3.org/TR/soap/}
	\item libpcap Homepage: \url{http://www.tcpdump.org/}
	\item MVP-E Project (Team 10): \url{http://www.cs.drexel.edu/SeniorDesign/2010Material/Projects2010.html}
\end{itemize*}
%%%%%%%%%%%%%%%%%%%%%%%%%%%%%%%%%%%%%%%%%%%%%%%%%\ %

%___________________________________________________________________________

\subsection{Overview%
  \label{overview}%
}

The rest of this document is organized as follows: 
\begin{itemize*}
	\item[-]Section \ref{overall-description} gives a high level overview of the project requirements.  
	\item[-]Section \ref{specific-requirements} gives more detailed project requirements.  
	\item[-]Sections \ref{input-formats} and \ref{output-formats} provides detail in the inputs and outputs of the system, 
	\item[-]Section \ref{functional} specifies what the software will do, 
	\item[-]Section \ref{performance} gives performance requirements for the underlying system.
\end{itemize*}
The rest of Section \ref{specific-requirements} explains other minor requirements.


%___________________________________________________________________________

\section{Overall Description%
  \label{overall-description}%
}

%___________________________________________________________________________

\subsection{Product Perspective%
  \label{product-perspective}%
}

STAGE can be used as a testbed for sensors and respective algorithms in the laboratory against virtual operations using data from simulated sensors and nodes. The ability to independently and quickly vary the network topology, application suites, environment, or any other variable will be essential to collect data which would be cost-prohitive to produce in a real-world scenario. Furthermore, the visualization tool can be used for conceptual demonstrations.

%___________________________________________________________________________

\subsubsection{System Interfaces%
  \label{system-interfaces}%
}

STAGE combines:
\begin{itemize}
    \item Scenario Language - Defining the entire scenario in a very simple language
    \item Agent Framework - Supporting the customization of semi-intelligent agents
    \item Topology Creator - Creating a virtual topology of nodes defined in the Scenario Language
    \item Distribution Framework - Distributing a large-scale emulation across numerous physical machines
    \item Data-Collection API - Collecting all relevant data through a simple API
    \item Visualization Engine - Conceptually visualizing terrain, node movement, link status, etc.
\end{itemize}

%___________________________________________________________________________

\subsubsection{User Interfaces%
  \label{user-interfaces}%
}

STAGE's user interface is two-fold, with each interface for different tasks and uses. The basic interface allows a user to simply and quickly define a scenario with the Scenario Language and begin an experiment, using the API to collect data in a way that is defined by the user. The Visualization Engine allows the user to interpret the experiment visually, without a need for data-collection or the API.

%___________________________________________________________________________

\subsubsection{Hardware Interfaces%
  \label{hardware-interfaces}%
}

STAGE requires that the machine running the system has one to many network interfaces to support distribution.

%___________________________________________________________________________

\subsubsection{Software Interfaces%
  \label{software-interfaces}%
}

STAGE requires the use of Linux Namespaces' kernel virtualization as well as OpenGL in order to use the visualizer.

%___________________________________________________________________________

\subsubsection{Memory Constraints%
  \label{memory-constraints}%
}

STAGE requires a machine with at least two gigabytes of RAM.

%___________________________________________________________________________

\subsubsection{Site Adaptation Requirements%
  \label{site-adaptation-requirements}%
}

STAGE requires that the system is configured per-site to create a distributed emulation, or to interact on-the-fly.

%___________________________________________________________________________

% \TODO %%%%%%%%%%%%%%%%%%%%%%%%%%%%%%%%%%%%%%%%%%%%
% Define the product's functions. This should be ~5 paragraphs
\subsection{Product Functions%
  \label{product-functions}%
}

This software performs two major functions: collecting network traffic related
to web services, and anyalizing collected data in a user specified
way using built-in or user-defined filters and processors.  In performing these
actions separately, either piece may be substituted by third-party software
which performs the same actions and communicates over the same interface
described in this document.  

Network administrators, among other technical people, want to know what web
services are being used on their network.  Some users may not report that
they're using specific web services on a network when they are, which may 
conflict with other services.  By being able to see all of the web service
invocations on a network, an administrator can determine where rogue web
services are being used and better determine whether they offer something
useful on the network.  

Also, not all networks have a central registry of web services offered on the
network, and most do not have a record of what web services from outside the
network are consumed by users on the network.  More, many web services aren't
clearly defined in WSDL or an equivalent format, so it is unclear how to consume
the web service without a formal definition.  --------- constructs an
ad-hoc registry of web services is observes being consumed in a defined way.
This registry can then be used to let users consume these previously undefined
web services.

Some networks have a requirement that all services being used on them be
encrypted as the network itself cannot be trusted.  --------- is able to
show which services being used on a network are secure (either fully encrypted
traffic or encrypted data with cleartext headers) and which are insecure.  By
looking at the specific traffic an administrator can learn what pieces of the
network are vulnerable.

Questions commonly asked consist of: whether a web service is actively being used, if
there's a pattern to who uses a specific web service, or whether there's a
pattern to when a specific web service is used.  Web service developers,
as well as network administrators, frequently want to know what users and/or
servers are using specific web services and when.  This allows them to
determine to whom a specific service is useful, who uses an unauthorized web
service, and other similar situations.

In debugging web services, developers need to see what data is being sent to
the web service during invocation and usage.  Instead of using a program like
Wireshark to have to look at the packets of a web service invocation,
developers can use --------- to see where, if anywhere, the data actually
sent differs from what should be sent.
%%%%%%%%%%%%%%%%%%%%%%%%%%%%%%%%%%%%%%%%%%%%%%%%%\ %

%___________________________________________________________________________

% \TODO %%%%%%%%%%%%%%%%%%%%%%%%%%%%%%%%%%%%%%%%%%%%
% List the user characteristics. Continue w/ the story-line format...
%   Alice - just visualization, demos, etc
%   Bob - algorithms researcher, needs to collect data, uses visualizer just to make sure it's working, uses API, writes many different agents for the same topology
%   Carol - protocol testing, wants to run a single protocol/agent/agent framework over many varying topologies
%   Dave - large-scale emulation, needs to use the distribution framework and cluster rather than his personal computer
\subsection{User Characteristics%
  \label{user-characteristics}%
}

The system has three different types of users:  administrators, developers and
traditional users.  Each type of user has different goals and needs to use
different parts of the system.  These user roles, while they may be performed
by the same user, are here split into the three distinct roles for clairty:

%___________________________________________________________________________

\subsubsection{Users: Jamie, Junior Data Analyst%
  \label{jamie}%
}

Jamie solely uses the GUI to run/view web service analyst reports.  Jamie wants
to complete the tasks below on a network in real-time as allowed by the
administrator, or based on a saved file of a previous capture in the GUI.
%
Tasks:
\begin{itemize*}
    \item Get the full list of all services invoked
    \item See how to invoke a specific service
    \item View common services
    \item View rare services
    \item Finding patterns in service invocations
    \item Finding exceptions (anti-patterns) to the previous patterns
    \item Run reports and filtering by any combination of:
    \begin{itemize*}
        \item specific services
        \item service specific criteria (e.g. Google Maps search vs. Google Maps display)
        \item source IP address
        \item destination IP address
        \item protocol
        \item protocol specific criteria (e.g. filtering SOAP content by Request Time)
        \item date/time
        \item secure services
        \item insecure services
    \end{itemize*}
    \item Run reports with custom filters and processors setup by Susan
\end{itemize*}

%___________________________________________________________________________

\subsubsection{Developers: Susan; Senior Applications Developer%
  \label{susan}%
}

Susan needs to add more functionality to the application.  Susan wants to
complete the following tasks on a network in real-time as allowed by the administrator,
or based on a saved file of a previous capture in either CLI or GUI mode.
%
Tasks:
\begin{itemize*}
    \item Use an API
    \item Run application in CLI mode and/or GUI mode for testing
    \item Inspect their own services
    \item Add new protocols and/or services to be reported on
\end{itemize*}

%___________________________________________________________________________

\subsubsection{Administrators: Zed; Systems Administrator%
  \label{zed}%
}

Zed needs to capture traffic coming over the network card to store it in Pcap
format and/or allow Susan and Jamie to run analysis against the live network
stream.  Zed needs to do this, as capturing from the network card requires root
level access on the computer the system is being run on.  
%
Zed also needs to be able to:
\begin{itemize*}
    \item Deploy the software
    \begin{itemize*}
        \item on a single machine for full network scanning
        \item on a single machine for scanning that same single machine
        \item on a single machine for analysis of previously saved captures
        \item with the GUI frontend (and associated files)
        \item with the Data Processing layer (and associated files)
    \end{itemize*}
    \item Capture from the network in real time
    \item Store the capture in a Pcap file
\end{itemize*}
%%%%%%%%%%%%%%%%%%%%%%%%%%%%%%%%%%%%%%%%%%%%%%%%%\ %
%___________________________________________________________________________

\subsection{Constraints%
  \label{constraints}%
}

The system must be able to interpret and/or process any agent or topology it is given.  If this isn't possible for any reason, the system should inform the user of this. No piece of this system requires elevated privileges.

%___________________________________________________________________________

\subsection{Assumptions and Dependencies%
  \label{assumptions-and-dependencies}%
}

It is assumed that a user, administrator, or developer has the ability to install required libraries which will named in the future as a result of the development process.

%___________________________________________________________________________

% \TODO %%%%%%%%%%%%%%%%%%%%%%%%%%%%%%%%%%%%%%%%%%%%
% Most of these are (obviouly) not relevant to our project. Add our own Functional Requirements!
\section{Specific Requirements%
  \label{specific-requirements}%
}

\subsection{Functional Requirements%
    \label{functional}%
}

\subsubsection{Visualization}
\paragraph{Node Position} The visualization shall display the physical location of network nodes.
\paragraph{Entity Position} The visualization shall display the physical location of entities other than network nodes.

\subsubsection{Topology Definition}
\paragraph{Interfaces} The network definition language shall specify interfaces on each virtual node.  Specifically,
each interface shall be described with a physical model (radio, wired, etc.) and an appropriate link-layer protocol.

\paragraph{Networks} The network definition language shall specify networks to which interfaces may interact.  Networks
are described with a unique identifier and physical layer model (e.g. radio propagation model).

\paragraph{Links} The network definition language shall specify one-to-one links between interfaces.  A link is
described by a physical layer model.

\subsubsection{Scenario Definition}
\paragraph{Mobility} The scenario definition language shall describe the mobility or each virtual node as a function of
time.
\paragraph{Events} Mobility shall be alterable as a result of events, specified in the scenario definition.
\paragraph{Non-node Behavior} The scenario definition language shall describe the movement of and events triggered by
entities other than virtual nodes.

\subsubsection{Agent Framework}
\paragraph{Agent Location} The agent framework shall specify on which virtual node(s) a given agent should operate.
\paragraph{Startup} The agent framework shall specify the startup parameters of each agent.
\paragraph{Collection} The agent framework shall specify the output parameters of each agent instance.

\subsubsection{Distribution Framework}
\paragraph{Physical Distribution} The distribution framework shall distribute virtual nodes evenly across physical
nodes.  Further, it shall handle the establishment of agents on the proper physical node.

\paragraph{Data Collection} The distribution framework shall aggregate simulation data at the end of a trial.

\paragraph{Simulation Server} The distribution framework all organize physical nodes to interact with the simulation
server.

%___________________________________________________________________________

\subsubsection{Data Input} \label{data_input}

\paragraph{Live Capture}

The system shall be able to read live traffic from a network device connected
to the system.  The user must be able to gain the necessary permissions in
order to read data from the network device.  If the system cannot read from the
network device for any reason, the user must be notified with an error, and the
system shall exit.

\paragraph{Offline Capture}

The system shall be able to read stored traffic from a Pcap formatted file.
The file must exist and be readable by the user running the system.  If the
file cannot be read, or is of an incorrect format, the user must be notified
with an error, and the system shall exit.

%___________________________________________________________________________

\subsubsection{Filters}
The system shall use filters in order to reduce the data into specific data
subsets.  These datasets will then be passed into other filters or
processors for analysis.

\paragraph{Secure Connections}

The system shall provide a filter that takes as input a set of service
invocations and will only return secured (encrypted) service invocations.  If
the input is malformed, the filter shall return nothing.

\paragraph{Unsecure Connections}

The system shall provide a filter that takes as input a set of service
invocations and will only return unsecured (unencrypted) service invocations.
If the input is malformed, the filter shall return nothing.

\paragraph{Source IP}

The system shall provide a filter that takes as input a set of available
services and returns a list of available services from a specified source IP
address, if any.  The source IP address is a required parameter for this
filter, if not present the filter shall return all services inputted.  If the
input is malformed, the filter shall return nothing.

\paragraph{Destination IP}

The system shall provide a filter that takes as input a set of available
services and returns a list of available services which have been invoked by a
particular user's IP address.  The user IP address is a required parameter for
this filter, if not present the filter shall return all services inputted.  If
the input is malformed, the filter shall return nothing.

%___________________________________________________________________________

\subsubsection{Processors}

The system shall use processors in order to transform data from one
form to another.  The output data will then be passed into other filters or
processors for analysis.

\paragraph{Service Listing}

The system shall provide a processor which will take as input a set of TCP
packets and return a list of service tuples.  These service tuples shall
contain:
\begin{itemize*}
    \item server name (obtained from DNS)
    \item server IP address
    \item type of service (SOAP, etc)
    \item service definition (WSDL, if possible)
\end{itemize*}
If the input is malformed, the processor should return nothing.

\paragraph{Service Invocations}

The system shall provide a processor which will take as input a set of TCP
packets and returns a list of invocation tuples.  The processor shall take as a
required parameter the IP address and service name of the service in question.
The invocation tuples represent a single service invocation, and shall contain:
\begin{itemize*}
    \item client IP address
    \item timestamp of invocation
    \item data sent (arguments) to service
\end{itemize*}
If the input is malformed, or the required parameters are missing or invalid,
the processor shall return nothing.

\paragraph{Useful Services} 
Within a network, some services are more useful to other users on the network
than others. The usefulness of a service can be determined based on the number
of invocations that are made to it over a given period of time.  Using the
information regarding service invocation, the system shall be able to determine
usefulness of a service relative to the number of invocations.

The output of this type of analysis shall be a list of services containing the
following information:
\begin{itemize*}
    \item server name (obtained from DNS)
    \item server IP address
    \item name of service
    \item number of invocations
    \item most recent invocation (YYYYMMDDHHMMSS)
\end{itemize*}
This list shall be sorted in descending order by the number of invocations.

\paragraph{Abnormal Services} 
With in a network system, there may be a pattern to which services are
accessed. This information is obtained based on what services different users
connect to. After determining a pattern for which users are connecting to a set
of services, a set of services and users that don't follow the pattern can be
abstracted. The system shall be able to determine this list of abnormal
services based on the invocation of different web services by users.
            
The output of this type of analysis shall be a list of abnormal services
containing the following information:
\begin{itemize*}
    \item server name (obtained from DNS)
    \item server IP address
    \item name of service
    \item number of invocations
    \item most recent invocation (YYYYMMDDHHMMSS)
    \item list of IP making it abnormal
\end{itemize*}
This list shall be sorted in order of number of invocations in ascending order.
This will place the most abnormal services at the top of the list.
%%%%%%%%%%%%%%%%%%%%%%%%%%%%%%%%%%%%%%%%%%%%%%%%%\ %

%___________________________________________________________________________

\subsubsection{Data Storage%
  \label{data-storage}%
}
  \begin{enumerate}
    \item No outside database system will be used for data storage, unless it is introduced by the user via the API
    \item Scenario definitions, including the topology and agents, will be stored and parsed by the system
  \end{enumerate}

%___________________________________________________________________________

\subsection{Runtime Environment} 
\begin{itemize*}
    \item Operating Systems:  Unix / Linux / Mac
    \item Processor:  A modern 32- or 64-bit x86 processor
    \item Memory:  1024MB RAM minimum
    \item Network connection:  Any recent network card which is currently connected to the emulation network (if the emulation is distributed)
    \item Visualizer:  A graphics card which will support OpenGL or an equivalent
\end{itemize*}

%___________________________________________________________________________

% \TODO %%%%%%%%%%%%%%%%%%%%%%%%%%%%%%%%%%%%%%%%%%%%
% Determine if these are relevant (probably) and update.
\subsection{Input Formats%
  \label{input-formats}%
}

\subsubsection{Packets}
\begin{enumerate}
    \item The system shall be able to read raw packets from a network device
        with libpcap.
    \item The system shall be able to read raw packets from a file in the PCAP
        format.
\end{enumerate}

\subsubsection{Protocols}
\begin{enumerate}
    \item The system shall support HTTP as a transport protocol.
    \item The system shall support SMTP as a transport protocol.
    \item The system shall be able to recognize SOAP service invocations.  An
        example is in Appendix \ref{a:soap}.
    \item The system shall support extensions to recognize other protocols.
    \item The system shall collate packets from recognized service invocations
        into single logical units.
\end{enumerate}

%___________________________________________________________________________

\subsection{Output Formats%
  \label{output-formats}%
}

\begin{enumerate}
    \item The system shall support exporting a subset of captured packets into
        a user specified PCAP formatted file.
    \item The system shall support exporting user defined PDF/HTML reports from
        data gathered by the system.
    \item The system shall support generating WSDL for discovered web services.
\end{enumerate}
%%%%%%%%%%%%%%%%%%%%%%%%%%%%%%%%%%%%%%%%%%%%%%%%%\ %

%___________________________________________________________________________

% \TODO %%%%%%%%%%%%%%%%%%%%%%%%%%%%%%%%%%%%%%%%%%%%
% Talk about our user interface!
\subsection{User Interface%
  \label{user-interface}%
}

These requirements apply to both the Graphical User Interface (GUI) and the
Command Line Interface (CLI).

\begin{enumerate}
    \item The interface shall provide access to raw unfiltered captured
        web service information.
    \item The interface shall provide the ability to filter web
        services based on user supplied criteria.
    \item The interface shall provide the ability to process web
        service lists to gain more information.
    \item The interface shall give the user the ability to produce
        reports based off the information in the interface.
    \item The interface shall provide a method for providing a
        detailed inspection of web services.
    \item The interface shall provide a method for viewing
        communication chain (Per user queries/responses from a server).
\end{enumerate}
%%%%%%%%%%%%%%%%%%%%%%%%%%%%%%%%%%%%%%%%%%%%%%%%%\ %
%___________________________________________________________________________

\subsection{Extensibility%
  \label{extensibility}%
}
\begin{enumerate}
    \item The system shall provide a means for users to create custom topologies.
    \item The system shall provide a means for users to create custom agents.
\end{enumerate}

\subsection{Testing%
    \label{testing}%
}
All code with in the system must be tested. Core functionality of the software will have 100\% code coverage with automated unit and integration tests.  The visualizer testing will be done manually as it is a smaller component and would otherwise require a disproportionately large investment in initial setup.  Unit tests will cover the algorithms used in our system while integration tests will be used to test our data-flow.  

\section{System Evolution}

With more time, the STAGE team could create additional tools to aid in the the development of more complex agents or topologies, so that user customization does not become too granular.  Furthermore, tools which can automate or assist in data collection and parsing could be implements.  Even further development could lead to a platform that included Windows or other operating systems.

%___________________________________________________________________________

% \TODO %%%%%%%%%%%%%%%%%%%%%%%%%%%%%%%%%%%%%%%%%%%%
% This is particularly relevant! Real (emulated) time vs. simulated time!
\subsection{Performance%
    \label{performance}%
}

%___________________________________________________________________________

\subsubsection{Real Time}\label{realtime} 
The system shall have the ability to run in real time. Real time is defined as
constantly processing the packets as they come off the wire. As the packets are
processed the data will be presented to the user in either graphical form, or
on a console.  Much like \nameref{nonrealtime} execution, all packets which
have been determined to belong to a web service communication as defined in
Section~\ref{definitions} will be stored in a Pcap file for later processing
should the user choose to save this data.

%___________________________________________________________________________

\subsubsection{Non-Real Time}\label{nonrealtime}
The system shall have the ability to run in non-real time. Non-real time is
defined as collecting the packets as they come off the wire and storing them in
a data store for later processing and analysis.  The format for storage of all  
web service packets is Pcap.
%%%%%%%%%%%%%%%%%%%%%%%%%%%%%%%%%%%%%%%%%%%%%%%%%\ %

%___________________________________________________________________________

\subsection{Documentation%
  \label{documentation}%
}
\begin{enumerate}
    \item The software shall provide a User Manual containing step-by-step instructions for user perspective in Section~\ref{user-characteristics}.
    \item The software shall provide Developer Documentation containing:
    \begin{enumerate}
        \item Agent Framework definitions and example agents
        \item Scenario Language syntax and usage
    \end{enumerate}
\end{enumerate}

%___________________________________________________________________________

\appendix
\pagebreak

\end{document}
